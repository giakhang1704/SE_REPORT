% FORMAT AND PACKAGES
% {
\documentclass[a4paper]{article}
\usepackage{listings}
\usepackage{xcolor}
\usepackage{indentfirst}
\usepackage{subcaption} % thay cho subfig

\usepackage{tcolorbox}
\tcbuselibrary{listingsutf8}


% Configure the listings package for C code
\lstset{
    language=C,
    basicstyle=\ttfamily\small,
    keywordstyle=\color{blue}\bfseries,
    stringstyle=\color{red},
    commentstyle=\color{green!50!black},
    numberstyle=\tiny\color{gray},
    numbers=left,
    stepnumber=1,
    numbersep=5pt,
    showspaces=false,
    showstringspaces=false,
    frame=single,
    breaklines=true,
    breakatwhitespace=true,
    tabsize=4
}
\usepackage{a4wide,amssymb,epsfig,latexsym,multicol,array,hhline,fancyhdr}
\usepackage{vntex}
\usepackage{amsmath}
\usepackage{lastpage}
\usepackage[lined,boxed,commentsnumbered]{algorithm2e}
\usepackage{enumerate}
\usepackage{longtable}
\usepackage{xcolor}
\usepackage{graphicx}							% Standard graphics package
\usepackage{array}
\usepackage{tabularx, caption}
\usepackage{multirow}
\usepackage{multicol}
\usepackage{rotating}
\usepackage{graphics}
\usepackage{geometry}
\usepackage{setspace}
\usepackage{epsfig}
\usepackage{tikz}
\usepackage{xfrac}
\usepackage{bm}
\usepackage{biblatex}
\usepackage[colorlinks]{hyperref}
% \usepackage[acronym,toc]{glossaries}
% \usepackage[symbols,nogroupskip,nonumberlist]{glossaries-extra}
\usepackage[
 sort=none,% no sorting or indexing required
 abbreviations,% create list of abbreviations
 symbols,% create list of symbols
 stylemods,style=list, % set the default glossary style
 nogroupskip, nonumberlist, nomain
]{glossaries-extra}
\usepackage{float}
\usepackage[utf8]{inputenc}
\usepackage[vietnamese]{babel}
\usepackage{array}
\usepackage{booktabs}

% FORMATTING
% {
\DeclareMathOperator{\arccot}{arccot}
\captionsetup[table]{name=Bảng}
\captionsetup[figure]{name=Hình}
\newenvironment{Description}{\list{}{%
    \let\makelabel\descriptionlabel    % this comes from the original description environment
    \setlength{\rightmargin}{\leftmargin}% this comes from the original quote environment
    \setlength{\labelwidth}{0pt}%          this is new
    }}{\endlist}

\addbibresource{citations.bib}
    
\hypersetup{urlcolor=blue,linkcolor=black,citecolor=black,colorlinks=true} 
\usetikzlibrary{arrows,snakes,backgrounds}
\definecolor{mathblue}{RGB}{0,114,188}
% \makeatletter  \def\m@th{\mathsurround\z@\color{mathblue}} \makeatother
% \everymath{\color{mathblue}}
% \setmathfont[Color=000000]{Arial}
%\usepackage{pstcol} 								% PSTricks with the standard color package
\newtheorem{theorem}{{\bf Theorem}}
\newtheorem{property}{{\bf Property}}
\newtheorem{proposition}{{\bf Proposition}}
\newtheorem{corollary}[proposition]{{\bf Corollary}}
\newtheorem{lemma}[proposition]{{\bf Lemma}}

\AtBeginDocument{\renewcommand{\listfigurename}{List of Figures}}
\AtBeginDocument{\renewcommand{\listtablename}{List of Tables}}
\AtBeginDocument{\renewcommand*\contentsname{Mục lục}}
\AtBeginDocument{\renewcommand*\refname{References}}
%\usepackage{fancyhdr}

\setlength{\headheight}{40pt}
\pagestyle{fancy}
\fancyhead{} % clear all header fields
\fancyhead[L]{
 \begin{tabular}{rl}
    \begin{picture}(25,15)(0,0)
    \put(0,-8){\includegraphics[width=10mm, height=10mm]{Image/hcmut.png}}
    %\put(0,-8){\epsfig{width=10mm,figure=hcmut.eps}}
   \end{picture}&
	%\includegraphics[width=8mm, height=8mm]{hcmut.png} & %
	\begin{tabular}{l}
		\textbf{\bf \ttfamily Trường Đại học Bách Khoa - ĐHQG-HCM}\\
		\textbf{\bf \ttfamily Khoa Khoa học và Kỹ thuật Máy tính}
	\end{tabular} 	
 \end{tabular}
}
\fancyhead[R]{
	\begin{tabular}{l}
		\tiny \bf \\
		\tiny \bf 
	\end{tabular}  }
\fancyfoot{} % clear all footer fields
\fancyfoot[L]{\bf \ttfamily Báo cáo Bài tập lớn Công nghệ Phần mềm (CO3001)}
\fancyfoot[R]{\bf \ttfamily Trang {\thepage}/\pageref{LastPage}}
\renewcommand{\headrulewidth}{0.3pt}
\renewcommand{\footrulewidth}{0.3pt}

\setcounter{secnumdepth}{4}
\setcounter{tocdepth}{4}

\makeatletter
\newcounter {subsubsubsection}[subsubsection]
\renewcommand\thesubsubsubsection{\thesubsubsection .\@alph\c@subsubsubsection}
\newcommand\subsubsubsection{\@startsection{subsubsubsection}{4}{\z@}%
                                     {-3.25ex\@plus -1ex \@minus -.2ex}%
                                     {1.5ex \@plus .2ex}%
                                     {\normalfont\normalsize\bfseries}}
\newcommand*\l@subsubsubsection{\@dottedtocline{3}{10.0em}{4.1em}}
\newcommand*{\subsubsubsectionmark}[1]{}
% \def\m@th{\mathsurround\z@\color{mathblue}}
\makeatother
% }
% }

% ACRONYMS & SYMBOLS
% {
% \makeglossaries
\setabbreviationstyle{long-short}
\newabbreviation{ode}{ODE}{(First-Order) Ordinary Differential Equation}
\newabbreviation{ivp}{IVP}{Initial-Value Problem}
\newabbreviation{lte}{LTE}{Local Truncation Error}
\newabbreviation{ds}{DS}{Dynamical System}
\newabbreviation{fig}{Fig.}{Figure}
\newabbreviation{tab}{Tab.}{Table}
\newabbreviation{sys}{Sys.}{System of Equations}
\newabbreviation{eq}{Eq.}{Equation}
\newabbreviation{eg}{e.g.}{For Example}
\newabbreviation{ie}{i.e.}{That Is}
% \glsnoexpandfields
\glsxtrnewsymbol[description = {Set of natural numbers}]{natural}{\ensuremath{\mathbb{N}}}
\glsxtrnewsymbol[description = {Set of real numbers}]{real}{\ensuremath{\mathbb{R}}}
\glsxtrnewsymbol[description = {Set of positive real numbers}]{real_positive}{\ensuremath{\mathbb{R}^+}}

% }


% DOCUMENT
\begin{document}

% TITLE PAGE
\begin{titlepage}
\begin{center}
\Large ĐẠI HỌC QUỐC GIA THÀNH PHỐ HỒ CHÍ MINH \\
\Large TRƯỜNG ĐẠI HỌC BÁCH KHOA \\
\Large KHOA KHOA HỌC VÀ KỸ THUẬT MÁY TÍNH
\end{center}


\begin{figure}[h!]
\begin{center}
\includegraphics[width=4.5cm]{Image/hcmut.png}
\end{center}
\end{figure}

\vspace{0.1cm}


\begin{center}
\begin{tabular}{c}
\multicolumn{1}{l}{\textbf{{\Large Bài tập lớn Công nghệ Phần mềm (CO3001) - HK251}}}\\
~~\\
\hline
\\
\\
\textbf{\textit{{\Large TUTOR SUPPORT SYSTEM AT}}} \\
\\
\textbf{\textit{{\Large HO CHI MINH CITY UNIVERSITY OF TECHNOLOGY}}} \\
\\
\\
\hline
\end{tabular}
\end{center}

\vspace{0.1cm}

\begin{table}[H]
\centering
{\large
\textbf{GVHD: Mai Đức Trung} \\[2pt] % GVHD trên cùng, không có khoảng sau dấu :
   % Tên nhóm to và đậm hơn, nằm dưới GVHD
\begin{tabular}{ll}
\textbf{Tên nhóm: Nhóm 6}&\\   
\textbf{Thành viên:} & \\[2pt]
Nguyễn Đăng Khoa    & 2311614 \\
Hồ Minh Nhiên       & 2312517 \\
Vũ Huy Gia Khang    & 2311486 \\
Nguyễn Văn Hiếu     & 2310967 \\
Lê Thúy Hiền        & 2310990 \\
Hoàng Thị Hằng      & 2310901 \\
Đỗ Quang Long       & 2311896 \\
\end{tabular}
}
\end{table}
\vspace{0.4 cm}
\begin{center}
{\large THÀNH PHỐ HỒ CHÍ MINH, NĂM 2025}
\end{center}
\end{titlepage}

\section*{Danh sách thành viên}
% --- Bảng danh sách thành viên (không dùng caption) ---
\begingroup
\renewcommand{\arraystretch}{2}   % giãn dòng trong ô
\setlength{\tabcolsep}{8pt}       % đệm hai bên cột

\centering
\begin{longtable}{|c|p{4cm}|c|c|}
\hline
\textbf{STT} & \textbf{Họ và tên} & \textbf{MSSV} & \textbf{Tỉ lệ hoàn thành} \\
\hline
\endfirsthead

\hline
\textbf{STT} & \textbf{Họ và tên} & \textbf{MSSV} & \textbf{Tỉ lệ hoàn thành} \\
\hline
\endhead

\hline \multicolumn{4}{r}{\textit{(còn tiếp trang sau)}} \\ \hline
\endfoot

\hline
\endlastfoot

1 & Nguyễn Đăng Khoa  & 2311614 & 100\% \\ \hline
2 & Hồ Minh Nhiên     & 2312517 & 100\% \\ \hline
3 & Vũ Huy Gia Khang  & 2311486 & 100\% \\ \hline
4 & Nguyễn Văn Hiếu   & 2310967 & 100\% \\ \hline
5 & Lê Thúy Hiền      & 2310990 & 100\% \\ \hline
6 & Hoàng Thị Hằng    & 2310901 & 100\% \\ \hline
7 & Đỗ Quang Long     & 2311896 & 100\% \\ \hline

\end{longtable}
\endgroup
% --- Hết bảng ---

\newpage
\tableofcontents
\newpage



\setlength{\parskip}{0.2em}  % khoảng cách nhỏ giữa các đoạn
\setstretch{1.5}


% % %-----------------------TASK 1-----------------------%
\section{Task 1}
\subsection{Phân tích ngữ cảnh}
\subsubsection{Bối cảnh dự án}
Trong môi trường giáo dục đại học hiện nay, việc hỗ trợ sinh viên không chỉ dừng lại ở giảng dạy kiến thức trên lớp mà còn bao gồm tư vấn, kèm cặp và phát triển kỹ năng. Nhằm đáp ứng nhu cầu đó, Trường Đại học Bách Khoa – ĐHQG TP.HCM (HCMUT) đã triển khai chương trình Tutor/Mentor, trong đó các tutor là giảng viên, nghiên cứu sinh hoặc sinh viên năm trên có thành tích học tập tốt, được phân công để đồng hành và hỗ trợ các nhóm sinh viên trong quá trình học tập.  

Để quản lý chương trình hiệu quả hơn, mở rộng quy mô và tránh quản lý thủ công, rườm rà và dễ sai sót ,  nhà trường mong muốn xây dựng một hệ thống phần mềm để quản lý và
vận hành chương trình Tutor một cách hiệu quả. Hệ thống sẽ hỗ trợ quản lý hồ sơ sinh viên và tutor, đăng ký và ghép cặp, tổ chức và quản lý lịch tư vấn, đồng thời cung cấp công cụ thông báo, nhắc nhở và đánh giá. Bên cạnh đó, các khoa, phòng ban có thể khai thác dữ liệu tổng hợp để giám sát chất lượng đào tạo, phân bổ nguồn lực, cộng điểm rèn luyện hoặc xét học bổng cho sinh viên. 


Hệ thống sẽ được tích hợp với các hạ tầng công nghệ của HCMUT như \textbf{HCMUT\_SSO} (đăng nhập tập trung), \textbf{HCMUT\_DATACORE} (đồng bộ dữ liệu cá nhân và phân quyền tự động) và \textbf{HCMUT\_LIBRARY} (truy cập, chia sẻ học liệu chính thống). Việc tích hợp này giúp đảm bảo an toàn, đồng bộ dữ liệu và tạo điều kiện thuận lợi cho sinh viên, tutor cũng như các phòng ban trong quá trình sử dụng.  
Ngoài các chức năng cốt lõi, hệ thống cũng có thể mở rộng với các tính năng nâng cao như:  
Ghép cặp tutor – sinh viên thông minh dựa trên AI,  
     Xây dựng cộng đồng trực tuyến cho tutor và mentee,  
     Tổ chức các chương trình hỗ trợ học thuật và phi học thuật,  
     Cung cấp dịch vụ hỗ trợ học tập cá nhân hóa.  
\newpage
\subsubsection{Relevant stakeholders}

\begin{longtable}{|>{\raggedright\arraybackslash}p{3.5cm}|
                    >{\raggedright\arraybackslash}p{5.5cm}|
                    >{\raggedright\arraybackslash}p{5.5cm}|}
\hline
\textbf{Stakeholders} & \textbf{Roles} & \textbf{Expectations} \\
\hline
Tutor &
- Đăng nhập qua HCMUT\_SSO \newline
- Xem hồ sơ cá nhân \newline
- Thiết lập lịch rảnh \newline
- Mở/Tạo/Hủy lịch hẹn với sinh viên \newline
- Quản lý buổi học trực tuyến/trực tiếp \newline
- Theo dõi và ghi nhận tiến độ học tập sinh viên \newline
- Truy cập, chia sẻ tài liệu qua HCMUT\_LIBRARY \newline
- Nhận phản hồi từ sinh viên &
- Quản lý hồ sơ cá nhân và chuyên môn dễ dàng \newline
- Được bảo mật thông tin \newline
- Lịch rảnh dễ quản lý, linh hoạt \newline
- Hệ thống nhắc lịch tự động \newline
- Công cụ theo dõi tiến độ, xuất báo cáo thuận tiện \newline
- Nhận phản hồi minh bạch sau buổi học \\
\hline
Student &
- Đăng nhập bằng HCMUT\_SSO \newline
- Đăng ký/Hủy đăng ký tham gia chương trình Tutor \newline
- Xem/cập nhật hồ sơ, nhu cầu học tập \newline
- Tìm/chọn tutor hoặc nhận gợi ý \newline
- Đăng ký/Hủy đăng ký lịch hẹn với tutor \newline
- Tham gia buổi học trực tiếp/trực tuyến \newline
- Truy cập tài liệu qua HCMUT\_LIBRARY \newline
- Gửi phản hồi, đánh giá chất lượng buổi học \newline
- Theo dõi tiến độ học tập &
- Đăng nhập an toàn, đồng bộ \newline
- Giao diện dễ dùng, thao tác đơn giản \newline
- Công cụ tìm/gợi ý tutor phù hợp \newline
- Quản lý lịch hẹn dễ dàng, nhắc lịch tự động \newline
- Học online/offline linh hoạt \newline
- Truy cập học liệu chính thống \newline
- Được phản hồi/đánh giá tutor minh bạch \newline
- Công cụ theo dõi tiến độ, kết quả cuối khóa minh bạch \\
\hline
Phòng Đào tạo &
- Quản lý học vụ toàn trường \newline
- Phân bổ nguồn lực \newline
- Nhận báo cáo tổng quan chương trình &
- Báo cáo tổng hợp nhiều định dạng \newline
- Thống kê phân bổ tutor, phòng học \newline
- Phân tích dữ liệu nhiều học kỳ để nhận diện xu hướng môn cần hỗ trợ \\
\hline
Phòng Công tác Sinh viên &
- Theo dõi kết quả tham gia của sinh viên \newline
- Dùng dữ liệu cộng điểm rèn luyện, xét học bổng &
- Báo cáo chi tiết về kết quả tham gia \newline
- Dữ liệu minh bạch, đáng tin cậy \newline
- Tích hợp sẵn, giảm thao tác thủ công \\
\hline
Khoa / Bộ môn &
- Giám sát chất lượng học tập qua dữ liệu đánh giá \newline
- Kiểm tra hiệu quả chương trình theo môn/chuyên ngành \newline
- Đề xuất cải tiến chương trình &
- Dữ liệu thống kê rõ ràng theo từng môn học \newline
- Phân tích tiến bộ sinh viên \newline
- Căn cứ khách quan để ra quyết định \\
\hline
HCMUT\_DATACORE &
- Cung cấp, đồng bộ dữ liệu cá nhân (họ tên, MSSV, khoa, email, tình trạng học tập/giảng dạy) &
- Dữ liệu chính xác, nhất quán \newline
- Tích hợp, tránh nhập liệu thủ công \newline
- Bảo mật dữ liệu theo quy định \\
\hline
HCMUT\_SSO &
- Xác thực tập trung \newline
- Cấp quyền truy cập theo vai trò &
- Đăng nhập một lần (SSO) \newline
- Bảo mật thông tin cá nhân \newline
- Phân quyền chính xác, hạn chế truy cập sai vai trò \\
\hline
HCMUT\_Library &
- Cung cấp học liệu, tài nguyên tích hợp hệ thống &
- Đảm bảo tính chính thống và bản quyền \newline
- Đồng bộ học liệu với môn học/tutor \newline
- Cho phép truy cập/chia sẻ học liệu \newline
- Có thống kê truy cập để cải thiện dịch vụ \\
\hline
Coordinator &
- Quản lý danh sách tutor/sinh viên \newline
- Điều phối khung chương trình chung \newline
\newline
- Gửi báo cáo đến các phòng ban \newline
- Thu thập phản hồi, đề xuất cải tiến &
- Quản lý tập trung giáo trình, slide, đề thi \newline
- Công cụ báo cáo, thống kê giảng dạy \newline
- Đảm bảo tính nhất quán giữa các tutor \newline
- Tiết kiệm thời gian, nâng cao chất lượng môn học \\
\hline
\end{longtable}

\subsubsection{Mục tiêu}
\setlength{\parindent}{2em}  % thụt đầu dòng 2em cho mỗi đoạn

Dự án Tutor/Mentor tại HCMUT được khởi xướng với mục tiêu xây dựng một hệ thống phần mềm hiện đại, đồng bộ và thân thiện, đóng vai trò là nền tảng trung tâm để quản lý, điều phối và nâng cao chất lượng chương trình. Thay vì vận hành rời rạc và thủ công, hệ thống này hướng đến số hóa toàn diện, từ khâu quản lý hồ sơ, đăng ký tham gia cho đến quản lý lịch hẹn, phản hồi và đánh giá. Qua đó, không chỉ giảm tải đáng kể khối lượng công việc hành chính cho cán bộ mà còn mang lại trải nghiệm thuận tiện, minh bạch và chủ động hơn cho sinh viên cũng như đội ngũ Tutor.

Bên cạnh chức năng quản lý cốt lõi, dự án còn nhấn mạnh việc khai thác dữ liệu thông minh để hỗ trợ các đơn vị chức năng như Phòng Đào tạo, Phòng Công tác Sinh viên theo dõi tiến độ, đánh giá hiệu quả và tối ưu hóa nguồn lực. Các dữ liệu này cũng trở thành căn cứ quan trọng để cộng điểm rèn luyện, xét học bổng hay triển khai chính sách hỗ trợ sinh viên phù hợp. Điều này không chỉ giúp cải thiện chất lượng hoạt động Tutor/Mentor mà còn nâng cao toàn diện trải nghiệm học tập và phúc lợi cho người học.

Trên nền tảng dữ liệu thống nhất và công nghệ hiện đại, dự án còn tăng cường việc phát triển một môi trường học tập thông minh và năng động hơn thông qua các tính năng mở rộng như AI gợi ý ghép cặp, cộng đồng học tập trực tuyến, tích hợp các chương trình học thuật và phi học thuật cùng khả năng hỗ trợ cá nhân hóa học tập. Những tính năng này sẽ giúp sinh viên chủ động hơn trong việc học và kết nối. Đây cũng là nền tảng để nhà trường tối ưu và phát triển thêm các dịch vụ học tập số, hướng tới một hệ sinh thái giáo dục số bền vững và linh hoạt, đáp ứng nhu cầu ngày càng đa dạng của người học.

Như vậy, mục tiêu của dự án không chỉ dừng lại ở việc tạo ra một công cụ quản lý đơn thuần mà còn là một bước tiến quan trọng trong chiến lược chuyển đổi số của HCMUT. Hệ thống sẽ trở thành cầu nối giữa sinh viên, tutor và nhà trường, thúc đẩy tương tác, lan tỏa tri thức và định hình một môi trường học tập hiện đại, sáng tạo và lấy người học làm trung tâm.


\subsubsection{Phạm vi}
Trong phạm vi \textit{(scope)} của dự án, hệ thống Tutor Support sẽ tập trung vào các chức năng cốt lõi để hỗ trợ hiệu quả cho cả sinh viên và tutor. Cụ thể, hệ thống cho phép quản lý hồ sơ tutor và sinh viên thông qua việc đồng bộ dữ liệu cá nhân cơ bản (họ tên, MSSV/Mã cán bộ, khoa/chuyên ngành, email học vụ…) từ HCMUT\_DATACORE, giúp giảm thiểu nhập liệu thủ công và đảm bảo tính chính xác. Sinh viên có thể đăng nhập bằng tài khoản SSO để đăng ký tham gia chương trình. Ngoài ra, hệ thống hỗ trợ việc ghép cặp tutor – sinh viên theo hai cách: sinh viên tự chọn hoặc được gợi ý dựa trên nhu cầu chuyên môn cần hỗ trợ và lịch rảnh Sau khi ghép cặp, sinh viên và tutor có thể chủ động đặt lịch, hủy hoặc đổi lịch, đồng thời hệ thống sẽ tự động gửi thông báo và nhắc nhở để cả hai bên có thể nhận được những thay đổi mới nhất. Các buổi gặp có thể diễn ra trực tiếp hoặc trực tuyến. Bên cạnh đó, hệ thống cho phép sinh viên gửi phản hồi, đánh giá chất lượng buổi học, còn tutor có thể ghi nhận tiến bộ của sinh viên; các dữ liệu này sẽ được tổng hợp thành báo cáo cho khoa/bộ môn và các phòng ban liên quan nhằm theo dõi hiệu quả của chương trình.

Về ranh giới \textit{(boundary)} với các hệ thống khác, Tutor Support sẽ giao tiếp trực tiếp với HCMUT\_SSO để thực hiện xác thực đăng nhập, đồng bộ dữ liệu cá nhân từ HCMUT\_DATACORE, và kết nối với HCMUT\_LIBRARY để sinh viên và tutor có thể truy cập vào nguồn tài liệu học thuật. Đây là các tích hợp nhằm đảm bảo tính đồng bộ, chính thống và an toàn dữ liệu. Dự án cũng xác định rõ những chức năng nằm ngoài phạm vi: không xử lý các vấn đề liên quan đến thanh toán học phí, không thay thế hoàn toàn hệ thống LMS của trường, không can thiệp vào việc quản lý chương trình giảng dạy, lịch học hay điểm số. Tutor Support chỉ giữ vai trò hỗ trợ bổ sung, đồng thời tạo giá trị riêng trong việc kết nối sinh viên và tutor một cách thuận tiện hơn.
 

\newpage
\subsection{Yêu cầu chức năng}
\subsubsection{Đối với sinh viên}
\begin{itemize}
    \item Đăng nhập: 
    .\begin{itemize}
    \item Hệ thống phải cho phép sinh viên đăng nhập bằng tài khoản đã được cấp thông qua HCMUT\_SSO.
    \item Hệ thống phải kiểm tra tính hợp lệ của tên đăng nhập và mật khẩu.
    \item Sau khi xác thực thành công, hệ thống phải gán quyền “sinh viên” để truy cập các chức năng phù hợp.
    \item Nếu xác thực thất bại, hệ thống phải hiển thị thông báo lỗi và yêu cầu sinh viên nhập lại thông tin.
\end{itemize}

    \item Đăng xuất: \newline
    Hệ thống phải cho phép sinh viên đăng xuất để kết thúc phiên làm việc hiện tại. Sau khi đăng xuất, hệ thống phải đưa sinh viên về trang đăng nhập và vô hiệu hóa quyền truy cập còn lại.
    
    \item Xem hồ sơ cá nhân: \newline
Hệ thống phải cho phép sinh viên xem thông tin cá nhân của mình được đồng bộ từ HCMUT\_DATACORE, bao gồm họ tên, mã số sinh viên (MSSV), khoa, chuyên ngành, email học vụ và các thông tin liên quan đến học tập.  

    \item Xem tài liệu:\begin{itemize}
        \item Hệ thống phải cho phép sinh viên truy cập và xem các tài liệu học tập được cung cấp từ HCMUT\_LIBRARY.   
        \item Hệ thống phải cho phép sinh viên xem trực tuyến hoặc tải xuống tài liệu để sử dụng.  
    \end{itemize}
    \item Đăng ký Tutor: \newline
    Hệ thống phải cho phép sinh viên chọn tutor phù hợp dựa trên chuyên môn, lĩnh vực hỗ trợ hoặc gợi ý thông minh. Sau khi sinh viên gửi yêu cầu đăng ký, hệ thống phải lưu yêu cầu ở trạng thái “Chờ duyệt” trong 12 giờ và cho phép sinh viên hủy yêu cầu trong thời gian này.  

    \item Đăng kí buổi tư vấn: \newline
    Hệ thống phải cho phép sinh viên chọn và đăng ký buổi tư vấn còn chỗ trống do tutor tổ chức. 
    \item  Hủy đăng kí buổi gặp mặt: \newline   
    Hệ thống phải cho phép sinh viên huỷ các buổi tư vấn và buổi hẹn đã đăng ký trong thời gian cho phép (trước khi buổi gặp mặt bắt đầu).
    \item Đặt lịch hẹn: \newline
    Hệ thống phải cho phép sinh viên đặt lịch hẹn riêng với tutor dựa trên lịch rảnh do tutor cung cấp và nhập nội dung cần được hỗ trợ cho buổi hẹn.  
 
    \item Phản hồi chất lượng buổi học:
    \begin{itemize}
    \item Hệ thống phải cung cấp form phản hồi để sinh viên đánh giá chất lượng buổi học sau mỗi buổi tư vấn.  
        \item Hệ thống phải cho phép sinh viên xem lại lịch sử phản hồi của mình.  
        \item Form phản hồi phải bao gồm:  
        \begin{itemize}
            \item Điểm số đánh giá (1 đến 5).  
            \item Nhận xét ngắn gọn về nội dung buổi học, thái độ và khả năng truyền đạt của tutor.  
        \end{itemize}
    \end{itemize}  
\end{itemize}

\subsubsection{Đối với Tutor}

\begin{itemize}
    \item Đăng nhập:
    \begin{itemize}
        \item Hệ thống phải cho phép tutor đăng nhập bằng tài khoản HCMUT\_SSO.  
        \item Hệ thống phải kiểm tra tính hợp lệ của tên đăng nhập và mật khẩu.  
        \item Sau khi xác thực thành công, hệ thống phải gán quyền “tutor” để truy cập các chức năng tương ứng.  
        \item Nếu xác thực thất bại, hệ thống phải hiển thị thông báo lỗi và yêu cầu tutor nhập lại thông tin.  
    \end{itemize}

    \item Đăng xuất: 
     Hệ thống phải cho phép tutor đăng xuất để kết thúc phiên làm việc hiện tại. Sau khi đăng xuất, hệ thống phải đưa tutor về trang đăng nhập và vô hiệu hóa quyền truy cập còn lại.
    \item Xem hồ sơ cá nhân:
    Hệ thống phải cho phép tutor xem thông tin cá nhân của mình được đồng bộ từ HCMUT\_DATACORE, bao gồm họ tên, mã định danh, email, số điện thoại và thông tin liên hệ khác.   

    \item Thiết lập lịch rảnh: \newline
    Hệ thống phải cho phép tutor thiết lập lịch rảnh để sinh viên tham khảo khi đặt lịch hẹn. Lịch rảnh bao gồm thông tin về ngày, giờ. Khi tutor thay đổi hoặc hủy lịch rảnh, hệ thống phải tự động cập nhật lại.

    \item Tạo buổi tư vấn :  \newline
    Hệ thống cho phép tutor tạo các buổi tư vấn với thông tin về ngày, giờ, chủ đề và hình thức (online hoặc offline). Sau khi tạo, hệ thống phải gửi thông báo buổi tư vấn đến các sinh viên đã chọn tutor. 
    \item Xử lý yêu cầu đặt lịch hẹn:  
Hệ thống phải cho phép tutor xem các yêu cầu đặt buổi gặp từ sinh viên và duyệt hoặc từ chối từng yêu cầu.  

    \item Hủy buổi gặp mặt: \newline
    Hệ thống cho phép tutor hủy các buổi tư vấn đã tạo, buổi hẹn đã duyệt và nhập lý do . 
    \item Xem tài liệu: 
    \begin{itemize}
        \item Hệ thống phải cho phép tutor truy cập và xem các tài liệu học tập được cung cấp từ HCMUT\_LIBRARY.   
        \item Hệ thống phải cho phép tutor xem trực tuyến hoặc tải xuống tài liệu để sử dụng.  
    \end{itemize}
    \item Đăng tải tài liệu: \newline
    Hệ thống phải cho phép tutor đăng tải tài liệu học tập với thông tin cơ bản (file, tiêu đề, mô tả, loại tài liệu). Tài liệu được lưu ở trạng thái “Chờ duyệt” và chỉ hiển thị cho sinh viên sau khi được quản trị viên phê duyệt.
    \item Xóa tài liện: \newline
     Hệ thống phải cho phép tutor xóa các tài liệu đã đăng tải nếu không còn sử dụng hoặc muốn thay thế.
    \item Ghi nhận tiến độ học tập của sinh viên: 
    \begin{itemize}
        \item Hệ thống phải cho phép tutor ghi nhận tiến độ và phản hồi về quá trình học tập của sinh viên.  
        \item Hệ thống phải cho phép lưu trữ dữ liệu này để sử dụng trong các báo cáo thống kê.  
    \end{itemize}
\end{itemize}

\subsubsection{Đối với Điều phối viên – Coordinator}
\begin{itemize}
    \item Quản lý thông tin tutor và sinh viên:\newline
    Hệ thống phải cho phép điều phối viên  xem và quản lý thông tin của các tutor và sinh viên trong chương trình. Điều này bao gồm việc theo dõi các hồ sơ cá nhân, chuyên môn, và nhu cầu hỗ trợ của sinh viên.
     \item Hệ thống phải cho phép điều phối viên đăng nhập/đăng xuất: Gọi API SSO, gán quyền “Coordinator”
    \item Gửi báo cáo cho các phòng ban: \newline
    Hệ thống phải cho phép điều phối viên gửi các báo cáo đã được hệ thống tạo sẵn tới các phòng ban và khoa/bộ môn. Điều phối viên có thể chọn báo cáo, định dạng xuất báo cáo, và phương thức gửi.
    \item Điều phối khung chương trình chung cho các tutor: \newline
    Hệ thống phải cho phép điều phối viên kiểm soát khung chương trình chung cho các tutor, từ đó các tutor thực hiện và đảm bảo tính đồng bộ trong các buổi hẹn. Các buổi tư vấn diễn ra theo một chương trình đã được chuẩn bị sẵn, với các chủ đề, thời gian và tài liệu học tập được thống nhất.
    
\end{itemize}

\subsubsection{Đối với Ban quản lý (Phòng Đào tạo, Phòng Công tác Sinh viên, Khoa/Bộ môn)} 

\begin{itemize}
    \item	Hệ thống phải cung cấp báo cáo tổng quan về chương trình: số lượng Tutor/sinh viên đăng kí tham gia, phòng học sử dụng, số buổi đã tổ chức, tỷ lệ tham gia/hoàn thành,…
    \item  Hỗ thống phải cho phép xuất báo cáo (PDF/Excel/CSV) để lưu trữ hoặc tích hợp hệ thống khác
    \item Hệ thống cho phép gửi báo cáo định kỳ tự động theo tuần/tháng hoặc theo chu kỳ do quản trị viên cấu hình.
    \item Hệ thống cho phép gửi báo cáo thủ công khi cần: điều phối viên là người chính, quản trị viên có thể can thiệp khi cần.
    \item Hệ thống cho phép các phòng ban nhận báo cáo  qua email mà không cần tài khoản đăng nhập hệ thống Tutor.
\end{itemize}

\subsubsection{Đối với HCMUT\_DATACORE} 
\begin{itemize} 
    \item Hệ thống phải cho phép cung cấp dữ liệu cá nhân của sinh viên và tutor (họ tên, MSSV, mã tutor, khoa/chuyên ngành, email học vụ, trạng thái học tập/giảng dạy, \ldots) cho các hệ thống khác để phục vụ quá trình quản lý và học tập.
    \item Hệ thống phải cho phép đồng bộ dữ liệu cá nhân với các hệ thống liên quan để đảm bảo tính chính xác và nhất quán.
    \item Hệ thống phải cho phép chia sẻ dữ liệu người dùng, tình trạng học tập và điểm số giữa các hệ thống HCMUT (DATACORE, SSO và Library) nhằm hỗ trợ tra cứu và quản lý.
\end{itemize} 

\subsubsection{Đối với HCMUT\_SSO} 
\begin{itemize} 
    \item Hệ thống phải cho phép người dùng đăng nhập tập trung một lần để truy cập các ứng dụng và dịch vụ khác của HCMUT mà không cần đăng nhập lại.
    \item Hệ thống phải cho phép người dùng đổi mật khẩu và khôi phục tài khoản để đảm bảo an toàn và bảo mật thông tin cá nhân.
    \item Hệ thống phải cho phép cấp quyền truy cập theo vai trò (sinh viên, tutor, điều phối viên, \ldots) để giới hạn phạm vi chức năng phù hợp với từng loại người dùng.
\end{itemize} 

\subsubsection{Đối với HCMUT\_Library} 
\begin{itemize} 
    \item Hệ thống phải cho phép liên kết các khóa học với tài liệu số (giáo trình, bài giảng, tài liệu tham khảo, \ldots) để sinh viên và tutor có thể truy cập hoặc tải về phục vụ học tập.
    \item Hệ thống phải cho phép tutor đăng tải và xóa tài liệu học tập cho sinh viên thông qua hệ thống thư viện số.
    \item Hệ thống phải cho phép ghi nhận và thống kê lượt truy cập hoặc tải tài liệu để hỗ trợ công tác quản lý và cải thiện chất lượng học liệu.
\end{itemize}

\subsection{Yêu cầu phi chức năng (Non-functional Requirements)}


\subsubsection{Hiệu năng và khả năng mở rộng (Performance \& Scalability)}
\begin{itemize}
    \item \textbf{Phản hồi hệ thống:} Hệ thống phải cung cấp phản hồi trong vòng 3 giây đối với các thao tác tiêu chuẩn (ví dụ: đăng nhập, xem tài liệu , xem hồ sơ cá nhân,...).
    \item \textbf{Xử lý đồng thời:} Hệ thống phải hỗ trợ ít nhất 5,000 sinh viên sử dụng đồng thời mà không gián đoạn dịch vụ.
    \item \textbf{Thời gian phản hồi các thao tác nặng:} Các thao tác như ghép cặp AI, tạo báo cáo, tải tài liệu phải hoàn thành trong $<=$ 2 giây cho 95\% các yêu cầu.
    \item \textbf{Khả năng mở rộng:} Hệ thống có thể mở rộng để phục vụ tối đa 8,000 sinh viên mà không gián đoạn dịch vụ.
\end{itemize}

\subsubsection{Bảo mật và tuân thủ pháp lý (Security \& Compliance)}
\begin{itemize}
    \item \textbf{Mã hóa dữ liệu:} 100\% dữ liệu nhạy cảm (thông tin cá nhân, điểm số, nhật ký học tập) phải được mã hóa theo TLS 1.3 khi truyền tải và AES-256 khi lưu trữ.
    \item \textbf{Xác thực người dùng:} Tích hợp HCMUT\_SSO, chỉ cho phép đăng nhập thông qua Single Sign-On (SSO). Vai trò người dùng được đồng bộ tự động từ hệ thống tập trung.
    \item \textbf{Phân quyền truy cập:} Kiểm soát truy cập theo vai trò (RBAC), quyền truy cập được kiểm tra định kỳ 6 tháng. Chỉ cho phép người dùng truy cập dữ liệu và tài liệu mà họ được phép.
    \item \textbf{Bảo vệ chống tấn công:} Chống DDoS, SQL injection, XSS; kiểm tra bảo mật định kỳ hàng quý.
    \item \textbf{Tuân thủ pháp lý:} Hệ thống phải tuân thủ Luật An ninh Mạng 2018, GDPR và các quy định địa phương liên quan đến dữ liệu cá nhân và giao dịch học thuật.
\end{itemize}

\subsubsection{Tính khả dụng và khôi phục (Availability \& Disaster Recovery)}
\begin{itemize}
    \item \textbf{Tính khả dụng:} Hệ thống phải đảm bảo uptime $>=$ 99\%, tương đương không quá 88 giờ downtime/năm, đặc biệt trong các giai đoạn cao điểm như kỳ thi hoặc thời hạn nộp bài.
    \item \textbf{Khôi phục sau sự cố:} Thời gian khôi phục sau sự cố (RTO) $<=$ 3 giờ, phục hồi dữ liệu trong vòng 24 giờ (RPO $<=$ 24 giờ).
    \item \textbf{Bảo trì:} Thời gian bảo trì $<=$ 5 giờ/tháng, thực hiện trong giờ thấp điểm (0h00–1h00), thông báo trước ít nhất 24 giờ.
\end{itemize}

\subsubsection{Khả năng tương thích (Compatibility)}
\begin{itemize}
    \item Hệ thống tương thích với các trình duyệt phổ biến: Chrome, Firefox, Edge (phiên bản mới nhất).
    \item Hỗ trợ cả thiết bị máy tính và thiết bị di động.
    \item Tích hợp với HCMUT\_DATACORE và HCMUT\_LIBRARY để đảm bảo dữ liệu đồng bộ và chính xác.
\end{itemize}

\subsubsection{Khả năng sử dụng (Usability)}
\begin{itemize}
    \item Giao diện trực quan, song ngữ (Tiếng Việt/Anh), người dùng mới có thể nắm bắt cách sử dụng cơ bản trong khoảng thời gian 10 phút.
    \item Thông báo lỗi rõ ràng, hiển thị song ngữ và kèm hướng dẫn khắc phục.
\end{itemize}

\subsubsection{Khả năng bảo trì (Maintainability)}
\begin{itemize}
    \item Hệ thống module hóa (ví dụ: module quản lý lịch, module tải tài liệu, module ghép cặp AI) để dễ dàng cập nhật mà không ảnh hưởng toàn hệ thống.
    \item Tài liệu kỹ thuật đầy đủ, đảm bảo lập trình viên tương lai có thể hiểu và chỉnh sửa hệ thống dễ dàng.
\end{itemize}

\subsubsection{Hỗ trợ kỹ thuật (Technical Support)}
\begin{itemize}
    \item 100\% yêu cầu hỗ trợ kỹ thuật phải được phản hồi trong vòng 2 giờ làm việc.
    \item Vấn đề phải được giải quyết trong vòng 24 giờ từ khi tiếp nhận.
\end{itemize}
\subsubsection{Độ tin cậy (Reliability)}
\begin{itemize}
    \item \textbf{Tỷ lệ lỗi của người dùng:} Khi thực hiện các thao tác trong hệ thống (ví dụ: đăng nhập, ghép cặp Tutor-Sinh viên, tải tài liệu), tỷ lệ lỗi do người dùng không được vượt quá 5\%.
    \item \textbf{Chống lỗi tạm thời (Fault Tolerance):} Hệ thống phải có cơ chế tự động phục hồi khi gặp các lỗi tạm thời, ví dụ: mất kết nối với HCMUT\_DATACORE hoặc lỗi tải tài liệu. 
    \item \textbf{Thông báo sự cố:} Khi sự cố kéo dài, hệ thống phải thông báo rõ ràng cho người dùng và hướng dẫn các bước khắc phục (ví dụ: thử lại thao tác hoặc liên hệ bộ phận hỗ trợ).
\end{itemize}



\newpage
\subsection{Use-case diagram}
\subsubsection{Các actor của hệ thống}

\begin{table}[H]
\centering
\begin{tabular}{|c|l|p{8cm}|}
\hline
\textbf{STT} & \textbf{Actor} & \textbf{Mô tả} \\ \hline
1 & Tutor & Người hỗ trợ, tạo buổi tư vấn, xử lý lịch hẹn, đăng/xóa tài liệu và ghi nhận tiến độ học tập \\ \hline
2 & Người dùng & Người sử dụng hệ thống (có thể xem tài liệu, hồ sơ cá nhân, đăng nhập/đăng xuất) \\ \hline
3 & Sinh viên & Đăng ký tutor, đăng ký/hủy buổi tư vấn, phản hồi chất lượng buổi học \\ \hline
4 & Điều phối viên & Quản lý, gửi báo cáo, điều phối chương trình tutor chung \\ \hline
5 & HCMUT\_LIBRARY & Hệ thống thư viện trực tuyến \\ \hline
6 & HCMUT\_SSO & Hệ thống xác thực và cấp quyền truy cập \\ \hline
7 & HCMUT\_DATACORE & Hệ thống cung cấp và đồng bộ dữ liệu người dùng \\ \hline
\end{tabular}
\end{table}

\subsubsection{Các nhánh use case chính}

\begin{table}[H]
\centering
\begin{tabular}{|c|l|p{9cm}|}
\hline
\textbf{STT} & \textbf{Use Case Name} & \textbf{Mô tả} \\ \hline
1 & Đăng nhập & Người dùng đăng nhập vào hệ thống \\ \hline
2 & Đăng xuất & Người dùng thoát khỏi hệ thống \\ \hline
3  & Xem tài liệu & Người dùng/tutor xem các tài liệu đã được tải lên \\ \hline
4  & Xem hồ sơ cá nhân & Người dùng xem thông tin cá nhân \\ \hline
5 & Điều phối chương trình chung cho Tutor & Điều phối viên điều phối khung chương trình chung cho Tutor \\ \hline
6 & Gửi báo cáo cho các phòng ban & Điều phối viên gửi báo cáo đến các phòng ban \\ \hline
7 & Đăng ký tutor & Sinh viên đăng ký tutor \\ \hline
8 & Đặt lịch hẹn & Sinh viên gửi yêu cầu đặt lịch với tutor \\ \hline
9 & Phản hồi chất lượng buổi học & Sinh viên phản hồi sau buổi học \\ \hline
10 & Đăng ký buổi tư vấn & Sinh viên đặt lịch buổi tư vấn \\ \hline
11 & Hủy đăng ký buổi gặp mặt & Tutor hủy lịch tư vấn đã tạo \\ \hline
12  & Tạo buổi tư vấn & Tutor tạo lịch tư vấn cho sinh viên \\ \hline
13 & Hủy buổi gặp & Tutor hủy lịch buổi tư vấn đã lên lịch \\ \hline
14  & Ghi nhận tiến độ học tập của sinh viên & Tutor ghi nhận tình hình học tập của sinh viên \\ \hline
15  & Thiết lập lịch rảnh & Tutor thiết lập khung giờ rảnh để sinh viên có thể đặt lịch \\ \hline
16  & Đăng tải tài liệu & Tutor đăng tài liệu hỗ trợ sinh viên \\ \hline
17  & Xóa tài liệu & Tutor xóa tài liệu đã tải lên \\ \hline
18  & Xử lý yêu cầu đặt lịch hẹn & Tutor xử lý và phản hồi các yêu cầu hẹn từ sinh viên \\ \hline




\end{tabular}
\end{table}


\subsubsection{Use-case chi toàn bộ hệ thống}
\begin{figure}[H]
    \centering
    \includegraphics[width=1\linewidth]{Image/UC.drawio.png}
    \caption{Use-case cho toàn bộ hệ thống}
    \label{fig:placeholder}
\end{figure}

Xem rõ hơn \href{https://drive.google.com/file/d/1OBO3KzPEBzUu60808emp18U4jJLK_f-d/view?usp=sharing}{tại đây}.
\newpage

\subsection{Use-case detail/scenario}

\begin{figure}[H]
    \centering
    \includegraphics[width=1\linewidth]{Image/person.png}
    \caption{Nhóm chức năng của người dùng}
    \label{fig:placeholder}
\end{figure}
\subsubsection{Đăng nhập}
\begin{longtable}{|p{3.5cm}|p{11cm}|}
\hline
\textbf{Use Case ID} & UC-01 \\ \hline

\textbf{Use Case Name} & Đăng nhập \\ \hline

\textbf{Actor(s)} & Người dùng (Sinh viên / Tutor) hoặc Điều phối viên \\ \hline

\textbf{Description} & User đăng nhập vào hệ thống thông qua dịch vụ xác thực tập trung HCMUT\_SSO để truy cập các chức năng của hệ thống. \\ \hline

\textbf{Trigger} & User truy cập vào hệ thống Tutor Support System và chưa đăng nhập. \\ \hline

\textbf{Pre–Condition(s)} &
User có tài khoản hợp lệ trong hệ thống HCMUT\_SSO. \newline
Hệ thống HCMUT\_SSO đang hoạt động bình thường. \\ \hline

\textbf{Post–Condition(s)} &
User được xác thực thành công và truy cập vào hệ thống với vai trò tương ứng. \newline
Phiên làm việc của user được khởi tạo. \newline
Thông tin cá nhân cơ bản được đồng bộ từ HCMUT\_DATACORE. \\ \hline

\textbf{Normal Flow} &
1. User truy cập vào trang chủ của hệ thống Tutor Support System. \newline
2. Hệ thống hiển thị trang đăng nhập với nút "Đăng nhập qua HCMUT\_SSO". \newline
3. User nhấn vào nút "Đăng nhập qua HCMUT\_SSO". \newline
4. Hệ thống chuyển hướng user đến trang đăng nhập của HCMUT\_SSO. \newline
5. User nhập thông tin đăng nhập (Tài khoản BKNetID và mật khẩu). \newline
6. HCMUT\_SSO xác thực thông tin đăng nhập. \newline
7. Nếu hợp lệ, HCMUT\_SSO trả về token xác thực cho hệ thống. \newline
8. Hệ thống sử dụng token để truy vấn thông tin cơ bản từ HCMUT\_DATACORE (họ tên, mã định danh, khoa/chuyên ngành, email, vai trò, trạng thái tài khoản). \newline
9. Hệ thống kiểm tra vai trò và trạng thái tài khoản để xác định quyền truy cập. \newline
10. Hệ thống phân quyền và chuyển hướng user đến trang chủ phù hợp với vai trò (Sinh viên / Tutor / Điều phối viên). \\ \hline

\textbf{Alternative Flow} &
\textit{(Không có — chức năng quên mật khẩu đã được chuyển thành Use Case riêng.)} \\ \hline

\textbf{Exception Flow} &
E1. Thông tin đăng nhập không chính xác. \newline
\quad E1.1 HCMUT\_SSO hiển thị thông báo lỗi trực tiếp trên form đăng nhập. \newline
\quad E1.2 Use case quay lại bước 5. \newline\newline

E2. Tài khoản bị khóa / hết hạn / yêu cầu đổi mật khẩu. \newline
\quad E2.1 HCMUT\_SSO hiển thị thông báo lỗi tương ứng trên form đăng nhập. \newline
\quad E2.2 Use case kết thúc không thành công tại SSO. \newline\newline

E3. Không có quyền truy cập hệ thống. \newline
\quad E3.1 Sau khi đăng nhập thành công, hệ thống kiểm tra vai trò và phát hiện user không thuộc nhóm được phép sử dụng hệ thống. \newline
\quad E3.2 Tutor Support System hiển thị thông báo "Bạn không có quyền truy cập hệ thống." \newline
\quad E3.3 Phiên đăng nhập không được tạo; use case kết thúc. \newline\newline

E4. Lỗi kết nối đến HCMUT\_SSO. \newline
\quad E4.1 Hệ thống hiển thị thông báo "Không thể kết nối đến dịch vụ xác thực, vui lòng thử lại sau." \newline
\quad E4.2 Use case kết thúc không thành công. \newline\newline

E5. Lỗi đồng bộ dữ liệu từ HCMUT\_DATACORE. \newline
\quad E5.1 Hệ thống hiển thị thông báo "Không thể tải thông tin người dùng, vui lòng thử lại sau." \newline
\quad E5.2 Hệ thống hủy phiên đăng nhập. \newline
\quad E5.3 Use case kết thúc không thành công. \\ \hline

\end{longtable}

\subsubsection{Đăng xuất}
\begin{longtable}{|p{3.5cm}|p{11cm}|}
\hline
\textbf{Use Case ID} & UC-02 \\ \hline
\textbf{Use Case Name} & Đăng xuất \\ \hline
\textbf{Actor(s)} & Người dùng (Sinh viên / Tutor) và Điều phối viên\\ \hline
\textbf{Description} & User kết thúc phiên làm việc và đăng xuất khỏi hệ thống. \\ \hline
\textbf{Trigger} & User chọn chức năng "Đăng xuất" trong hệ thống. \\ \hline
\textbf{Pre–Condition(s)} &
User đã đăng nhập thành công vào hệ thống. \newline
Phiên làm việc của user đang hoạt động. \\ \hline
\textbf{Post–Condition(s)} &
Phiên làm việc của user được kết thúc. \newline
User được chuyển về trang đăng nhập. \newline
Token xác thực được vô hiệu hóa. \\ \hline
\textbf{Normal Flow} &
1. User nhấn vào biểu tượng tài khoản hoặc menu người dùng. \newline
2. Hệ thống hiển thị menu với tùy chọn "Đăng xuất". \newline
3. User chọn "Đăng xuất". \newline
4. Hệ thống hiển thị hộp thoại xác nhận "Bạn có chắc chắn muốn đăng xuất?". \newline
5. User xác nhận đăng xuất. \newline
6. Hệ thống vô hiệu hóa token xác thực hiện tại. \newline
7. Hệ thống xóa thông tin phiên làm việc. \newline
8. Hệ thống chuyển hướng user về trang đăng nhập. \newline
9. Hệ thống hiển thị thông báo "Đăng xuất thành công". \\ \hline
\textbf{Alternative Flow} &
5a. User chọn "Hủy" trong hộp thoại xác nhận. \newline
5a1. Hệ thống đóng hộp thoại xác nhận. \newline
5a2. User tiếp tục sử dụng hệ thống. \newline
5a3. Use case kết thúc. \\ \hline
\textbf{Exception Flow} &
E1. Lỗi khi vô hiệu hóa token. \newline
\quad E1.1 Hệ thống ghi log lỗi. \newline
\quad E1.2 Hệ thống vẫn xóa thông tin phiên làm việc tại client. \newline
\quad E1.3 Hệ thống chuyển hướng user về trang đăng nhập. \newline\newline
E2. Phiên làm việc đã hết hạn trước khi đăng xuất. \newline
\quad E2.1 Hệ thống phát hiện phiên đã hết hạn. \newline
\quad E2.2 Hệ thống chuyển hướng user về trang đăng nhập. \newline
\quad E2.3 Hệ thống hiển thị thông báo "Phiên làm việc đã hết hạn". \\ \hline
\end{longtable}
\newpage
\subsubsection{Xem tài liệu}

\begin{longtable}{|p{3.5cm}|p{11cm}|}
\hline
\textbf{Use Case ID} & UC-03 \\ \hline
\textbf{Use Case Name} & Xem tài liệu \\ \hline
\textbf{Actor(s)} & User, gồm Sinh viên và Tutor \\ \hline
\textbf{Description} & User truy cập và xem tài liệu học tập được đồng bộ từ hệ thống HCMUT\_LIBRARY. \\ \hline
\textbf{Trigger} & User truy cập vào trang tài liệu của môn học trong hệ thống. \\ \hline
\textbf{Pre–Condition(s)} &
User đã đăng nhập thành công bằng HCMUT\_SSO. \\ \hline

\textbf{Post–Condition(s)} &
Tài liệu được hiển thị thành công cho user. \\ \hline

\textbf{Normal Flow} &
1. User chọn chức năng “Danh sách tài liệu”. \newline
2. Hệ thống gửi yêu cầu truy vấn tài liệu đến HCMUT\_LIBRARY. \newline
3. HCMUT\_LIBRARY trả về danh sách các tài liệu khả dụng. \newline
4. Hệ thống hiển thị danh sách tài liệu (tên, mô tả, loại file, ngày cập nhật). \newline
5. User chọn một tài liệu cụ thể. \newline
6. Hệ thống dẫn đến đường dẫn của tài liệu trên HCMUT\_LIBRARY. \\ \hline
\textbf{Alternative Flow} &
5a. User chọn “Tìm kiếm”. \newline
5a1. User nhập các thông tin vào bộ lọc: tên tài liệu, từ khóa, tác giả, ngày cập nhật,... \newline
5a2. User bấm “Tìm kiếm” hoặc phim Enter. \newline
5a3. Hệ thống hiển thị kết quả phù hợp từ HCMUT\_LIBRARY. \\ \hline
\textbf{Exception Flow} &
2a. Lỗi kết nối tới HCMUT\_LIBRARY. \newline
2a1. Hệ thống hiển thị thông báo “Không thể kết nối tới thư viện, vui lòng thử lại sau”. \newline
2a2. Use case kết thúc không thành công. \newline\newline
5a4. Không tìm thấy tài liệu khả dụng theo tiêu chí tìm kiếm. \newline
5a4.1. Hệ thống hiển thị thông báo “Không tìm thấy tài liệu phù hợp”. \newline
5a4.2. Quay lại bước 5a2 để user nhập lại tiêu chí tìm kiếm. \\ \hline
\end{longtable}
\newpage
\subsubsection{Xem hồ sơ cá nhân}

\begin{longtable}{|p{3.5cm}|p{11cm}|}
\hline
\textbf{Use Case ID} & UC-04 \\ \hline
\textbf{Use Case Name} & Xem hồ sơ cá nhân \\ \hline
\textbf{Actor(s)} & Sinh viên / Tutor / Điều phối viên  \\ \hline
\textbf{Description} & User đăng nhập hệ thống và truy cập mục “Hồ sơ cá nhân” để xem thông tin đã lưu: thông tin cá nhân, liên hệ, tài khoản, lịch sử hoạt động liên quan. \\ \hline
\textbf{Trigger} & Người dùng muốn kiểm tra thông tin của chính mình (nhấn vào biểu tượng “Hồ sơ cá nhân”). \\ \hline
\textbf{Pre–Condition(s)} &
User đã đăng nhập hợp lệ vào hệ thống. \\ \hline
\textbf{Post–Condition(s)} &
Thông tin hồ sơ được hiển thị thành công. \\ \hline
\textbf{Normal Flow} &
1. User truy cập menu “Hồ sơ cá nhân”. \newline
2. Hệ thống truy vấn dữ liệu hồ sơ từ cơ sở dữ liệu từ HCMUT\_DATACORE. \newline
3. Hệ thống hiển thị các thông tin: họ tên, mã định danh, email, số điện thoại, vai trò (SV/Tutor/Điều phối viên), và lịch sử hoạt động gần nhất. \newline
4. Actor có thể chọn xem chi tiết từng phần (ví dụ: thông tin liên hệ, lịch sử đăng nhập, lịch hẹn). \\ \hline
\textbf{Exception Flow} &
E1. Phiên đăng nhập hết hạn. \newline
\quad E1.1 Hệ thống từ chối hiển thị, chuyển đến màn hình đăng nhập. \newline\newline
E2. Lỗi truy vấn cơ sở dữ liệu. \newline
\quad E2.1 Hệ thống hiển thị thông báo “Không thể tải hồ sơ, vui lòng thử lại sau”. \newline
\\ \hline
\end{longtable}


\newpage
\subsubsection{Điều phối khung chương trình chung cho các tutor}

\begin{longtable}{|p{3.5cm}|p{11cm}|}
\hline
\textbf{Use Case ID} & UC-05 \\ \hline
\textbf{Use Case Name} & Điều phối khung chương trình chung cho các tutor \\ \hline
\textbf{Actor(s)} & Điều phối viên (Coordinator) \\ \hline
\textbf{Description} & Điều phối viên soạn thảo và kiểm soát khung chương trình chung cho các tutor, từ đó hướng dẫn các tutor thực hiện và đảm bảo tính đồng bộ trong các buổi tư vấn. \\ \hline
\textbf{Trigger} & Điều phối viên truy cập vào mục ``Quản lý khung chương trình''. \\ \hline
\textbf{Pre--Condition(s)} &
Điều phối viên đã đăng nhập vào hệ thống. \newline
Điều phối viên có quyền quản trị và chỉnh sửa khung chương trình. \\ \hline
\textbf{Post--Condition(s)} &
Khung chương trình  được tạo và lưu thành công; hệ thống gửi thông báo đến các tutor. \\ \hline

\textbf{Normal Flow} &
1. Điều phối viên mở ``Quản lý khung chương trình''. \newline
2. Hệ thống hiển thị danh sách khung chương trình và các tùy chọn. \newline
3. Điều phối viên chọn ``Tạo khung chương trình mới''. \newline
4. Hệ thống hiển thị form  với các trường (chủ đề, thời gian, tài liệu, yêu cầu học tập). \newline
5. Điều phối viên điền thông tin và nhấn``Lưu''. \newline
6. Hệ thống kiểm tra ràng buộc (các trường thông tin bắt buộc, trùng chủ đề) và lưu khung chương trình. \newline
7. Hệ thống gửi thông báo cho các tutor về khung chương trình mới. \\ \hline

\textbf{Alternative Flow(s)} &
3a1. Điều phối viên chọn ``Cập nhật khung chương trình cũ'' và chọn một khung từ danh sách. \newline
3a2. Hệ thống hiển thị form đã điền sẵn. \newline
3a3. Điều phối viên chỉnh sửa thông tin và nhấn ``Lưu''. \newline
3a4. Hệ thống kiểm tra ràng buộc và cập nhật khung chương trình. \newline
3a5. Hệ thống gửi thông báo cho các tutor về khung chương trình đã cập nhật. \newline\newline

3b. Điều phối viên chọn ``Hủy''; hệ thống quay về danh sách khung chương trình. Use case kết thúc không thay đổi dữ liệu.  \newline\newline

6a1. Hệ thống hiển thị lỗi (thiếu dữ liệu / chủ đề đã tồn tại). \newline
6a2. Hệ thống hiển thị lại form thông tin; Điều phối viên chỉnh sửa và quay lại Bước 5.

\\ \hline

\textbf{Exception Flow} &
6e1. Hệ thống thông báo: ``Lưu chương trình thất bại, vui lòng thử lại sau''. \newline
6e2. Hệ thống ghi log lỗi, trạng thái dữ liệu chưa được lưu. \newline
6e3. Use case kết thúc không thành công. \\ \hline
\end{longtable}



\subsubsection{Gửi báo cáo đến các phòng ban}
\begin{longtable}{|p{3.5cm}|p{11cm}|}
\hline
\textbf{Use Case ID} & UC-06 \\ \hline
\textbf{Use Case Name} & Gửi báo cáo cho các phòng ban \\ \hline
\textbf{Actor(s)} & Điều phối viên (Coordinator) \\ \hline
\textbf{Description} & Điều phối viên chọn báo cáo được hệ thống tạo và gửi đến các đơn vị liên quan (Phòng Đào tạo, Phòng CTSV, Khoa/Bộ môn) dưới định dạng phù hợp. \\ \hline
\textbf{Trigger} & Điều phối viên mở mục “Báo cáo” và chọn gửi báo cáo đã có cho các phòng ban. \\ \hline
\textbf{Pre–Condition(s)} &
Điều phối viên đã đăng nhập và có quyền gửi báo cáo. \newline
Báo cáo đã được tạo/sinh sẵn (theo kỳ, theo môn, theo chương trình). \\ \hline
\textbf{Post–Condition(s)} &
Báo cáo được gửi thành công tới phòng ban đã chọn. \newline
Hệ thống ghi nhận nhật ký gửi (thời gian, người gửi, nơi nhận, định dạng). \\ \hline
\textbf{Normal Flow} &
1. Điều phối viên vào mục “Báo cáo” và chọn một báo cáo có sẵn. \newline
2. Hệ thống hiển thị thông tin báo cáo (tên, kỳ, phạm vi, bản xem trước). \newline
3. Điều phối viên chọn nơi nhận: Phòng Đào tạo, Phòng CTSV, Khoa/Bộ môn (có thể chọn nhiều). \newline
4. Điều phối viên chọn định dạng xuất (PDF/Excel) và kênh gửi (email/link nội bộ). \newline
5. Điều phối viên nhấn “Gửi”. \newline
6. Hệ thống tạo file (nếu cần), gửi tới nơi nhận đã chọn, ghi audit log. \newline
7. Hệ thống hiển thị “Gửi báo cáo thành công” và cung cấp mã tra cứu. \\ \hline
\textbf{Alternative Flow} &
4a. Điều phối viên thêm ghi chú kèm báo cáo. \newline
4a1. Hệ thống đính kèm ghi chú vào email/thông điệp gửi. \newline
4a2. Quay lại bước 5. \newline\newline
3b. Điều phối viên chọn lịch gửi định kỳ (hàng tháng/quý). \newline
3b1. Hệ thống lưu lịch và sẽ tự động gửi theo chu kỳ. \\ \hline
\textbf{Exception Flow} &
2a. Không tồn tại báo cáo cho phạm vi đã chọn. \newline
2a1. Hệ thống thông báo “Chưa có báo cáo phù hợp”, gợi ý tạo báo cáo trước. \newline
2a2. Use case kết thúc. \newline\newline
5a. Lỗi dịch vụ gửi (email/notification). \newline
5a1. Hệ thống hiển thị “Gửi thất bại”, ghi log lỗi và giữ báo cáo ở trạng thái chưa gửi. \newline
5a2. Điều phối viên có thể thử gửi lại hoặc tải file về để gửi thủ công. \\ \hline
\end{longtable}




\newpage
\begin{figure}[h!]
  \centering
  \includegraphics[width=0.8\textwidth]{Image/sinhvien.jpg}
  \caption{Nhóm chức năng của sinh viên}
\end{figure}
\subsubsection{Đăng ký Tutor}
\begin{longtable}{|p{3.5cm}|p{11cm}|}
\hline
\textbf{Use Case ID} & UC-07 \\ \hline
\textbf{Use Case Name} & Đăng ký Tutor \\ \hline
\textbf{Actor(s)} & Sinh viên \\ \hline
\textbf{Description} & Cho phép sinh viên đăng ký tham gia chương trình Tutor/Mentor và chọn Tutor phù hợp. \\ \hline
\textbf{Trigger} & Sinh viên truy cập hệ thống và chọn chức năng “Đăng ký Tutor”. \\ \hline
\textbf{Pre–Condition(s)} & - Sinh viên đăng nhập thành công. \newline - Chương trình Tutor đang mở đăng ký. \\ \hline
\textbf{Post–Condition(s)} & - Yêu cầu đăng ký tutor được lưu và chờ duyệt trong 12 giờ. \newline - Sau khi duyệt thành công, sinh viên sẽ được gắn cặp với tutor đã đăng ký. Nếu như hủy sẽ bắt đầu lại. \\ \hline
\textbf{Normal Flow} & 
1. Sinh viên vào chức năng Đăng ký Tutor. \newline
2. Hệ thống hiển thị giao diện cho phép sinh viên chọn lĩnh vực/môn học cần hỗ trợ. \newline
3. Sinh viên điền thông tin và bấm tiếp theo. \newline
4. Hệ thống hiển thị danh sách tutor phù hợp. \newline
5. Sinh viên chọn Tutor hoặc chọn gợi ý Tutor thông minh. \newline
6. Sinh viên bấm xác nhận đăng ký. \newline
7. Hệ thống lưu yêu cầu đăng ký với trạng thái Chờ duyệt (12h). \newline
8. Sau 12h, hệ thống xác nhận đăng ký thành công. \\ \hline
\textbf{Alternative Flow} & 
3a. Sinh viên không điền thông tin nhưng bấm tiếp theo. \newline
3a.1. Hệ thống thông báo lỗi yêu cầu nhập lại. \newline
3a.2. Quay lại bước 2. \newline
4a. Không có tutor phù hợp. \newline
4a.1. Hệ thống thông báo và đưa lựa chọn “Đăng ký tutor khác”. \newline
4a.2. Nếu sinh viên đồng ý, hệ thống hiển thị toàn bộ tutor còn slot đăng ký, quay về bước 5. \newline
4a.3. Nếu sinh viên không đồng ý, hệ thống quay lại màn hình chính. Kết thúc usecase. \newline
6a. Sinh viên không xác nhận đăng ký, chọn kết thúc. \newline
6a.1. Hệ thống quay về màn hình chính. Kết thúc usecase. \newline
6b. Sinh viên không xác nhận đăng ký, chọn sửa đổi. \newline
6b.1. Quay về bước 2. \newline
8a. Sinh viên nhấn hủy đăng ký trong khi chờ duyệt. \newline
8a.1. Hệ thống hiển thị hủy thành công, đưa lựa chọn sinh viên có muốn đăng ký mới. \newline
8a.2. Nếu sinh viên đồng ý quay về bước 2. \newline
8a.3. Nếu sinh viên không đồng ý quay về màn hình chính, kết thúc usecase. \\ \hline
\textbf{Exception Flow} & 
- Lỗi mạng trong quá trình đăng ký → Hệ thống hiển thị thông báo lỗi mạng, yêu cầu sinh viên thực hiện lại sau. \newline
- Đợt đăng ký Tutor đã kết thúc → Hệ thống hiển thị thông báo “Đợt đăng ký Tutor đã kết thúc” → Kết thúc Use Case. \newline
- Lỗi lưu dữ liệu đăng ký → Hệ thống hiển thị thông báo “Không lưu được dữ liệu, vui lòng thử lại” → Quay về bước 6 Normal Flow. \\ \hline
\end{longtable}

%====================================================
\subsubsection{Đặt lịch hẹn}
\begin{longtable}{|p{3.5cm}|p{11cm}|}
\hline
\textbf{Use Case ID} & UC-08 \\ \hline
\textbf{Use Case Name} & Đặt lịch hẹn \\ \hline
\textbf{Actor(s)} & Sinh viên, Tutor \\ \hline
\textbf{Description} & Cho phép sinh viên đặt lịch hẹn riêng với tutor dựa trên lịch rảnh do tutor cung cấp và nhập nội dung cần được hỗ trợ cho buổi hẹn. \\ \hline
\textbf{Trigger} & Sinh viên chọn chức năng “Đặt lịch hẹn” trong hệ thống. \\ \hline
\textbf{Pre–Condition(s)} & - Sinh viên đã đăng nhập thành công. \newline - Sinh viên đã chọn tutor. \newline - Tutor đã cung cấp lịch rảnh. \\ \hline
\textbf{Post–Condition(s)} & - Yêu cầu đặt lịch hẹn được lưu ở trạng thái chờ duyệt. \newline - Tutor nhận thông báo về yêu cầu. \\ \hline
\textbf{Normal Flow} & 
1. Sinh viên truy cập chức năng “Đặt lịch hẹn”. \newline
2. Hệ thống hiển thị lịch rảnh của tutor. \newline
3. Sinh viên chọn khoảng thời gian phù hợp và nhập nội dung cần hỗ trợ. \newline
4. Sinh viên xác nhận đặt lịch. \newline
5. Hệ thống lưu yêu cầu và gửi thông báo cho tutor. \\ \hline
\textbf{Alternative Flow} & 
3a. Nếu sinh viên không chọn thời gian hoặc không nhập nội dung: \newline
3a1. Hệ thống thông báo lỗi và yêu cầu nhập lại. \newline
3a2. Quay lại bước 3. \newline
3b. Nếu tutor không còn lịch rảnh: \newline
3b1. Hệ thống thông báo Tutor bận. \newline
3b2. Hệ thống quay về màn hình chính. Kết thúc usecase. \\ \hline
\textbf{Exception Flow} & 
- Lỗi hệ thống hoặc mất kết nối → Hệ thống thông báo lỗi và yêu cầu sinh viên thử lại. \\ \hline
\end{longtable}

%====================================================
\subsubsection{Phản hồi chất lượng buổi học}
\begin{longtable}{|p{3.5cm}|p{11cm}|}
\hline
\textbf{Use Case ID} & UC-09 \\ \hline
\textbf{Use Case Name} & Phản hồi chất lượng buổi học \\ \hline
\textbf{Actor(s)} & Sinh viên (Student) \\ \hline
\textbf{Description} & Sinh viên muốn gửi phản hồi về chất lượng buổi học hoặc chất lượng hỗ trợ của Tutor. \\ \hline
\textbf{Trigger} & Sau khi kết thúc buổi học, sinh viên chọn mục “Phản hồi chất lượng”. \\ \hline
\textbf{Pre–Condition(s)} & Sinh viên đã đăng nhập vào hệ thống. \newline Sinh viên đã tham gia ít nhất một buổi học với Tutor. \\ \hline
\textbf{Post–Condition(s)} & Phản hồi được lưu trữ thành công trong hệ thống. \newline Hệ thống gửi thông báo đến Tutor và ghi nhận phản hồi cho báo cáo tổng hợp. \\ \hline
\textbf{Normal Flow} & 
1. Sinh viên truy cập mục “Phản hồi chất lượng”. \newline
2. Hệ thống hiển thị form phản hồi (mức độ hài lòng, nội dung chi tiết). \newline
3. Sinh viên nhập nội dung phản hồi và chọn mức độ đánh giá. \newline
4. Sinh viên nhấn “Gửi”. \newline
5. Hệ thống kiểm tra hợp lệ, đủ thông tin cần nhập \newline
6. Hệ thống lưu phản hồi vào cơ sở dữ liệu. \newline
7. Hệ thống gửi thông báo xác nhận cho sinh viên. \\ \hline
\textbf{Alternative Flow} & 
5a.1 Hệ thống kiểm tra thấy inh viên không nhập đủ nội dung phản hồi. \newline
5a1. Hệ thống hiển thị thông báo: “Vui lòng nhập nội dung phản hồi trước khi gửi”. \newline
5a2. Quay lại bước 2.\newline
4a. Sinh viên chọn “Hủy” thay vì gửi phản hồi. \newline
4a1. Hệ thống hủy thao tác và không lưu thông tin. \newline
4a2. Use case kết thúc sớm. \\ \hline
\textbf{Exception Flow} & 
6a.   Hệ thống gặp lỗi cơ sở dữ liệu. \newline
6a1. Hệ thống hiển thị thông báo: “Gửi phản hồi thất bại, vui lòng thử lại sau hoặc liên hệ bộ phận kỹ thuật”. \newline
6a2. Hệ thống ghi log lỗi và giữ trạng thái chưa gửi phản hồi. \newline
6a3. Use case kết thúc không thành công. \\ \hline
\end{longtable}

%====================================================
\subsubsection{Đăng ký buổi tư vấn}
\begin{longtable}{|p{3.5cm}|p{11cm}|}
\hline
\textbf{Use Case ID} & UC-10 \\ \hline
\textbf{Use Case Name} & Đăng ký buổi tư vấn \\ \hline
\textbf{Description} & Cho phép sinh viên đăng ký buổi tư vấn với tutor đã chọn trước đó. \\ \hline
\textbf{Trigger} & Sinh viên chọn chức năng “Đăng ký buổi tư vấn” trong hệ thống. \\ \hline
\textbf{Primary Actor} & Sinh viên \\ \hline
\textbf{Secondary Actor} & Tutor \\ \hline
\textbf{Pre-condition} & - Sinh viên đã đăng nhập thành công. \newline - Sinh viên đã chọn tutor. \newline - Tutor đã mở lịch hẹn. \\ \hline
\textbf{Post-condition} & - Sinh viên đã đăng ký thành công buổi tư vấn với tutor. \newline - Hệ thống giảm slot còn lại. \newline - Tutor và sinh viên đều nhận thông báo xác nhận. \\ \hline
\textbf{Normal flow} & 
1. Sinh viên chọn chức năng “Đăng ký buổi tư vấn”. \newline
2. Hệ thống hiển thị danh sách buổi tư vấn còn slot. \newline
3. Sinh viên chọn buổi tư vấn phù hợp. \newline
4. Hệ thống ghi nhận đăng ký, hiển thị thông báo "Đăng ký thành công" và gửi xác nhận. \newline
5. Hệ thống tự động cập nhật số lượng sinh viên tham gia. \\ \hline
\textbf{Alternative flow} & 
2a. Không có buổi tư vấn còn slot. \newline
- Hệ thống hiển thị “Không có sẵn buổi tư vấn” và quay về màn hình chính. \\ \hline
\textbf{Exception flow} & 
- Lỗi hệ thống hoặc mất kết nối → Hệ thống hiển thị “Không thể đăng ký, vui lòng thử lại sau”. \\ \hline
\end{longtable}

%====================================================
\subsubsection{Hủy đăng ký buổi gặp mặt}
\begin{longtable}{|p{3.5cm}|p{10cm}|}
\hline
\textbf{Use Case ID} & UC-11 \\ \hline
\textbf{Use Case Name} & Hủy đăng ký buổi gặp mặt \\ \hline
\textbf{Description} & Cho phép sinh viên hủy buổi tư vấn hoặc lịch hẹn đã đăng ký với tutor khi có thay đổi trong lịch trình hoặc không thể tham gia. \\ \hline
\textbf{Trigger} & Sinh viên chọn chức năng “Hủy đăng ký buổi gặp mặt”. \\ \hline
\textbf{Primary Actor} & Sinh viên \\ \hline
\textbf{Secondary Actor} & Tutor \\ \hline
\textbf{Pre-condition} & - Sinh viên đã đăng nhập. \newline - Sinh viên đã đăng ký buổi gặp mặt. \newline - Buổi gặp mặt chưa bắt đầu. \\ \hline
\textbf{Post-condition} & - Hủy thành công, hệ thống cập nhật slot và thêm lại khung giờ vào lịch rảnh của tutor. \newline - Sinh viên nhận thông báo hủy. \\ \hline
\textbf{Normal flow} & 
1. Sinh viên chọn “Buổi gặp mặt của tôi”. \newline
2. Hệ thống hiển thị danh sách buổi gặp mặt đã đăng ký. \newline
3. Sinh viên chọn buổi muốn hủy. \newline
4. Hệ thống yêu cầu sinh viên nhập lý do hủy. \newline
5. Sinh viên nhập lý do. \newline
6. Hệ thống yêu cầu xác nhận hủy. \newline
7. Sinh viên xác nhận. \newline
8. Hệ thống kiểm tra thời gian và hủy buổi gặp mặt. \newline
9. Hệ thống gửi thông báo “Hủy thành công”. \\ \hline
\textbf{Alternative flow} & 
2a. Không có buổi gặp mặt đã đăng ký. Hệ thống hiển thị “Không có buổi gặp mặt nào”. \newline
6a. Sinh viên hủy thao tác hủy. Hệ thống quay lại danh sách buổi gặp mặt. \\ \hline % <--- ĐÃ SỬA: Thêm xuống dòng ở đây
\textbf{Exception flow} & 
- Lỗi hệ thống hoặc mất kết nối. Hệ thống hiển thị “Không thể hủy, vui lòng thử lại sau”. \\ \hline
\end{longtable}
%====================================================


\subsubsection{Xem danh sách buổi gặp mặt của sinh viên}
\begin{longtable}{|p{3.5cm}|p{11cm}|}
\hline
\textbf{Use Case ID} & UC-12 \\ \hline
\textbf{Use Case Name} & Xem danh sách buổi gặp mặt của sinh viên \\ \hline
\textbf{Description} & Cho phép sinh viên xem danh sách các buổi gặp mặt (sessions) mà mình đã đặt lịch hẹn, bao gồm thông tin giảng viên, thời gian, trạng thái buổi và chủ đề tư vấn. \\ \hline
\textbf{Trigger} & Sinh viên chọn chức năng “Xem danh sách buổi gặp” trong hệ thống. \\ \hline
\textbf{Primary Actor} & Sinh viên \\ \hline
\textbf{Secondary Actor} & Tutor \\ \hline
\textbf{Pre-condition} & - Sinh viên đã đăng nhập thành công. \newline - Hệ thống có ít nhất một buổi hẹn đã đăng ký thành công. \\ \hline
\textbf{Post-condition} & - Danh sách buổi gặp được hiển thị đầy đủ. \newline - Sinh viên có thể thực hiện hành động liên quan (xem chi tiết, hủy nếu cần). \\ \hline
\textbf{Normal flow} & 
1. Sinh viên truy cập chức năng “Xem danh sách buổi gặp”. \newline
2. Hệ thống hiển thị danh sách các buổi gặp (thời gian, giảng viên, trạng thái, chủ đề). \newline
3. Sinh viên chọn một buổi để xem chi tiết. \\ \hline
\textbf{Alternative flow} & 
2a. Không có buổi gặp nào. Hệ thống thông báo “Danh sách trống” và quay về màn hình chính. \newline
3a. Sinh viên hủy thao tác xem chi tiết. Hệ thống quay lại danh sách buổi gặp. \\ \hline
\textbf{Exception flow} & 
- Lỗi hệ thống hoặc mất kết nối. Hệ thống thông báo lỗi và yêu cầu sinh viên thử lại sau. \\ \hline
\end{longtable}


\newpage
\begin{figure}[H]
    \centering
    \includegraphics[width=0.9\linewidth]{Image/tt.drawio.png}
    \caption{Nhóm chức năng của Tutor}
    \label{fig:placeholder}
\end{figure}
% ================== USE CASE 12 ==================
\subsubsection{Tạo buổi tư vấn}
\begin{longtable}{|p{3.5cm}|p{11cm}|}
\hline
\textbf{Use Case ID} & UC-12 \\ \hline
\textbf{Use Case Name} & Tạo buổi tư vấn \\ \hline
\textbf{Actor(s)} & Tutor \\ \hline
\textbf{Description} & 
Tutor muốn tạo một buổi tư vấn  để sinh viên có thể đăng ký tham gia sau đó. Buổi tư vấn có thể nhằm hỗ trợ học tập, hướng nghiệp hoặc phát triển kỹ năng mềm. \\ \hline

\textbf{Trigger} & 
Tutor muốn tạo mới một buổi tư vấn và nhấn vào nút “Tạo buổi tư vấn”. \\ \hline

\textbf{Pre–Condition(s)} &   
- Tutor đã đăng nhập thành công vào hệ thống. \newline
- Kết nối mạng ổn định và hệ thống đang hoạt động bình thường. \\ \hline

\textbf{Post–Condition(s)} & 
Buổi tư vấn được tạo thành công và hiển thị trong danh sách buổi tư vấn mở cho sinh viên đăng ký.  
Hệ thống gửi thông báo xác nhận cho Tutor.  
Lịch rảnh của Tutor được cập nhật tương ứng. \\ \hline

\textbf{Normal Flow} & 

    1. Tutor bấm vào nút “Tạo buổi tư vấn”.\newline
    2. Hệ thống hiển thị form nhập thông tin buổi tư vấn (ngày/giờ, tiêu đề, mô tả nội dung, hình thức tổ chức — online hoặc trực tiếp, và số lượng sinh viên tối đa nếu có).\newline
    3. Tutor nhập đầy đủ thông tin và bấm “Xác nhận”.\newline
    4. Hệ thống kiểm tra xung đột thời gian với lịch rảnh và các buổi tư vấn/lịch hẹn khác.\newline
    5. Hệ thống tạo buổi tư vấn mới, lưu vào cơ sở dữ liệu và cập nhật lịch rảnh của Tutor.\newline
    6. Hệ thống gửi thông báo xác nhận cho Tutor và hiển thị buổi tư vấn trong danh sách để sinh viên có thể đăng ký.\newline
 \\ \hline

\textbf{Alternative Flow} & 
\begin{enumerate}
    \item[4a.] Hệ thống phát hiện xung đột thời gian (trùng với buổi khác hoặc thời gian không hợp lệ như đã qua).
    \begin{enumerate}
        \item[4a1.] Hệ thống hiển thị thông báo lỗi: “Thời gian không hợp lệ hoặc bị trùng với lịch khác”.
        \item[4a2.] Giao diện quay lại bước 3 để Tutor chỉnh sửa.
    \end{enumerate}
        \end{enumerate}
     

\begin{enumerate}
    \item[3a.] Tutor hủy thao tác.
    \begin{enumerate}
        \item[3a1.] Trong quá trình nhập thông tin, Tutor chọn “Hủy thao tác”.
        \item[3a2.] Hệ thống không tạo buổi tư vấn mới.
        \item[3a3.] Use case kết thúc sớm.
           \end{enumerate}
     
   
\end{enumerate} \\ \hline

\textbf{Exception Flow} & 
- Mất kết nối mạng trong quá trình thao tác → Hệ thống hiển thị thông báo lỗi kết nối. \newline
- Lỗi hệ thống (server không phản hồi) → Hệ thống thông báo “Có lỗi xảy ra, vui lòng thử lại sau”. 
     \begin{enumerate}
    \item[5a.] Sau khi kiểm tra thời gian và lưu dữ liệu (bước 5), hệ thống gặp lỗi cơ sở dữ liệu (DB error).
    \begin{enumerate}
        \item[5a1.] Hệ thống hiển thị thông báo: “Tạo buổi tư vấn thất bại, vui lòng thử lại sau hoặc liên hệ bộ phận kỹ thuật”.
        \item[5a2.] Hệ thống ghi log lỗi và không tạo buổi tư vấn.
        \item[5a3.] Use case kết thúc không thành công.
    \end{enumerate}
\end{enumerate} \\ \hline

\end{longtable}

% ================== USE CASE 13 ==================
\subsubsection{Hủy buổi gặp}
\begin{longtable}{|p{3.5cm}|p{11cm}|}
\hline
\textbf{Use Case ID} & UC-13 \\ \hline
\textbf{Use Case Name} & Hủy buổi gặp \\ \hline
\textbf{Actor(s)} & Tutor \\ \hline
\textbf{Description} & 
Tutor muốn hủy một buổi gặp đã được lên lịch với sinh viên (lịch hẹn hoặc buổi tư vấn). \\ \hline
\textbf{Trigger} & 
Tutor nhấn vào nút “Hủy lịch” để thực hiện hủy một buổi hẹn đã có. \\ \hline
\textbf{Pre–Condition(s)} & 
Tutor đã đăng nhập vào hệ thống.  
Tutor đã thiết lập khoảng thời gian rảnh.  
Sinh viên đã đăng ký chương trình với Tutor và có thể đặt lịch. \\ \hline
\textbf{Post–Condition(s)} & 
Buổi gặp được hủy thành công.  
Hệ thống gửi thông báo tự động cho Tutor và sinh viên.  
Hệ thống cập nhật lại lịch rảnh của Tutor. \\ \hline

\textbf{Normal Flow} & 

    1. Tutor bấm vào nút “Hủy lịch”.\newline
    2. Hệ thống hiển thị danh sách buổi gặp đã lên lịch.\newline
    3. Tutor chọn một buổi gặp cụ thể.\newline
    4. Hệ thống hiển thị form nhập lý do hủy.\newline
    5. Tutor nhập lý do\newline
    6. Tutor bấm “Xác nhận”. \newline
     7. Hệ thống kiểm tra Tutor đã nhập lý do.\newline
     8. Hệ thống cập nhật lịch rảnh và lịch hẹn của Tutor.\newline
    9. Hệ thống gửi thông báo thành công.\newline
 \\ \hline

\textbf{Alternative Flow} & 
\begin{enumerate}
    \item[2a.] Hệ thống kiểm tra và phát hiện không có buổi gặp nào đã được lên lịch.
    \begin{enumerate}
        \item[2a1.] Hệ thống hiển thị thông báo: “Hiện tại không có buổi gặp nào để hủy”.
        \item[2a2.] Use case kết thúc sớm.
    \end{enumerate}

    \item[6a.] Tutor hủy thao tác.
    \begin{enumerate}
        \item[6a1.] Trong quá trình nhập lý do, Tutor chọn “Hủy thao tác”.
        \item[6a2.] Hệ thống giữ nguyên lịch hẹn cũ.
        \item[6a3.] Use case kết thúc sớm.
    \end{enumerate}

    \item[7a.] Tutor bỏ trống lý do hủy.
    \begin{enumerate}
        \item[7a1.] Hệ thống hiển thị thông báo: “Vui lòng nhập lý do hủy”.
        \item[7a2.] Giao diện hệ thống trở lại bước 4.
    \end{enumerate}

\end{enumerate} \\ \hline

\textbf{Exception Flow} & 

- Mất kết nối mạng trong quá trình thao tác → Hệ thống hiển thị thông báo lỗi kết nối. \newline
- Lỗi hệ thống (server không phản hồi) → Hệ thống thông báo “Có lỗi xảy ra, vui lòng thử lại sau”. 
\begin{enumerate}
    \item[8a.] Hệ thống cố gắng cập nhật nhưng gặp lỗi cơ sở dữ liệu.
    \begin{enumerate}
        \item[8a1.] Hệ thống hiển thị thông báo: “Cập nhật thất bại, vui lòng thử lại sau hoặc liên hệ bộ phận kỹ thuật”.
        \item[8a2.] Hệ thống ghi log lỗi và giữ nguyên lịch hẹn cũ.
        \item[8a3.] Use case kết thúc không thành công.
    \end{enumerate}
\end{enumerate} \\ \hline
\end{longtable}

% ================== USE CASE 14 ==================
\subsubsection{Ghi nhận tiến độ học tập của sinh viên}
\begin{longtable}{|p{3.5cm}|p{11cm}|}
\hline
\textbf{Use Case ID} & UC-14 \\ \hline
\textbf{Use Case Name} & Ghi nhận tiến độ học tập của sinh viên \\ \hline
\textbf{Actor(s)} & Tutor \\ \hline
\textbf{Description} & Tutor muốn ghi nhận và cập nhật tiến độ học tập của sinh viên dựa trên các buổi tư vấn hoặc hỗ trợ, để theo dõi hiệu quả học tập và tổng hợp báo cáo gửi cho phòng/ban. \\ \hline
\textbf{Trigger} & Tutor muốn ghi nhận tiến độ học tập của sinh viên sau một buổi tư vấn. Tutor nhấn vào nút “Ghi Nhận Tiến Độ”. \\ \hline
\textbf{Pre–Condition(s)} & Tutor đã đăng nhập vào hệ thống. Tutor đã thực hiện ít nhất một buổi tư vấn/hỗ trợ với sinh viên. Dữ liệu sinh viên (họ tên, MSSV, trạng thái học tập) đã được đồng bộ từ HCMUT\_DATACORE. \\ \hline
\textbf{Post–Condition(s)} & Tiến độ học tập của sinh viên được cập nhật thành công trong hệ thống. Hệ thống gửi thông báo cho tutor và sinh viên . \\ \hline
\textbf{Normal Flow} & 
\begin{enumerate}
    \item Tutor truy cập mục “Ghi Nhận Tiến Độ” ở trang chủ.
    \item Hệ thống hiển thị danh sách sinh viên đã được ghép đôi với Tutor và các buổi tư vấn liên quan.
    \item Tutor chọn một sinh viên và buổi tư vấn cụ thể.
    \item Hệ thống hiển thị form nhập tiến độ (nội dung học, đánh giá, ghi chú).
    \item Tutor nhập thông tin tiến độ (bắt buộc: nội dung học, đánh giá; tùy chọn: ghi chú) và bấm “Xác nhận”.
    \item Hệ thống kiểm tra tính hợp lệ của dữ liệu.
    \item Hệ thống lưu thông tin tiến độ.
    \item Hệ thống gửi thông báo xác nhận cho Tutorvà sih viên.
\end{enumerate} \\ \hline
\textbf{Alternative Flow} & 
\begin{enumerate}
    \item[6a.] Hệ thống kiểm tra tính hợp lệ của dữ liệu và phát hiện trường bắt buộc bị bỏ trống.
    \begin{enumerate}
        \item[6a1.] Hệ thống hiển thị thông báo: “Vui lòng điền đầy đủ thông tin bắt buộc”.
        \item[6a2.] Giao diện hệ thống trở lại bước 5.
    \end{enumerate}
    \item[5a.] Trong quá trình nhập thông tin, Tutor chọn “Hủy ghi nhận”.
    \begin{enumerate}
        \item[5a1.] Hệ thống quay lại màn hình trang chủ.
        \item[5a2.] Use case kết thúc sớm.
    \end{enumerate}
\end{enumerate} \\ \hline
\textbf{Exception Flow} & 
\begin{enumerate}
    \item[7a.] Sau khi kiểm tra dữ liệu (bước 6) và cố gắng lưu/đồng bộ (bước 7), hệ thống mất kết nối mạng.
    \begin{enumerate}
        \item[7a1.] Hệ thống hiển thị thông báo: “Không thể kết nối. Vui lòng thử lại sau”.
        \item[7a2.] Hệ thống ghi log lỗi và không cập nhật dữ liệu.
        \item[7a3.] Use case kết thúc không thành công.
    \end{enumerate}
    \item[7b.] Sau khi kiểm tra dữ liệu (bước 6) và cố gắng lưu/đồng bộ (bước 7), hệ thống gặp lỗi server không phản hồi.
    \begin{enumerate}
        \item[7b1.] Hệ thống hiển thị thông báo: “Có lỗi xảy ra, vui lòng thử lại sau”.
        \item[7b2.] Hệ thống ghi log lỗi và không cập nhật dữ liệu.
        \item[7b3.] Use case kết thúc không thành công.
    \end{enumerate}
\end{enumerate} \\ \hline
\end{longtable}

% ================== USE CASE 15 ==================
\subsubsection{Thiết lập lịch rảnh}
\begin{longtable}{|p{3.5cm}|p{11cm}|}
\hline
\textbf{Use Case ID} & UC-15 \\ \hline
\textbf{Use Case Name} & Thiết lập lịch rảnh \\ \hline
\textbf{Actor(s)} & Tutor \\ \hline
\textbf{Description} & Tutor cập nhật lịch rảnh của mình lên hệ thống để phục vụ việc xếp lịch học. \\ \hline

\textbf{Trigger} & Tutor nhấn vào nút “Thêm lịch rảnh”. \\ \hline

\textbf{Pre–Condition(s)} &
Thiết bị đảm bảo kết nối internet và hệ thống duy trì hoạt động. \newline
Tài khoản đăng nhập đã được phân quyền tutor. \newline
Tutor đã đăng nhập vào hệ thống web. \\ \hline

\textbf{Post–Condition(s)} &
Lịch rảnh của Tutor được thêm mới vào hệ thống và hiển thị trong danh sách lịch rảnh. \\ \hline

\textbf{Normal Flow} &
1. Tutor bấm vào nút “Thêm lịch rảnh” ở trang chủ. \newline
2. Hệ thống hiển thị biểu mẫu thêm lịch rảnh. \newline
3. Tutor nhập thông tin ngày giờ và khoảng thời gian rảnh. \newline
4. Tutor bấm vào nút “Xác nhận”. \newline
5. Hệ thống hiển thị thông báo “Thêm lịch rảnh thành công”. \\ \hline

\textbf{Alternative Flow} &
3b. Tutor bấm nút “Hủy”: \newline
3b1. Hệ thống quay lại màn hình trang chủ. \newline
3b2. Không lưu lại bất kỳ thông tin nào đã nhập. 
3a. Tutor không điền đủ thông tin về thời gian. \newline
3a1. Hệ thống hiển thị thông báo lỗi: “Không điền đủ thông tin”. \newline
3a2. Hệ thống yêu cầu Tutor nhập đủ các trường bắt buộc trước khi tiếp tục. \newline
4a. Lịch rảnh mới thêm trùng với lịch rảnh cũ. \newline
4a1. Hệ thống hiển thị cảnh báo: “Lịch rảnh này đã tồn tại”. \newline
4a2. Hệ thống yêu cầu Tutor chỉnh sửa lại thông tin thời gian. \newline
\\ \hline
\textbf{Exception Flow} &

- Mất kết nối mạng trong quá trình thêm lịch → Hệ thống hiển thị thông báo lỗi “Không thể kết nối. Vui lòng thử lại sau”. \newline
- Lỗi hệ thống (server không phản hồi) → Hệ thống thông báo “Có lỗi xảy ra, vui lòng thử lại sau”. \\ \hline

\end{longtable}

% ================== USE CASE 16 ==================
\subsubsection{ Đăng tải tài liệu}
\begin{longtable}{|p{3.5cm}|p{11cm}|}
\hline
\textbf{Use Case ID} & UC-16 \\ \hline
\textbf{Use Case Name} &  Đăng tải tài liệu \\ \hline
\textbf{Actor(s)} & Tutor,  HCMUT\_LIBRARY  \\ \hline
\textbf{Description} & 
Tutor đăng tải tài liệu học tập (giáo trình, slide, …) lên hệ thống thư viện để chia sẻ cho sinh viên. 
Tài liệu phải chờ quản trị viên (HCMUT\_LIBRARY) duyệt trước khi hiển thị chính thức. \\ \hline

\textbf{Trigger} & Tutor bấm vào nút “Đăng tải tài liệu”. \\ \hline

\textbf{Pre–Condition(s)} & 
- Tutor đã đăng nhập thành công vào hệ thống. \newline
- Tài khoản Tutor có quyền thêm tài liệu. \newline
- Kết nối mạng ổn định và hệ thống đang hoạt động bình thường. \\ \hline

\textbf{Post–Condition(s)} & 
- Tài liệu được lưu vào hệ thống ở trạng thái “Chờ duyệt”. \newline
- Sau khi HCMUT\_LIBRARY duyệt thành công, tài liệu được hiển thị cho sinh viên. \\ \hline

\textbf{Normal Flow} & 
1. Tutor chọn chức năng “Đăng tải tài liệu” từ giao diện hệ thống. \newline
2. Hệ thống hiển thị giao diện đăng tải tài liệu. \newline
3. Tutor chọn file. \newline
4. Hệ thống tải tài liệu lên. \newline
5. Hệ thống kiểm tra định dạng, dung lượng file. \newline
6. Hệ thống lưu tài liệu ở trạng thái “Chờ duyệt”. \newline
7. Hệ thống hiển thị thông báo “Yêu cầu đăng tải đã được gửi. Tài liệu sẽ được hiển thị sau khi quản trị viên phê duyệt”. \newline
8. HCMUT\_LIBRARY kiểm duyệt tài liệu. \newline
9. Nếu được duyệt, tài liệu được cập nhật trạng thái “Đã duyệt” và hiển thị cho sinh viên. \\ \hline

\textbf{Alternative Flow} & 
\begin{enumerate}
    \item[3a.] Tutor chọn “Hủy”.
    \begin{enumerate}
        \item[3a1.] Hệ thống hủy thao tác đăng tải.
        \item[3a2.] Quay lại màn hình trước đó, không lưu bất kỳ thông tin nào.
    \end{enumerate}
    \item[5a.] File sai định dạng.
    \begin{enumerate}
        \item[5a1.] Hệ thống hiển thị thông báo “Định dạng không được hỗ trợ”.
        \item[5a2.] Quay lại bước 3.
    \end{enumerate}
    \item[5b.] File vượt dung lượng ($>$2GB).
    \begin{enumerate}
        \item[5b1.] Hệ thống hiển thị thông báo “File của bạn lớn hơn 2GB”.
        \item[5b2.] Quay lại bước 3.
    \end{enumerate}
\end{enumerate}
\\ \hline


\textbf{Exception Flow} & 
- Lỗi lưu trữ hoặc mất kết nối → Hệ thống hiển thị thông báo “Không thể kết nối. Vui lòng thử lại sau”. \newline
- HCMUT\_LIBRARY từ chối duyệt → Hệ thống gửi thông báo cho Tutor “Tài liệu bị từ chối. Vui lòng kiểm tra lại nội dung hoặc liên hệ quản trị viên.” \\ \hline

\end{longtable}


% ================== USE CASE 17 ==================
\subsubsection{Xóa tài liệu}
\begin{longtable}{|p{3.5cm}|p{11cm}|}
\hline
\textbf{Use Case ID} & UC-17 \\ \hline
\textbf{Use Case Name} & Xóa tài liệu \\ \hline
\textbf{Actor(s)} & Tutor, HCMUT\_LIBRARY \\ \hline
\textbf{Description} & 
Tutor xóa tài liệu đã đăng tải (trước hoặc sau khi duyệt) nhằm quản lý lại nội dung trong thư viện số. 
Yêu cầu xóa cần được quản trị viên (HCMUT\_LIBRARY) duyệt trước khi tài liệu bị gỡ bỏ khỏi hệ thống. \\ \hline

\textbf{Trigger} & Tutor chọn chức năng “Xóa tài liệu”. \\ \hline

\textbf{Pre–Condition(s)} & 
- Tutor đã đăng nhập thành công vào hệ thống. \newline
- Tài liệu cần xóa thuộc quyền sở hữu của Tutor. \newline
- Hệ thống hoạt động ổn định và có kết nối mạng. \\ \hline

\textbf{Post–Condition(s)} & 
- Tài liệu được chuyển sang trạng thái “Chờ duyệt xóa”. \newline
- Sau khi HCMUT\_LIBRARY duyệt, tài liệu bị gỡ bỏ khỏi hệ thống. \\ \hline

\textbf{Normal Flow} & 
1. Tutor chọn chức năng “Quản lý tài liệu”. \newline
2. Hệ thống hiển thị danh sách tài liệu mà Tutor đã đăng tải. \newline
3. Tutor chọn một hoặc nhiều tài liệu cần xóa. \newline
4. Tutor chọn nút “Xóa tài liệu”. \newline
5. Hệ thống hiển thị hộp thoại xác nhận. \newline
6. Tutor chọn “Đồng ý”. \newline
7. Hệ thống ghi nhận yêu cầu xóa và chuyển tài liệu sang trạng thái “Chờ duyệt xóa”. \newline
8. Hệ thống hiển thị thông báo: “Yêu cầu xóa đã được gửi. Tài liệu sẽ bị gỡ bỏ sau khi quản trị viên phê duyệt”. \newline
9. HCMUT\_LIBRARY kiểm duyệt yêu cầu xóa. \newline
10. Nếu được duyệt, hệ thống gỡ bỏ tài liệu khỏi thư viện số. \\ \hline

\textbf{Alternative Flow} & 
6a. Tutor chọn “Hủy” tại hộp thoại xác nhận. \newline
6a1. Hệ thống hủy thao tác xóa. \newline
6a2. Hệ thống quay lại màn hình danh sách tài liệu, không có thay đổi nào được thực hiện. \\ \hline

\textbf{Exception Flow} & 
- Lỗi khi ghi nhận yêu cầu xóa → Hệ thống hiển thị thông báo “Không thể gửi yêu cầu xóa. Vui lòng thử lại sau”. \newline
- Mất kết nối mạng trong quá trình thao tác → Hệ thống hiển thị thông báo lỗi kết nối. \newline
- Lỗi hệ thống (server không phản hồi) → Hệ thống thông báo “Có lỗi xảy ra, vui lòng thử lại sau”. \newline
- HCMUT\_LIBRARY từ chối yêu cầu xóa → Hệ thống gửi thông báo cho Tutor “Yêu cầu xóa bị từ chối. Vui lòng kiểm tra lại hoặc liên hệ quản trị viên.” \\ \hline

\end{longtable}


% ================== USE CASE 18 ==================
\subsubsection{Xử lý yêu cầu đặt lịch hẹn}
\begin{longtable}{|p{3.5cm}|p{11cm}|}
\hline
\textbf{Use Case ID} & UC-18 \\ \hline
\textbf{Use Case Name} & Xử lý yêu cầu đặt lịch hẹn \\ \hline
\textbf{Actor(s)} & Tutor \\ \hline
\textbf{Description} & 
Tutor xem danh sách các yêu cầu đặt lịch từ sinh viên, sau đó phê duyệt hoặc từ chối từng yêu cầu. Khi sinh viên gửi yêu cầu, hệ thống đã tạm giữ (giữ chỗ) slot rảnh tương ứng trong lịch của Tutor để tránh sinh viên khác đặt trùng. Nếu Tutor từ chối, Tutor có thể chọn có mở lại slot đó (trả về lịch rảnh) hay giữ slot vẫn bị khóa. \\ \hline

\textbf{Trigger} & 
Tutor bấm vào nút ``Quản lý yêu cầu đặt lịch hẹn''. \\ \hline

\textbf{Pre–Condition(s)} & 
- Tutor đã đăng nhập vào hệ thống. \newline
- Sinh viên đã gửi yêu cầu đặt lịch hẹn cho Tutor. \newline
- Đối với mỗi yêu cầu đang chờ xử lý, slot rảnh tương ứng trong lịch của Tutor được hệ thống tạm giữ. \newline
- Hệ thống hoạt động bình thường và có kết nối mạng. \\ \hline

\textbf{Post–Condition(s)} & 
- Yêu cầu lịch hẹn được Tutor xử lý (duyệt hoặc từ chối). \newline
- Hệ thống gửi thông báo kết quả cho sinh viên. \newline
- Nếu duyệt: lịch hẹn được ghi nhận chính thức vào hệ thống, slot tương ứng được đánh dấu là bận trong lịch của Tutor. \newline
- Nếu từ chối: yêu cầu bị hủy và không ghi vào hệ thống, slot tương ứng có thể được mở lại hoặc bị khóa tùy vào thiết lập của Tutor. \\ \hline

\textbf{Normal Flow} & 
\begin{enumerate}
    \item Tutor truy cập mục ``Quản lý yêu cầu lịch hẹn''.
    \item Hệ thống hiển thị danh sách các yêu cầu lịch hẹn đang chờ xử lý.
    \item Tutor chọn một yêu cầu cụ thể để xem chi tiết (thông tin sinh viên, thời gian, nội dung).
    \item Tutor chọn ``Phê duyệt'' hoặc ``Từ chối'' yêu cầu.
    \item Hệ thống ghi nhận quyết định của Tutor.
    \item Hệ thống cập nhật trạng thái slot của Tutor và gửi thông báo kết quả xử lý cho sinh viên.
\end{enumerate} \\ \hline

\textbf{Alternative Flow} & 
\begin{enumerate}
    \item[2a.] Không có yêu cầu nào đang chờ xử lý.
        \item[2a1.] Hệ thống hiển thị thông báo: ``Không có yêu cầu lịch hẹn nào''.
        \item[2a2.] Tutor chọn ``Quay lại'' để trở về trang quản lý.
        \item[2a3.] Use case kết thúc sớm.

    \item[4a.] Tutor chọn ``Từ chối'' yêu cầu.
    \begin{enumerate}
        \item[4a1.] Hệ thống hiển thị hộp thoại yêu cầu Tutor nhập lý do từ chối (tùy chọn) và chọn cách xử lý slot.
        \item[4a2.] Tutor nhập lý do (hoặc bỏ trống) và chọn một trong hai tùy chọn xử lý slot:
            \item[4a2.1.] ``Mở lại slot'': Slot được đánh dấu lại là rảnh trong lịch của Tutor, sinh viên khác có thể đặt.
            \item[4a2.2.] ``Không mở lại slot'': Slot vẫn được giữ ở trạng thái bận, không hiển thị như một slot rảnh cho sinh viên khác.
        \item[4a3.] Tutor xác nhận thao tác từ chối.
        \item[4a4.] Hệ thống ghi nhận quyết định từ chối, xử lý trạng thái slot theo lựa chọn ở bước 4a2 và gửi thông báo cho sinh viên.
    \end{enumerate}

    \item[4b.] Tutor đổi ý và chọn ``Hủy thao tác'' trước khi xác nhận.
        \item[4b1.] Hệ thống không lưu bất kỳ thay đổi nào.
        \item[4b2.] Giao diện quay lại danh sách yêu cầu (hoặc màn hình chi tiết yêu cầu tùy thiết kế).
        \item[4b3.] Use case kết thúc sớm.
\end{enumerate} \\ \hline

\textbf{Exception Flow} & 
\begin{enumerate}
    \item[5a.] Khi ghi nhận quyết định, hệ thống gặp lỗi cơ sở dữ liệu hoặc mất kết nối.
        \item[5a1.] Hệ thống hiển thị thông báo: ``Xử lý thất bại, vui lòng thử lại sau hoặc liên hệ bộ phận kỹ thuật''.
        \item[5a2.] Hệ thống ghi log lỗi, không thay đổi trạng thái yêu cầu và trạng thái slot.
        \item[5a3.] Use case kết thúc không thành công.
\end{enumerate} \\ \hline

\end{longtable}








\section{Task 2}
\subsection{Sequence Diagram và Activity Diagram}


\subsubsection{Đăng nhập}
\begin{figure}[H]
    \centering
    \includegraphics[width=1\linewidth]{Image/ĐN SQ.drawio.png}
    \caption{Sequence Diagram cho UC Đăng nhập}
\end{figure}
\begin{figure}[H]
    \centering
    \includegraphics[width=1\linewidth]{Image/ĐN AC.drawio.png}
    \caption{Activity Diagram cho UC Đăng nhập}
\end{figure}
\textbf{Mô tả:}

\textbf{Luồng sự kiện chính: Đăng nhập thành công}
\begin{enumerate}
    \item \textbf{Người dùng} truy cập vào trang chủ của hệ thống \textbf{Tutor Support System}.
    \item \textbf{System} hiển thị giao diện đăng nhập với nút “\textbf{Đăng nhập qua HCMUT\_SSO}”.
    \item \textbf{Người dùng} nhấn vào nút “Đăng nhập qua HCMUT\_SSO”.
    \item \textbf{System} chuyển hướng người dùng đến trang đăng nhập của \textbf{HCMUT\_SSO}.
    \item \textbf{Người dùng} nhập thông tin đăng nhập (\texttt{BKNetID} và mật khẩu).
    \item \textbf{HCMUT\_SSO} xác thực thông tin đăng nhập.
    \item Nếu thông tin hợp lệ, \textbf{HCMUT\_SSO} trả về \textbf{auth token} cho \textbf{Tutor Support System}.
    \item \textbf{Tutor Support System} sử dụng token để truy vấn \textbf{HCMUT\_DATACORE} và lấy thông tin người dùng (họ tên, mã số, email, vai trò, trạng thái).
    \item \textbf{Tutor Support System} kiểm tra quyền truy cập:
    \begin{itemize}
        \item Nếu người dùng có vai trò hợp lệ, hệ thống phân quyền và khởi tạo phiên đăng nhập.
    \end{itemize}
    \item \textbf{System} chuyển hướng người dùng đến trang chủ tương ứng với vai trò (Sinh viên / Tutor / Điều phối viên).
\end{enumerate}

\textbf{Các luồng phụ và luồng lỗi (Alternative \& Error Flows)}

\begin{enumerate}

    \item \textbf{Thông tin đăng nhập không chính xác:}\\
    Nếu người dùng nhập sai BKNetID hoặc mật khẩu:
    \begin{itemize}
        \item \textbf{HCMUT\_SSO} hiển thị thông báo lỗi trực tiếp trên form đăng nhập.
    \end{itemize}
    Use Case quay lại bước nhập thông tin đăng nhập (bước 5).

    \item \textbf{Tài khoản bị khóa :}\\
    Nếu tài khoản không thể xác thực do trạng thái tài khoản không hợp lệ:
    \begin{itemize}
        \item \textbf{HCMUT\_SSO} hiển thị thông báo lỗi tương ứng trên form đăng nhập.
    \end{itemize}
    Use Case kết thúc không thành công tại SSO.

    \item \textbf{Không có quyền truy cập hệ thống:}\\
    Nếu người dùng đăng nhập thành công nhưng vai trò không thuộc nhóm được phép sử dụng hệ thống:
    \begin{itemize}
        \item \textbf{Tutor Support System} hiển thị thông báo:
        \textit{“Bạn không có quyền truy cập hệ thống.”}
    \end{itemize}
    Phiên đăng nhập không được tạo; Use Case kết thúc.

\end{enumerate}

\subsubsection{Quên mật khẩu}
\begin{figure}[H]
    \centering
    \includegraphics[width=1\linewidth]{Image/AC QMK.drawio.png}
    \caption{Activity Diagram cho UC Quên mật khẩu }
\end{figure}

\textbf{Mô tả:}
Use Case mô tả quy trình người dùng yêu cầu đặt lại mật khẩu tài khoản thông qua hệ thống HCMUT\_SSO. Người dùng nhập BKID, nhận email khôi phục, truy cập liên kết đặt lại mật khẩu và tạo mật khẩu mới.


\textbf{Luồng sự kiện chính: Khôi phục mật khẩu thành công}
\begin{enumerate}
    \item \textbf{Người dùng} truy cập trang đăng nhập và chọn nút ``\textbf{Quên mật khẩu}''.
    \item \textbf{SSO} hiển thị form yêu cầu nhập BKID.
    \item \textbf{Người dùng} nhập BKID.
    \item \textbf{SSO} kiểm tra email tương ứng với BKID.
    \item Nếu email hợp lệ, \textbf{SSO} gửi email chứa liên kết khôi phục mật khẩu.
    \item \textbf{Người dùng} mở email và nhấn vào liên kết đặt lại mật khẩu.
    \item \textbf{SSO} hiển thị form đặt lại mật khẩu.
    \item \textbf{Người dùng} nhập mật khẩu mới và xác nhận mật khẩu.
    \item \textbf{SSO} kiểm tra tính hợp lệ của mật khẩu mới.
    \item \textbf{SSO} cập nhật mật khẩu vào hệ thống.
    \item \textbf{SSO} hiển thị thông báo ``\textbf{Thay đổi mật khẩu thành công}''.
\end{enumerate}


\textbf{Các luồng phụ và luồng lỗi (Alternative \& Error Flows)}

\begin{enumerate}

    \item \textbf{Email không hợp lệ hoặc không tồn tại:}\\
    Nếu BKID không có email hợp lệ:
    \begin{itemize}
        \item \textbf{SSO} hiển thị thông báo ``Email không hợp lệ'' trên form.
    \end{itemize}
    Use Case kết thúc không thành công.


    \item \textbf{Người dùng không nhận được email:}\\
    Nếu người dùng chờ nhưng không nhận email:
    \begin{itemize}
        \item Người dùng chọn ``\textbf{Gửi lại email}''.
        \item \textbf{SSO} gửi lại email khôi phục.
        \item \textbf{SSO} hiển thị thông báo ``Email đã được gửi lại''.
    \end{itemize}


    \item \textbf{Liên kết khôi phục hết hạn hoặc không hợp lệ:}\\
    \begin{itemize}
        \item \textbf{SSO} hiển thị thông báo ``Liên kết không hợp lệ hoặc đã hết hạn''.
    \end{itemize}
    Use Case kết thúc; người dùng phải thực hiện lại từ đầu.


    \item \textbf{Mật khẩu mới không hợp lệ:}\\
    Khi người dùng nhập mật khẩu mới không đạt yêu cầu:
    \begin{itemize}
        \item \textbf{SSO} hiển thị lỗi tương ứng (quá ngắn, không trùng khớp…).
    \end{itemize}
    Use Case quay lại bước nhập mật khẩu mới.
    \item \textbf{Lỗi hệ thống khi cập nhật mật khẩu:}\\
    \begin{itemize}
        \item \textbf{SSO} hiển thị thông báo lỗi: ``Không thể thay đổi mật khẩu. Vui lòng thử lại sau.'' 
    \end{itemize}
    Use Case kết thúc không thành công.

\end{enumerate}


\subsubsection{Đặt lịch hẹn}
\begin{figure}[H]
    \centering
    \includegraphics[width=1\linewidth]{Image/Datlichhen_Sequence.png}
    \caption{Sequence Diagram cho UC Đặt lịch hẹn}
\end{figure}
\begin{figure}[H]
    \centering
    \includegraphics[width=1\linewidth]{Image/Datlichhen_Activity.png}
    \caption{Activity Diagram cho UC Đặt lịch hẹn}
\end{figure}
\textbf{Mô tả: }
\textbf{Luồng sự kiện chính: Đặt lịch thành công}
\begin{enumerate}
    \item Sinh viên truy cập vào trang \textbf{đặt lịch} của một giảng viên cụ thể.
    \item \textbf{UI} gửi yêu cầu tìm lịch rảnh của Tutor tới \textbf{System} .
    \item \textbf{System} thực hiện truy vấn \textbf{Database} để lấy đồng thời:
    \begin{itemize}
        \item Thông tin chi tiết của giảng viên (tên, email);
        \item Danh sách các khung giờ rảnh (\texttt{time\_available}).
    \end{itemize}
    \item \textbf{System} tổng hợp dữ liệu và trả về cho \textbf{UI} để hiển thị đầy đủ cho Sinh viên.
    \item Sinh viên xem lịch và chọn một khung giờ phù hợp, sau đó nhấn nút \textbf{Xác nhận}.
    \item \textbf{UI} gửi yêu cầu đặt lịch kèm theo thông tin cần thiết đến \textbf{System}.
    \item \textbf{System} tạo một bản ghi yêu cầu mới và lưu vào \textbf{Database} với trạng thái ban đầu là \texttt{pending} (đang chờ duyệt).
    \item \textbf{System} trả về thông báo thành công cho \textbf{UI}.
    \item \textbf{UI} hiển thị thông báo:
    \textit{“Yêu cầu đã được gửi, vui lòng chờ Tutor duyệt.”}
\end{enumerate}

\textbf{Các luồng phụ và luồng lỗi (Alternative \& Error Flows)}

\begin{enumerate}
    \item \textbf{Thông tin không hợp lệ:} \\
    Nếu thông tin mà Sinh viên gửi lên không hợp lệ (ví dụ: không chọn khung giờ hoặc nội dung yêu cầu trống), 
    \textbf{System} sẽ từ chối xử lý và trả về thông báo lỗi cho \textbf{UI}. 
    \textbf{UI} hiển thị:
    \textit{“Thời gian bạn chọn không hợp lệ, vui lòng chọn lại”}. Lúc này, giao diện màn hình quay về trang chọn thời gian và nhập nội dung hỗ trợ.

    \item \textbf{Không có lịch trống:} \\
    Khi \textbf{System} truy vấn \textbf{Database} ở bước xem lịch, nếu kết quả trả về là danh sách rỗng,
    \textbf{System} sẽ thông báo cho \textbf{UI}. 
    \textbf{UI} hiển thị:
    \textit{“Tutor đã hết lịch trống.”}

    \item \textbf{Lỗi khi lưu dữ liệu:}\\
    Trong quá trình lưu thông tin đặt lịch, nếu xảy ra lỗi (ví dụ: mất kết nối, lỗi truy vấn), 
    \textbf{Database} ném ngoại lệ. \textbf{System} trả về thông báo lỗi chung cho người dùng.
    \textit{“Hệ thống đang gặp sự cố, vui lòng thử lại sau.”}
\end{enumerate}



\subsubsection{Ghi nhận tiến độ}
\begin{figure}[H]
    \centering
    \includegraphics[width=1\linewidth]{Image/GNTĐ SQ.drawio.png}
    \caption{Sequence Diagram cho UC Ghi nhận tiến độ}
\end{figure}
\begin{figure}[H]
    \centering
    \includegraphics[width=1\linewidth]{Image/AC GNTĐ.drawio.png}
    \caption{Activity Diagram cho UC Ghi nhận tiến độ }
\end{figure}

\textbf{Mô tả:}\\
Use case mô tả quy trình Tutor ghi nhận tiến độ học tập hoặc kết quả buổi học của sinh viên trong hệ thống Tutor Support System. Tiến độ bao gồm nội dung mô tả, đánh giá, mức độ hoàn thành và các ghi chú liên quan.


\textbf{Luồng sự kiện chính: Ghi nhận tiến độ thành công}
\begin{enumerate}
    \item \textbf{Tutor} truy cập vào hệ thống \textbf{Tutor Support System}.
    \item \textbf{System} hiển thị giao diện chính với mục chức năng ``\textbf{Ghi nhận tiến độ}''.
    \item \textbf{Tutor} chọn chức năng ``Ghi nhận tiến độ''.
    \item \textbf{System} hiển thị danh sách sinh viên hoặc buổi học mà Tutor đang phụ trách.
    \item \textbf{Tutor} chọn một sinh viên hoặc buổi học cần ghi nhận tiến độ.
    \item \textbf{System} hiển thị form ghi nhận tiến độ, bao gồm: mô tả, đánh giá, mức độ hoàn thành, kết quả học tập.
    \item \textbf{Tutor} nhập đầy đủ các thông tin tiến độ.
    \item \textbf{Tutor} nhấn nút ``\textbf{Gửi báo cáo}''.
    \item \textbf{System} kiểm tra dữ liệu nhập (các trường bắt buộc, định dạng).
    \item Nếu dữ liệu hợp lệ, \textbf{System} lưu tiến độ vào cơ sở dữ liệu.
    \item \textbf{System} gửi thông báo xác nhận ghi nhận thành công.
    \item \textbf{System} hiển thị thông báo ``\textbf{Ghi nhận tiến độ thành công}''. 
\end{enumerate}


\textbf{Các luồng phụ và luồng lỗi (Alternative \& Error Flows)}

\begin{enumerate}

    \item \textbf{Chọn nhầm sinh viên hoặc buổi học:}\\
    Nếu Tutor chọn sai đối tượng:
    \begin{itemize}
        \item \textbf{System} hiển thị thông báo yêu cầu chọn lại.
    \end{itemize}
    Use Case quay lại bước chọn sinh viên/buổi học (bước 5).

    \vspace{0.2cm}
    \item \textbf{Tutor hủy thao tác trước khi gửi báo cáo:}\\
    Nếu Tutor quay lại hoặc thoát form ghi nhận:
    \begin{itemize}
        \item \textbf{System} hủy các dữ liệu đang nhập.
    \end{itemize}
    Use Case quay về danh sách sinh viên/buổi học.

    \vspace{0.2cm}
    \item \textbf{Dữ liệu nhập không hợp lệ:}\\
    Khi Tutor nhập thiếu thông tin hoặc sai định dạng:
    \begin{itemize}
        \item \textbf{System} hiển thị lỗi tương ứng trực tiếp trên form.
    \end{itemize}
    Use Case quay lại bước 7.

    \vspace{0.2cm}
    \item \textbf{Lỗi kết nối cơ sở dữ liệu:}\\
    Nếu hệ thống không thể lưu tiến độ:
    \begin{itemize}
        \item \textbf{System} hiển thị thông báo: ``Không thể lưu tiến độ. Vui lòng thử lại sau.''
    \end{itemize}
    Use Case kết thúc không thành công.

    \vspace{0.2cm}
    \item \textbf{Lỗi khi gửi thông báo xác nhận:}\\
    Nếu tiến độ đã được lưu nhưng hệ thống không gửi được thông báo:
    \begin{itemize}
        \item \textbf{System} hiển thị cảnh báo: ``Ghi nhận thành công nhưng không gửi được thông báo.''
    \end{itemize}
    Use Case kết thúc.

\end{enumerate}










\subsubsection{Đăng tải tài liệu}
\begin{figure}[H]
    \centering
    \includegraphics[width=0.9\linewidth]{Image/seq_DangTaiLieu.drawio.png}
    \caption{Sequence diagram:  Đăng tải tài liệu}
    \label{fig:placeholder}
\end{figure}


\begin{figure}[H]
    \centering
    \includegraphics[width=0.9\linewidth]{Image/act_DangTaiLieu.drawio.png}
    \caption{Activity diagram: Đăng tải tài liệu}
    \label{fig:placeholder}
\end{figure}

\noindent \textbf{Mô tả}

\noindent \textbf{Luồng sự kiện chính: Đăng tải tài liệu thành công}

\begin{enumerate}
  \item Tutor bấm vào mục ``Đăng tải tài liệu'' trong trang ``Khóa học''.
  \item Hệ thống hiển thị form đăng tải tài liệu.
  \item Tutors nhấn ``Chọn file'' và chọn tệp cần đăng hoặc kéo / thả file cần đăng.
  \item Hệ thống kiểm tra ràng buộc:
    \begin{enumerate}
      \item Kiểm tra định dạng tệp có được hỗ trợ.
      \item Kiểm tra dung lượng tệp không vượt quá 2GB.
    \end{enumerate}
  \item Nếu hợp lệ, Hệ thống tải tệp lên kho lưu trữ.
  \item Hệ thống thông báo ``Tài liệu đã được đăng tải. Chờ duyệt.''
  \item Hệ thống gửi yêu cầu duyệt sang \textit{HCMUT\_LIBRARY}.
  \item \textit{HCMUT\_LIBRARY} duyệt tài liệu và chấp nhận.
  \item Hệ thống hiển thị thông báo ``Đã duyệt'' và tài liệu xuất hiện trong danh sách của khóa học.
  \item Use Case kết thúc thành công.
\end{enumerate}

\noindent \textbf{Các luồng phụ và luồng lỗi (Alternative \& Error Flows)}

\noindent\textbf{Hủy đăng tải}
\begin{enumerate}
  \item Ở bước 2, Tutor chọn ``Hủy''.
  \item Hệ thống quay về giao diện ``Khóa học''. Use Case kết thúc.
\end{enumerate}

\noindent \textbf{Định dạng không được hỗ trợ}
\begin{enumerate}
  \item Ở bước 4(a), nếu tệp không đúng định dạng cho phép, hệ thống thông báo ``Định dạng không hợp lệ''.
  \item Quay về bước 3 để tutor chọn file lại.
\end{enumerate}

\noindent \textbf{Dung lượng vượt quá giới hạn}
\begin{enumerate}
  \item Ở bước 4(b), nếu dung lượng $> \text{2GB}$, \textit{Hệ thống} thông báo ``Dung lượng file vượt quá 2GB''.
  \item Quay về bước 3 để tutor chọn file lại.
\end{enumerate}

\noindent \textbf{Lỗi lưu trữ / mất kết nối}
\begin{enumerate}
  \item Trong bước 5, nếu lỗi phát sinh, Hệ thống thông báo ``Không thể kết nối. Vui lòng thử lại sau.''
  \item Use Case tạm dừng; Tutor có thể thực hiện lại từ bước 1 khi kết nối ổn định.
\end{enumerate}

\noindent \textbf{Từ chối phê duyệt}
\begin{enumerate}
  \item Ở bước 8, nếu bị từ chối, Hệ thống hiển thị ``Tài liệu bị từ chối'' (kèm lý do nếu có).
  \item Tutor có thể chỉnh sửa/đổi tệp và đăng lại từ bước 1. Use Case kết thúc không thành công.
\end{enumerate}

\subsubsection{Phản hồi chất lượng buổi học}
\begin{figure}[H]
    \centering
    \includegraphics[width=1\linewidth]{Image/phanhoi-Trang-1.drawio.png}
    \caption{Sequence Diagram cho Phản hồi chất lượng buổi học}
\end{figure}
\begin{figure}[H]
    \centering
    \includegraphics[width=1\linewidth]{Image/phanhoi-Trang-2.drawio.png}
    \caption{Activity Diagram cho Phản hồi chất lượng buổi học}
\end{figure}
\textbf{Mô tả:}


\textbf{Luồng sự kiện chính: Gửi phản hồi thành công}

\begin{enumerate}
    \item Sinh viên truy cập vào hệ thống và chọn mục ``Phản hồi chất lượng''
    \item Hệ thống hiển thị danh sách các buổi học đã tham gia.
    \item Sinh viên chọn buổi học cần phản hồi.
    \item Hệ thống (UI) hiển thị form phản hồi bao gồm: mức độ hài lòng và ô nhập nội dung.
    \item Sinh viên nhập nội dung phản hồi, chọn mức độ đánh giá, sau đó nhấn ``Gửi''.
    \item Hệ thống nhận dữ liệu, thực hiện kiểm tra tính hợp lệ (đủ nội dung, không chứa ký tự lỗi, định dạng hợp lệ).
    \item Nếu dữ liệu hợp lệ, Hệ thống lưu phản hồi vào \textbf{CSDL Phản hồi}.
    \item Khi lưu thành công, Hệ thống gửi phản hồi xác nhận về cho UI.
    \item UI hiển thị thông báo: ``Gửi phản hồi thành công.''
    \item Use case kết thúc.
\end{enumerate}
\vspace{1em}
\textbf{Luồng phụ và luồng lỗi }

\begin{enumerate}
    \item \textbf{Sinh viên nhấn ``Quay lại''}
    \begin{itemize}
        \item Khi form phản hồi đang hiển thị, nếu sinh viên chọn ``Quay lại'', Hệ thống đóng form và không lưu bất kỳ thông tin nào.
        \item Use case kết thúc sớm.
    \end{itemize}
    
    \item \textbf{Dữ liệu không hợp lệ}
    \begin{itemize}
        \item Nếu sinh viên không nhập đủ nội dung phản hồi, Hệ thống hiển thị thông báo lỗi:
        \begin{quote}
            ``Nội dung phản hồi không hợp lệ, vui lòng nhập lại.''
        \end{quote}
        \item Sau đó, giao diện quay lại form phản hồi ban đầu (reset form rỗng).
        \item Sinh viên có thể chọn nhập lại nội dung và gửi lại hoặc nhấn hủy.
    \end{itemize}
    
    \item \textbf{Lỗi khi lưu dữ liệu}
    \begin{itemize}
        \item Trong quá trình lưu phản hồi, nếu CSDL xảy ra lỗi (ví dụ: mất kết nối, lỗi truy vấn), Hệ thống bắt ngoại lệ và gọi bộ ghi log (\texttt{Log lỗi}) để ghi chi tiết sự cố.
        \item Sau đó, Hệ thống gửi phản hồi lỗi về cho UI.
        \item UI hiển thị thông báo:
        \begin{quote}
            ``Gửi phản hồi thất bại, vui lòng thử lại sau hoặc liên hệ bộ phận kỹ thuật.''
        \end{quote}
        \item Use case kết thúc không thành công.
    \end{itemize}
\end{enumerate}

\subsubsection{Hủy buổi gặp mặt}
\begin{figure}[H]
    \centering
    \includegraphics[width=1\linewidth]{Image/phanhoi-Trang-4.drawio.png}
    \caption{Sequence Diagram cho Hủy buổi gặp mặt}
\end{figure}
\begin{figure}[H]
    \centering
    \includegraphics[width=1\linewidth]{Image/phanhoi-Trang-3.drawio.png}
    \caption{Activity Diagram cho Hủy buổi gặp mặt}
\end{figure}

\textbf{Mô tả:}


\textbf{Luồng sự kiện chính: Hủy buổi gặp thành công}

\begin{enumerate}
\item Tutor truy cập vào hệ thống và chọn mục ``Hủy lịch''.
\item Hệ thống (UI) hiển thị danh sách buổi gặp đã được lên lịch.
\item Tutor chọn một buổi gặp cụ thể cần hủy.
\item Hệ thống hiển thị form yêu cầu nhập lý do hủy.
\item Tutor nhập lý do và nhấn ``Xác nhận''.
\item Hệ thống kiểm tra tính hợp lệ của dữ liệu (có nhập lý do, không chứa ký tự lỗi).
\item Nếu dữ liệu hợp lệ, hệ thống cập nhật trạng thái buổi gặp trong cơ sở dữ liệu (hủy buổi gặp) và cập nhật lại lịch rảnh của Tutor.
\item Khi lưu thay đổi thành công, hệ thống gửi phản hồi xác nhận cho UI.
\item UI hiển thị thông báo: ``Hủy buổi gặp thành công.'' 
\item Use case kết thúc.
\end{enumerate}

\textbf{Luồng phụ và luồng lỗi}

\begin{enumerate}
    \item \textbf{Sinh viên nhấn ``Hủy''}
    \begin{itemize}
        \item Khi form phản hồi đang hiển thị, nếu sinh viên chọn ``Hủy'', Hệ thống đóng form và không lưu bất kỳ thông tin nào.
        \item Use case kết thúc sớm.
    \end{itemize}

    \item \textbf{Dữ liệu không hợp lệ}
    \begin{itemize}
        \item Nếu sinh viên không nhập nội dung phản hồi hoặc nội dung không đạt yêu cầu, Hệ thống hiển thị thông báo lỗi:
        \begin{quote}
            ``Nội dung phản hồi không hợp lệ, vui lòng nhập lại.''
        \end{quote}
        \item Sau đó, giao diện quay lại form phản hồi ban đầu (reset form rỗng).
        \item Sinh viên có thể chọn nhập lại nội dung và gửi lại hoặc nhấn hủy.
    \end{itemize}

    \item \textbf{Lỗi khi lưu dữ liệu}
    \begin{itemize}
        \item Trong quá trình lưu phản hồi, nếu CSDL xảy ra lỗi (ví dụ: mất kết nối, lỗi truy vấn), Hệ thống bắt ngoại lệ và gọi bộ ghi log (\texttt{Log lỗi}) để ghi chi tiết sự cố.
        \item Sau đó, Hệ thống gửi phản hồi lỗi về cho UI.
        \item UI hiển thị thông báo:
        \begin{quote}
            ``Gửi phản hồi thất bại, vui lòng thử lại sau hoặc liên hệ bộ phận kỹ thuật.''
        \end{quote}
        \item Use case kết thúc không thành công.
    \end{itemize}
\end{enumerate}


\subsubsection{Tạo buổi tư vấn}
\begin{figure}[H]
    \centering
    \includegraphics[width=1.1\textwidth]{Image/new_ac.drawio.png}
    \caption{Sequence Diagram cho Use Case “Tạo buổi tư vấn”}
\end{figure}

\begin{figure}[H]
    \centering
    \includegraphics[width=1.05\textwidth]{Image/ac.drawio.png}
    \caption{Activity Diagram cho Use Case “Tạo buổi tư vấn”}
\end{figure}

% ========== Mô tả ==========
\textbf{Mô tả}

\textbf{Luồng sự kiện chính: Tạo buổi tư vấn thành công}
\begin{enumerate}

  \item \textbf{Tutor} truy cập vào chức năng ``Tạo buổi tư vấn'' trên giao diện hệ thống.
  \item \textbf{UI} hiển thị Form \texttt{Tạo buổi tư vấn} cho Tutor nhập thông tin chi tiết.
  \item \textbf{Tutor} nhập các thông tin cần thiết, bao gồm: chủ đề, thời gian bắt đầu và kết thúc, hình thức (trực tiếp / trực tuyến), mô tả nội dung và địa điểm (nếu có).
  \item Khi hoàn tất, \textbf{Tutor} nhấn nút \texttt{Xác nhận}. \textbf{UI} gửi yêu cầu tạo buổi kèm dữ liệu tới \textbf{System}.
  \item \textbf{System} thực hiện các kiểm tra hợp lệ trên dữ liệu nhận được: kiểm tra trường bắt buộc, định dạng thời gian, và độ dài nội dung.
  \item Nếu hợp lệ, \textbf{System} truy vấn \textbf{Database} để kiểm tra: thời gian có nằm trong lịch rảnh của Tutor và có trùng với buổi tư vấn nào khác hay không.
  \item Nếu không có xung đột, \textbf{System} lưu thông tin buổi tư vấn mới vào \textbf{Database}.
  \item \textbf{System} trả về thông báo thành công cho \textbf{UI}.
  \item \textbf{UI} hiển thị thông báo: \textit{“Tạo buổi tư vấn thành công”} và cập nhật danh sách buổi tư vấn trên giao diện.

\end{enumerate}



\textbf{Các luồng phụ và luồng lỗi (Alternative \& Error Flows)}



\begin{enumerate}

\item \textbf{Thông tin không hợp lệ:}\\
Nếu dữ liệu \textbf{Tutor} nhập bị thiếu hoặc sai định dạng (ví dụ: để trống thời gian, nhập sai định dạng giờ), 
\textbf{System} sẽ từ chối xử lý và trả về thông báo lỗi cho \textbf{UI}.\\
\textbf{UI} hiển thị: \textit{“Thông tin không hợp lệ, vui lòng kiểm tra lại.”}\\
Form giữ nguyên để Tutor chỉnh sửa và gửi lại.

\item \textbf{Thời gian không nằm trong lịch rảnh:}\\
Khi kiểm tra \textbf{Database}, nếu thời gian nhập không thuộc các khung giờ rảnh đã đăng ký, 
\textbf{System} trả về lỗi cho \textbf{UI}.\\
\textbf{UI} hiển thị: \textit{“Thời gian bạn chọn không nằm trong lịch rảnh.”}\\
Tutor chọn lại thời gian hợp lệ.

\item \textbf{Xung đột thời gian (đã có buổi khác):}\\
Nếu \textbf{System} phát hiện có buổi tư vấn khác trùng hoặc chồng lấn khung giờ, 
hệ thống phản hồi lỗi tới \textbf{UI}.\\
\textbf{UI} hiển thị: \textit{“Thời gian đã được sử dụng. Vui lòng chọn khung giờ khác.”}\\
Tutor có thể chỉnh sửa và gửi lại.

\item \textbf{Hủy thao tác:}\\
Nếu \textbf{Tutor} nhấn nút \texttt{Hủy} trong khi nhập form, 
\textbf{UI} đóng form và không lưu bất kỳ thông tin nào.\\
Use Case kết thúc tại đây.

\item \textbf{Lỗi hệ thống / lỗi khi lưu dữ liệu:}\\
Trong quá trình lưu dữ liệu, nếu xảy ra lỗi kết nối hoặc lỗi truy vấn, 
\textbf{System} trả về thông báo lỗi chung.\\
\textbf{UI} hiển thị: \textit{“Hệ thống đang gặp sự cố, vui lòng thử lại sau.”}\\
Use Case kết thúc không thành công.

\end{enumerate}

\subsubsection{Điều phối khung chương trình chung}
\begin{figure}[H]
    \centering
    \includegraphics[width=1\linewidth]{Image/seq_DPV.drawio.png}
    \caption{Sequence Diagram cho UC Điều phối khung chương trình}
\end{figure}
\begin{figure}[H]
    \centering
    \includegraphics[width=1\linewidth]{Image/act_DPV.drawio.png}
    \caption{Activity Diagram cho UC Điều phối khung chương trình}
\end{figure}


\textbf{Mô tả:}
\textbf{Luồng chính: Cập nhật hoặc tạo khung chương trình thành công}
\begin{enumerate}
\item \textbf{Điều phối viên} truy cập vào hệ thống và chọn chức năng “\texttt{Quản lý khung chương trình}”.
\item \textbf{Hệ thống (UI)} gửi yêu cầu truy vấn danh sách khung chương trình hiện có đến \textbf{Bộ điều khiển}.
\item \textbf{Bộ điều khiển} thực hiện lệnh \texttt{SELECT} tới \textbf{CSDL} và nhận kết quả trả về.
\item \textbf{Hệ thống} hiển thị danh sách các khung chương trình hiện có trên giao diện cho điều phối viên.
\item \textbf{Điều phối viên} chọn thao tác:
\begin{itemize}
\item \texttt{Tạo khung chương trình mới}, hoặc
\item \texttt{Cập nhật khung chương trình cũ}.
\end{itemize}
\item Hệ thống hiển thị form nhập thông tin khung chương trình tương ứng (mới hoặc đã điền sẵn).
\item \textbf{Điều phối viên} nhập đầy đủ thông tin cần thiết (chủ đề, nội dung, thời lượng, mô tả...) và nhấn nút \texttt{Lưu}.
\item \textbf{Hệ thống} nhận dữ liệu, gửi đến \textbf{Bộ điều khiển} để kiểm tra ràng buộc:
\begin{itemize}
\item Đủ các trường thông tin bắt buộc.
\item Chủ đề chưa bị trùng.
\end{itemize}
\item Nếu hợp lệ, \textbf{Bộ điều khiển} thực hiện \texttt{INSERT} hoặc \texttt{UPDATE} vào \textbf{CSDL}.
\item \textbf{CSDL} phản hồi thành công (\texttt{OK}).
\item \textbf{Hệ thống} hiển thị thông báo “Cập nhật khung chương trình thành công” và gửi thông báo đến các \textbf{Tutor}.
\item Use case kết thúc thành công.
\end{enumerate}


\textbf{Luồng phụ và luồng lỗi:}

\begin{enumerate}
\item \textbf{Điều phối viên chọn “Hủy” thao tác}
\begin{itemize}
\item Khi đang ở form tạo/cập nhật, nếu điều phối viên chọn “Hủy”, hệ thống quay lại giao diện chính và không lưu bất kỳ thay đổi nào.
\item Use case kết thúc sớm.
\end{itemize}

\item \textbf{Thiếu trường thông tin}
\begin{itemize}
\item Khi điều phối viên nhấn “Lưu” nhưng chưa nhập đủ các trường bắt buộc (chủ đề, nội dung...), \textbf{Bộ điều khiển} phát hiện lỗi ràng buộc.
\item Hệ thống hiển thị thông báo lỗi:
\begin{quote}
“Thiếu thông tin, vui lòng nhập đầy đủ các trường bắt buộc.”
\end{quote}
\item Giao diện quay lại form nhập dữ liệu để người dùng bổ sung.
\end{itemize}

\item \textbf{Chủ đề đã tồn tại}
\begin{itemize}
\item Khi thực hiện kiểm tra ràng buộc, nếu phát hiện chủ đề nhập vào đã tồn tại trong CSDL, hệ thống hiển thị thông báo:

“Chủ đề này đã tồn tại, vui lòng chọn tên khác.”

\item Use case kết thúc mà không lưu dữ liệu.
\end{itemize}

\item \textbf{Lỗi CSDL (Database Error)}
\begin{itemize}
\item Trong quá trình lưu dữ liệu, nếu xảy ra lỗi như mất kết nối hoặc lỗi truy vấn, \textbf{Bộ điều khiển} nhận phản hồi \texttt{ERROR}.
\item Hệ thống ghi log lỗi vào bộ ghi log nội bộ.
\item UI hiển thị thông báo:

“Lưu khung chương trình thất bại. Vui lòng thử lại sau hoặc liên hệ bộ phận kỹ thuật.”

\item Use case kết thúc không thành công.
\end{itemize}
\end{enumerate}

\subsection{Mockup}

\subsubsection{Trang đăng nhập}
\begin{figure}[H]
    \centering
    \includegraphics[width=0.7\linewidth]{Image/login.png}
    \caption{Trang login của hệ thống}
    \label{fig:placeholder}
\end{figure}

\begin{figure}[H]
    \centering
    \includegraphics[width=0.7\linewidth]{Image/loginsso.png}
    \caption{Sau khi ấn đăng nhập hệ thống chuyển sang HCMUT\_SSO}
    \label{fig:placeholder}
\end{figure}
\subsubsection{Quên mật khẩu}
\begin{figure}[H]
    \centering
    \includegraphics[width=0.7\linewidth]{Image/forgotpass.png}
    \caption{Trang quên mật khẩu}
    \label{fig:placeholder}
\end{figure}
\begin{figure}[H]
    \centering
    \includegraphics[width=0.7\linewidth]{Image/newpass.png}
    \caption{Cập nhật mật khẩu mới}
    \label{fig:placeholder}
\end{figure}
\subsubsection{Đăng ký tutor}
Sau khi đăng nhập sinh viên phải đăng ký Tutor
\begin{figure}[H]
    \centering
    \includegraphics[width=0.7\linewidth]{Image/đăng ký tutor.png}
    \caption{Bước 1: Nhập thông tin}
    \label{fig:placeholder}
\end{figure}
\begin{figure}[H]
    \centering
    \includegraphics[width=0.7\linewidth]{Image/buoc2.png}
    \caption{Bước 2: Chọn Tutor}
    \label{fig:placeholder}
\end{figure}
\begin{figure}[H]
    \centering
    \includegraphics[width=0.7\linewidth]{Image/buoc3.png}
    \caption{Bước 3: Chờ xác nhận}
    \label{fig:placeholder}
\end{figure}

\subsubsection{Trang chủ hệ thống của sinh viên}
\begin{figure}[H]
    \centering
    \includegraphics[width=0.7\linewidth]{Image/trangchu.png}
    \caption{Trang chủ hệ thống của sinh viên}
    \label{fig:placeholder}
\end{figure}

\begin{figure}[H]
    \centering
    \includegraphics[width=1\linewidth]{Image/huydk.png}
    \caption{Sinh viên có thể hủy đăng ký tutor trong vòng 12 giờ sau khi đăng ký}
    \label{fig:placeholder}
\end{figure}

\begin{figure}[H]
    \centering
    \includegraphics[width=0.7\linewidth]{Image/thanhcong.png}
    \caption{Hệ thống báo đăng ký thành công}
    \label{fig:placeholder}
\end{figure}


\subsubsection{Sinh viên}
\begin{figure}[H]
    \centering
    \includegraphics[width=0.7\linewidth]{Image/profile.png}
    \caption{Xem thông tin cá nhân}
    \label{fig:placeholder}
\end{figure}


\begin{figure}[H]
    \centering
    \includegraphics[width=0.7\linewidth]{Image/datlich1.png}
    \caption{Đặt lịch hẹn với Tutor}
    \label{fig:placeholder}
\end{figure}




\begin{figure}[H]
    \centering
    \includegraphics[width=0.7\linewidth]{Image/dktuvan.png}
    \caption{Đăng ký buổi tư vấn}
    \label{fig:placeholder}
\end{figure}



\begin{figure}[htbp]
  \centering
  \begin{minipage}{0.48\linewidth}
    \includegraphics[width=\linewidth]{Image/huy.png}\\[0.25em]
    \centering (a)
  \end{minipage}\hfill
  \begin{minipage}{0.48\linewidth}
    \includegraphics[width=\linewidth]{Image/huy1}\\[0.25em]
    \centering (b)
  \end{minipage}

  \caption{Hủy buổi gặp mặt.}
  \label{fig:tutor-reg-pair}
\end{figure}
\begin{figure}[H]
    \centering
    \includegraphics[width=0.7\linewidth]{Image/danh sach.png}
    \caption{Danh sách tài liệu}
    \label{fig:placeholder}
\end{figure}
\begin{figure}[H]
    \centering
    \includegraphics[width=0.7\linewidth]{Image/image.png}
    \caption{Phản hồi chất lượng}
    \label{fig:placeholder}
\end{figure}


\subsubsection{Tutor}

\begin{figure}[H]
    \centering
    \includegraphics[width=0.7\linewidth]{Image/hstutor.jpg}
    \caption{Hồ sơ cá nhân của tutor}
    \label{fig:placeholder}
\end{figure}



\begin{figure}[H]
    \centering
    \includegraphics[width=0.7\linewidth]{Image/tbtv.jpg}
    \caption{Tạo buổi tư vấn}
    \label{fig:placeholder}
\end{figure}


\begin{figure}[H]
    \centering
    \includegraphics[width=0.7\linewidth]{Image/lshd.jpg}
    \caption{Lịch sử hoạt động}
    \label{fig:placeholder}
\end{figure}


\begin{figure}[htbp]
  \centering
  \begin{minipage}{0.48\linewidth}
    \includegraphics[width=\linewidth]{Image/hbgm1.jpg}\\[0.25em]
    \centering (a)
  \end{minipage}\hfill
  \begin{minipage}{0.48\linewidth}
    \includegraphics[width=\linewidth]{Image/hbgm2.jpg}\\[0.25em]
    \centering (b)
  \end{minipage}

  \caption{Hủy buổi gặp mặt.}
  \label{fig:tutor-reg-pair}
\end{figure}




\begin{figure}[htbp]
  \centering
  \begin{minipage}{0.48\linewidth}
    \includegraphics[width=\linewidth]{Image/lr1.jpg}\\[0.25em]
    \centering (a)
  \end{minipage}\hfill
  \begin{minipage}{0.48\linewidth}
    \includegraphics[width=\linewidth]{Image/lr2.jpg}\\[0.25em]
    \centering (b)
  \end{minipage}

  \caption{Thiết lập lịch rảnh.}
  \label{fig:tutor-reg-pair}
\end{figure}





\begin{figure}[htbp]
  \centering
  \begin{minipage}{0.5\linewidth}
    \includegraphics[width=\linewidth]{Image/gn1.jpg}\\[0.25em]
    \centering (a)
  \end{minipage}\hfill
  \begin{minipage}{0.5\linewidth}
    \includegraphics[width=\linewidth]{Image/gn2.jpg}\\[0.25em]
    \centering (b)
  \end{minipage}

  \vspace{0.6em}

  \begin{minipage}{0.5\linewidth}
    \includegraphics[width=\linewidth]{Image/gn3.jpg}\\[0.25em]
    \centering (c)
  \end{minipage}\hfill
  \begin{minipage}{0.5\linewidth}
    \includegraphics[width=\linewidth]{Image/gn4.jpg}\\[0.25em]
    \centering (d)
  \end{minipage}

  \caption{Ghi nhận tiến độ sinh viên.}
  \label{fig:tutor-reg-cluster}
\end{figure}

\begin{figure}[htbp]
  \centering
  % Hàng 1: 2 ảnh
  \begin{minipage}{0.48\linewidth}
    \includegraphics[width=\linewidth]{Image/ycdlh1.jpg}\\[0.25em]
    \centering (a)
  \end{minipage}\hfill
  \begin{minipage}{0.48\linewidth}
    \includegraphics[width=\linewidth]{Image/ycdhl2.jpg}\\[0.25em]
    \centering (b)
  \end{minipage}

  \vspace{0.6em}

  % Hàng 2: 1 ảnh, canh giữa
  \begin{minipage}{0.6\linewidth}
    \includegraphics[width=\linewidth]{Image/ycdhl3.jpg}\\[0.25em]
    \centering (c)
  \end{minipage}

  \caption{Xử lý yêu cầu đặt lịch hẹn.}
  \label{fig:tutor-reg-cluster}
\end{figure}


\begin{figure}[htbp]
  \centering
  \begin{minipage}{0.48\linewidth}
    \includegraphics[width=\linewidth]{Image/dttl1.jpg}\\[0.25em]
    \centering (a)
  \end{minipage}\hfill
  \begin{minipage}{0.48\linewidth}
    \includegraphics[width=\linewidth]{Image/dttl2.jpg}\\[0.25em]
    \centering (b)
  \end{minipage}

  \caption{Đăng tải tài liệu.}
  \label{fig:tutor-reg-pair}
\end{figure}





\begin{figure}[htbp]
  \centering
  \begin{minipage}{0.48\linewidth}
    \includegraphics[width=\linewidth]{Image/xtl1.jpg}\\[0.25em]
    \centering (a)
  \end{minipage}\hfill
  \begin{minipage}{0.48\linewidth}
    \includegraphics[width=\linewidth]{Image/xtl2.jpg}\\[0.25em]
    \centering (b)
  \end{minipage}

  \caption{Xóa tài liệu.}
  \label{fig:tutor-reg-pair}
\end{figure}






\subsubsection{Điều phối viên}

\begin{figure}[H]
    \centering
    \includegraphics[width=0.7\linewidth]{Image/gd_coor.jpg}
    \caption{Giao diện chính}
    \label{fig:placeholder}
\end{figure}



\begin{figure}[htbp]
  \centering
  % Hàng 1: 2 ảnh
  \begin{minipage}{0.48\linewidth}
    \includegraphics[width=\linewidth]{Image/qlkct1.jpg}\\[0.25em]
    \centering (a)
  \end{minipage}\hfill
  \begin{minipage}{0.48\linewidth}
    \includegraphics[width=\linewidth]{Image/qlkct2.jpg}\\[0.25em]
    \centering (b)
  \end{minipage}

  \vspace{0.6em}

  % Hàng 2: 1 ảnh (canh giữa)
  \begin{minipage}{0.6\linewidth}
    \includegraphics[width=\linewidth]{Image/qlkct3.jpg}\\[0.25em]
    \centering (c)
  \end{minipage}

  \caption{Quản lý khung chương trình chung cho các Tutor.}
  \label{fig:tutor-reg-cluster}
\end{figure}




\subsection{State-chart diagram}
\subsubsection{Yêu cầu đặt lịch hẹn}
\begin{figure}[H]
    \centering
    \includegraphics[width=1\linewidth]{Image/State_XulyDatlichhen.png}
    \caption{State Diagram cho yêu cầu đặt lịch hẹn}
\end{figure}

\begin{table}[h!]
\centering
\caption{Bảng mô tả trạng thái}
\begin{tabular}{|p{3.5cm}|p{11cm}|}
\hline
\textbf{Trạng thái} & \textbf{Mô tả} \\ \hline
Đang tạo yêu cầu & Sinh viên đang nhập thông tin để tạo yêu cầu. \\ \hline
Chờ duyệt & Yêu cầu đã được gửi thành công và đang chờ Tutor duyệt. \\ \hline
Đã duyệt & Yêu cầu được Tutor duyệt. \\ \hline
Bị từ chối & Yêu cầu đặt lịch bị từ chối vì lý do nào đó. \\ \hline
Bị lỗi & Xảy ra lỗi hệ thống. \\ \hline
\end{tabular}
\end{table}
\begin{table}[h!]
\centering
\caption{Bảng Stimulus (Event/Transition)}
\begin{tabular}{|p{3.5cm}|p{11cm}|}
\hline
\textbf{Stimulus} & \textbf{Mô tả} \\ \hline

submit (valid Input): save Request &
Sinh viên gửi yêu cầu với dữ liệu hợp lệ, hệ thống lưu yêu cầu vào cơ sở dữ liệu và chuyển sang trạng thái chờ duyệt. \\ \hline

submit (invalid Input): show Error &
Sinh viên gửi yêu cầu nhưng dữ liệu không hợp lệ. Hệ thống hiển thị lỗi và giữ nguyên trạng thái tạo yêu cầu. \\ \hline



Tutor từ chối  &
Tutor từ chối yêu cầu đặt lịch của sinh viên và cung cấp lý do chính đáng. Hệ thống cập nhật trạng thái thành “Đã hủy”. \\ \hline

Tutor kiểm tra và duyệt  &
Tutor truy cập vào hệ thống và duyệt yêu cầu đặt lịch của sinh viên. Hệ thống cập nhật trạng thái thành "Đã duyệt"\\ \hline

submit (Database Error) &
Lỗi hệ thống khi lưu dữ liệu vào cơ sở dữ liệu. \\ \hline



\end{tabular}
\end{table}
\subsubsection{Thiết lập lịch rảnh}
\begin{figure}[H]
    \centering
    \includegraphics[width=1\linewidth]{Image/ST Lịch rảnh.drawio.png}
    \caption{State Diagram cho thiết lập lịch rảnh}
\end{figure}
\begin{table}[h!]
\centering
\caption{Bảng mô tả trạng thái trong UC thiết lập lịch rảnh}
\begin{tabular}{|p{4cm}|p{10.5cm}|}
\hline
\textbf{Trạng thái} & \textbf{Mô tả} \\ \hline

Đang tạo lịch rảnh &
Sinh viên đang nhập thông tin khung giờ rảnh và hệ thống hiển thị form để người dùng điền. Hệ thống chờ người dùng xác nhận. \\ \hline

Chờ xác nhận &
Hệ thống đã nhận yêu cầu hợp lệ và đang xử lý. Hệ thống hiển thị trạng thái chờ trong thời gian ngắn (đếm ngược). \\ \hline

Đã thêm thành công &
Yêu cầu tạo lịch rảnh đã được lưu thành công. Hệ thống gửi thông báo thành công đến người dùng. \\ \hline

Bị lỗi &
Quá trình gửi yêu cầu hoặc xử lý gặp lỗi (timeout, mất kết nối, lỗi hệ thống). Hệ thống hiển thị thông báo lỗi và yêu cầu thực hiện lại. \\ \hline

Bị huỷ &
Người dùng tự hủy thao tác hoặc hệ thống hủy yêu cầu do lỗi nghiêm trọng. Thông báo hủy được gửi đến người dùng. \\ \hline

\end{tabular}
\end{table}
\begin{table}[h!]
\centering
\caption{Bảng Stimulus (Event/Transition) trong UC thiết lập lịch rảnh}
\begin{tabular}{|p{4cm}|p{10.5cm}|}
\hline
\textbf{Stimulus / Event} & \textbf{Mô tả} \\ \hline

Xác nhận (invalid) : showError() &
Người dùng nhấn nút xác nhận nhưng dữ liệu nhập không hợp lệ. Hệ thống hiển thị lỗi và giữ nguyên trạng thái Đang tạo lịch rảnh. \\ \hline

Xác nhận (valid) : saveRequest() &
Người dùng gửi yêu cầu hợp lệ. Hệ thống lưu yêu cầu và chuyển sang trạng thái Chờ xác nhận. \\ \hline

Timeout : showErrorMessage() &
Trong trạng thái Chờ xác nhận, hệ thống bị timeout xử lý. Hệ thống chuyển sang trạng thái Bị lỗi. \\ \hline

Success : sendSuccessNotification() &
Yêu cầu đã được xử lý thành công trong trạng thái Chờ xác nhận. Hệ thống chuyển sang trạng thái Đã thêm thành công. \\ \hline

Mất kết nối : showErrorMessage() &
Khi đang xử lý, hệ thống mất kết nối. Hệ thống chuyển sang trạng thái Bị lỗi. \\ \hline

Hủy : cancelRequest() &
Người dùng chủ động hủy thao tác. Hệ thống chuyển sang trạng thái Bị hủy. \\ \hline

\end{tabular}
\end{table}
\subsubsection{Đăng ký Tutor}
\begin{figure}[H]
    \centering
    \includegraphics[width=1\linewidth]{Image/phanhoi-Trang-5.drawio.png}
    \caption{State Diagram cho yêu cầu đăng ký Tutor}
\end{figure}

\begin{table}[H]
\centering
\caption{Bảng mô tả trạng thái}
\begin{tabular}{|p{3.5cm}|p{11cm}|}
\hline
\textbf{Trạng thái} & \textbf{Mô tả} \\ \hline
Đang tạo yêu cầu & Sinh viên đang nhập thông tin để tạo yêu cầu. \\ \hline
Chờ duyệt & Yêu cầu đã được gửi thành công và đang chờ hệ thống tự động duyệt trong vòng 12 giờ. \\ \hline
Đã duyệt & Yêu cầu được hệ thống tự động duyệt sau 12 giờ và không bị sinh viên bấm hủy. \\ \hline
Bị hủy & Sinh viên hủy yêu cầu (trong lúc tạo hoặc trong 12 giờ sau khi gửi). \\ \hline
Bị lỗi & Xảy ra lỗi hệ thống hoặc mất kết nối mạng. \\ \hline
\end{tabular}
\end{table}
\begin{table}[H]
\centering
\caption{Bảng Stimulus (Event/Transition)}
\begin{tabular}{|p{3.5cm}|p{11cm}|}
\hline
\textbf{Stimulus} & \textbf{Mô tả} \\ \hline

nộp [valid] / saveRequest() &
Sinh viên gửi yêu cầu với dữ liệu hợp lệ, hệ thống lưu yêu cầu vào cơ sở dữ liệu và chuyển sang trạng thái chờ duyệt. \\ \hline

nộp [invalid] / showError() &
Sinh viên gửi yêu cầu nhưng dữ liệu không hợp lệ. Hệ thống hiển thị lỗi và giữ nguyên trạng thái tạo yêu cầu. \\ \hline

Hủy tạo yêu cầu &
Sinh viên hủy khi đang tạo yêu cầu (yêu cầu chưa được lưu). \\ \hline

Hủy [time < 12h]  &
Sinh viên hủy yêu cầu sau khi đã gửi (trong 12 giờ). Hệ thống cập nhật trạng thái thành “Đã hủy”. \\ \hline

duyệt tự động [sau 12h và chưa bị hủy]  &
Hệ thống tự động duyệt yêu cầu sau 12 giờ nếu sinh viên không hủy. \\ \hline

Lỗi mạng &
Lỗi kết nối trong quá trình gửi yêu cầu, hệ thống thông báo lỗi. \\ \hline

nộp [lỗi CSDL] &
Lỗi hệ thống khi lưu dữ liệu vào cơ sở dữ liệu. \\ \hline



\end{tabular}
\end{table}
\subsubsection{Đăng tải và xóa tài liệu}
\begin{figure}[H]
    \centering
    \includegraphics[width=1\linewidth]{Image/state_file.drawio (2).png}
    \caption{State Diagram cho quản lý tài liệu (Đăng và Xóa)}
\end{figure}

\begin{table}[h!]
\centering
\caption{\textbf{Bảng mô tả trạng thái trong UC Quản lý tài liệu (Đăng và Xóa)}}
\begin{tabular}{|p{4cm}|p{10.5cm}|}
\hline
\textbf{Trạng thái} & \textbf{Mô tả} \\ \hline

Idle &
Trạng thái ban đầu hoặc sau khi giảng viên mở danh sách tài liệu đã đăng. Hệ thống chờ người dùng chọn tệp để đăng hoặc thao tác tiếp theo. \\ \hline

Đang đăng &
Hệ thống đang tải lên tệp mà giảng viên chọn. Trong trạng thái này, hệ thống hiển thị tiến trình tải và thực hiện hành động tải tệp thực tế. \\ \hline

Đã đăng &
Tệp đã được tải lên thành công và lưu vào thư viện tài liệu. Hệ thống hiển thị thông báo “Tải lên hoàn tất” và chờ hành động tiếp theo từ giảng viên (như xóa hoặc kết thúc). \\ \hline

Đang xóa &
Giảng viên chọn hành động xóa tài liệu đã đăng. Hệ thống hiển thị tiến trình xóa, thực hiện thao tác xóa và xác nhận kết quả trước khi thoát khỏi trạng thái này. \\ \hline

Đã xóa &
Tài liệu đã được xóa thành công. Hệ thống dọn dẹp tài nguyên, ghi nhận log hành động và kết thúc use case. \\ \hline

Lỗi &
Có lỗi xảy ra trong quá trình tải lên hoặc xóa (ví dụ: lỗi tệp, lỗi quyền truy cập, hoặc sự cố mạng). Hệ thống hiển thị thông báo lỗi, cho phép người dùng thử lại hoặc thoát. \\ \hline

\end{tabular}
\end{table}

\begin{table}[h!]
\centering
\caption{\textbf{Bảng mô tả Sự kiện/Chuyển trạng thái trong UC Quản lý tài liệu (Đăng và Xóa)}}
\begin{tabular}{|p{4cm}|p{10.5cm}|}
\hline
\textbf{Sự kiện / Kích thích} & \textbf{Mô tả} \\ \hline

Tutor chọn file + Submit &
Giảng viên chọn tệp cần đăng và gửi yêu cầu. Hệ thống chuyển từ trạng thái Idle sang Đang đăng. \\ \hline

Thành công / saveToLibrary() &
Khi quá trình tải lên hoàn tất thành công, hệ thống lưu tài liệu vào thư viện và chuyển sang trạng thái Đã đăng. \\ \hline

Thất bại (File, Quyền, Mạng) &
Nếu có lỗi trong quá trình tải lên, hệ thống chuyển sang trạng thái Lỗi. Người dùng có thể thử lại thao tác. \\ \hline

Tutor mở danh sách file đã đăng &
Từ trạng thái Idle, giảng viên có thể mở danh sách tài liệu đã đăng để xem hoặc xóa. \\ \hline

Tutor chọn Xóa các file đã đăng &
Giảng viên chủ động chọn hành động “Xóa tài liệu”. Hệ thống chuyển sang trạng thái Đang xóa và thực hiện thao tác xóa. \\ \hline

Thành công / logToLibrary() &
Sau khi xóa thành công, hệ thống ghi log và chuyển sang trạng thái Đã xóa. \\ \hline

Thất bại (Quyền, Đang dùng) &
Nếu không thể xóa do lỗi quyền hoặc tài liệu đang được sử dụng, hệ thống chuyển sang trạng thái Lỗi. \\ \hline

Thử lại &
Từ trạng thái Lỗi, giảng viên chọn “Thử lại” → hệ thống quay lại Đang đăng để thực hiện lại quy trình tải lên. \\ \hline

Thoát &
Từ trạng thái Lỗi hoặc sau khi hoàn thành thao tác, giảng viên chọn “Thoát” → hệ thống kết thúc. \\ \hline

Kết thúc / Không xóa &
Giảng viên không chọn xóa hoặc kết thúc sau khi tài liệu đã được tải lên → hệ thống kết thúc. \\ \hline

\end{tabular}
\end{table}
% % %%%%%%%%%%%%%%


\section{Task 3}
\subsection{Deployment view}
\begin{figure} [H]
    \centering
    \includegraphics[width=1\linewidth]{Image/deploymentdiagram.drawio.png}
    \caption{Deployment Diagram cho hệ thống}
    \label{fig:placeholder}
\end{figure}

\begin{enumerate}
    \item Client Computer (Máy tính máy khách)
    \begin{itemize}
        \item Thiết bị này đại diện cho phía client, nơi người dùng cuối truy cập vào hệ thống thông qua web browser (trình duyệt web).
        \item Giao diện phía client được hiện thực bằng HTML 5 (cùng với CSS, JavaScript), là artifact được tải về từ Web Server và render bởi trình duyệt.
        \item Component trình duyệt đảm nhiệm việc gửi yêu cầu tới máy chủ và hiển thị kết quả cho người dùng.
        \item Thiết bị này kết nối đến Web Server thông qua giao thức HTTPS (trên nền TCP/IP).
    \end{itemize}
    \item Web Server (Máy chủ Web)
    \begin{itemize}
        \item Máy chủ này chạy trên hệ điều hành Ubuntu 22.4 và sử dụng Nginx (một artifact phần mềm) làm máy chủ web.
        \item Phục vụ nội dung tĩnh: Lưu trữ và cung cấp artifact là Frontend Build (html, css, js) cho Client.
        \item Reverse Proxy: Nhận yêu cầu từ Client và chuyển tiếp (forward) các yêu cầu nghiệp vụ đến Application Server.
        \item Máy chủ Web giao tiếp với Client và Application Server qua giao thức HTTP/HTTPS (trên nền TCP/IP).
    \end{itemize}
    \item Application Server (Máy chủ Ứng dụng)
    \begin{itemize}
        \item Máy chủ này chạy trên hệ điều hành Ubuntu 22.4, phục vụ cho việc xử lý logic nghiệp vụ (backend) của hệ thống.
        \item Nó sử dụng JVM (Java Virtual Machine) làm môi trường thực thi để chạy server (máy chủ ứng dụng) Tomcat.
        \item Thành phần chính được triển khai là artifact my\_app.war (Web Application Archive), chứa toàn bộ logic nghiệp vụ (ví dụ: các lớp Service, Controller, và logic truy cập dữ liệu).
        \item Máy chủ Ứng dụng giao tiếp với Web Server, Database Server và hệ thống HCMUT\_SSO qua các giao thức trên nền TCP/IP.
    \end{itemize}
    \item Database Server (Máy chủ Cơ sở dữ liệu)
    \begin{itemize}
        \item Máy chủ này chạy trên hệ điều hành Ubuntu 22.4 và sử dụng MySQL làm hệ quản trị cơ sở dữ liệu (database system).
        \item Máy chủ Cơ sở dữ liệu kết nối với Máy chủ Ứng dụng qua giao thức JDBC (trên nền TCP/IP) để trao đổi dữ liệu.
    \end{itemize}
    \item HCMUT\_SSO (Hệ thống Xác thực)
    \begin{itemize}
        \item Đây là một External system (hệ thống bên ngoài) cung cấp dịch vụ xác thực tập trung (Single Sign-On).
        \item Nó cung cấp một hệ thống xác thực người dùng, phục vụ cho việc đăng nhập và bảo mật truy cập vào hệ thống, giúp ứng dụng không cần tự quản lý mật khẩu.
        \item Hệ thống xác thực kết nối với Máy chủ Ứng dụng qua giao thức HTTP/HTTPS (trên nền TCP/IP).
    \end{itemize}
\end{enumerate}
\subsection{Development/Implementation view}
\begin{figure}[H]
    \centering
    \includegraphics[width=1\linewidth]{Image/Developmentview.png}
    \caption{Component Diagram cho hệ thống}
    \label{fig:placeholder}
\end{figure}


\subsubsection{Subsystem 1: Authentication}
\textbf{Mục tiêu.}
Subsystem \emph{Authentication} chịu trách nhiệm xác thực người dùng và
kiểm soát truy cập dựa trên vai trò (role). Mọi yêu cầu cần đăng nhập,
kiểm tra token hoặc kiểm tra quyền đều đi qua subsystem này.


\begin{itemize}
    \item \textbf{Các component nội bộ:}
    \begin{itemize}
        \item \texttt{SSOIntegration}: kết nối tới hệ thống HCMUT\_SSO và
              HCMUT\_DATACORE để xác thực và đồng bộ thông tin người dùng.
        \item \texttt{TokenManager}: tạo và kiểm tra tính hợp lệ của token đăng nhập.
        \item \texttt{RoleManager}: quản lý role/permission và cung cấp chức năng
              kiểm tra quyền cho các action nghiệp vụ.
        \item \texttt{SessionManager}: quản lý phiên làm việc (session) của người dùng,
              ví dụ gia hạn hoặc huỷ session khi logout.
    \end{itemize}
\end{itemize}


%-------------------------------------------------------
\subsubsection{Subsystem 2: StudentManager}

\textbf{Mục tiêu.}
Subsystem \emph{StudentManager} quản lý các chức năng phía sinh viên như:
đăng ký tutor, tra cứu tutor
phù hợp, xem thông tin cá nhân,...
\begin{itemize}
\item\textbf{Các component nội bộ:}
\begin{itemize}
  \item \texttt{StudentService}: xử lý các thao tác liên quan đến hồ sơ sinh viên
        và giao diện sinh viên (xem thông tin bản thân, trạng thái tham gia).
  \item \texttt{TutorRegistration}: tiếp nhận và xử lý yêu cầu đăng ký tutor
        từ sinh viên.
  \item \texttt{MatchingEngine}: thực hiện ghép cặp sinh viên--tutor, đề xuất Tutor
\end{itemize}
\end{itemize}

%-------------------------------------------------------
\subsubsection{Subsystem 3: TutorManager}

\textbf{Mục tiêu.}
Subsystem \emph{TutorManager} quản lý thông tin tutor, lịch rảnh và tiến độ
học tập của sinh viên,... do tutor phụ trách.
\begin{itemize}
\item\textbf{Các component nội bộ:}
\begin{itemize}
  \item \texttt{ScheduleManager}: quản lý lịch rảnh và lịch làm việc của tutor.
  \item \texttt{TutorService}: xử lý các thao tác cập nhật thông tin, lịch rảnh của tutor.
  \item \texttt{ProgressTracker}: lưu trữ và truy vấn tiến độ học tập của sinh viên qua các buổi tư vấn.
\end{itemize}
\end{itemize}


%-------------------------------------------------------
\subsubsection{Subsystem 4: Appointment}

\textbf{Mục tiêu.}
Subsystem \emph{Appointment} quản lý toàn bộ vòng đời của cuộc hẹn
và buổi tư vấn (scheduling \& consultation), đồng thời cung cấp dịch vụ
gửi thông báo chung cho toàn hệ thống.
\begin{itemize}
\item\textbf{Các component nội bộ:}
\begin{itemize}
  \item \texttt{AppointmentService}: xử lý đặt lịch, xem lịch, huỷ lịch hẹn giữa sinh viên và tutor.
  \item \texttt{ConsultationService}: quản lý các buổi tư vấn theo nhóm hoặc lớp,
        cho phép tạo, đăng ký và huỷ tham gia.
  \item \texttt{NotificationService}: gửi email hoặc thông báo trong hệ thống
        tới sinh viên, tutor và coordinator.
\end{itemize}
\end{itemize}

\subsubsection{Subsystem 5: Document}

\textbf{Mục tiêu.}
Subsystem \emph{Document} quản lý tài liệu học tập được sử dụng trong
quá trình tư vấn, bao gồm lưu trữ nội bộ và tích hợp với hệ thống thư viện.
\begin{itemize}
\item\textbf{Các component nội bộ:}
\begin{itemize}
  \item \texttt{DocumentService}: cung cấp CRUD tài liệu trong hệ thống nội bộ.
  \item \texttt{LibraryIntegration}: đóng vai trò adapter kết nối với external
        system HCMUT\_LIBRARY.
  \item \texttt{FileUploadManager}: xử lý và kiểm tra file upload
        (định dạng, kích thước, \ldots).
\end{itemize}
\end{itemize}



%-------------------------------------------------------
\subsubsection{Subsystem 6: FeedbackManager}

\textbf{Mục tiêu.}
Subsystem \emph{FeedbackManager} thu thập feedback từ sinh viên,
quản lý dữ liệu tiến độ phục vụ đánh giá và tính toán các chỉ số
chất lượng hoạt động tutor.
\begin{itemize}
\item\textbf{Các component nội bộ.}
\begin{itemize}
  \item \texttt{FeedbackService}: tiếp nhận, lưu trữ và cung cấp dữ liệu feedback.
  \item \texttt{ProgressRecordService}: ghi nhận và truy xuất lịch sử tiến độ
        (view progress history) phục vụ đánh giá.
  \item \texttt{EvaluationManager}: kết hợp feedback và tiến độ để tính
        điểm chất lượng và các thống kê đánh giá.
\end{itemize}
\end{itemize}

%-------------------------------------------------------
\subsubsection{Subsystem 7: Coordinator}

\textbf{Mục tiêu.}
Subsystem \emph{Coordinator} chịu trách nhiệm điều phối \emph{khung chương trình chung}
(common program framework) cho hoạt động tutor và tổng hợp, gửi các báo cáo
thống kê . 
\begin{itemize}
\item\textbf{Các component nội bộ:}
 \begin{itemize}
  \item \texttt{CoordinatorService}: cung cấp API chính dành cho vai trò coordinator.
  \item \texttt{CurriculumManager}: quản lý và cập nhật khung chương trình chung,
        bao gồm cấu trúc buổi tư vấn, các chủ đề và quy tắc áp dụng.
  \item \texttt{ReportGenerator}: thu thập số liệu từ các subsystem khác
        và tạo báo cáo tổng hợp.
\end{itemize}
\end{itemize}

\subsection{Class diagram}
\begin{figure}[H]
    \centering
    \includegraphics[width=1\linewidth]{Image/CLASS DIAGRAM.drawio.png}
    \caption{Class Diagram cho hệ thống}
    \label{fig:placeholder}
\end{figure}

Xem rõ hơn \href{https://app.diagrams.net/#G1I1Oi3iTSW3VsruGY966R2kWD1AGPoxrH#%7B%22pageId%22%3A%22KZOnhYCO1evhBqCBx140%22%7D}{tại đây}.


\subsection{Design Pattern áp dụng}
\subsubsection{MVC mở rộng}

Hệ thống được xây dựng theo mô hình MVC mở rộng 5 tầng (Multi-layer MVC):
\begin{center}
View $\rightarrow$ Controller (API) $\rightarrow$ Service $\rightarrow$ Repository (Interface + Implementation) 


$\rightarrow$ Entity (Model)
\end{center}

So với MVC truyền thống (Model – View – Controller), phiên bản mở rộng bổ sung:
\begin{itemize}
    \item \textbf{Service Layer}: chứa toàn bộ logic nghiệp vụ (Business Logic).
    \item \textbf{Repository Layer}: phụ trách truy xuất dữ liệu, được thiết kế theo mô hình \textit{Interface + Implementation}:
    \begin{itemize}
        \item Interface (IRoomRepository, ISchedulingRepository, …): định nghĩa các phương thức trừu tượng.
        \item Implementation (RoomRepository, SchedulingRepository, …): hiện thực hóa logic truy vấn dữ liệu thật.
    \end{itemize}
\end{itemize}

\textbf{Lợi ích:}
\begin{itemize}
    \item Service chỉ phụ thuộc vào Interface → giảm coupling, dễ thay thế backend (SQL → NoSQL → file).
    \item Dễ unit test nhờ có thể mock các Repository.
    \item Đảm bảo áp dụng nguyên tắc Dependency Inversion (D) trong SOLID.
    \item Các mối quan tâm (Separation of Concerns) được tách biệt hoàn toàn, hệ thống dễ mở rộng, dễ bảo trì, và tránh phụ thuộc chéo giữa các tầng.
\end{itemize}

\subsubsection{Facade}

\textbf{Nhận diện Facade}\\
Trong hệ thống, interface \texttt{LearningFacade} được sử dụng như một \textbf{Facade Pattern}.  
Facade đóng vai trò là điểm truy cập duy nhất cho client như sinh viên, tutor hoặc nhóm hệ thống khác.  
Thông qua Facade, chi tiết phức tạp của các service được ẩn đi và việc tương tác trở nên đơn giản hơn.

Client chỉ làm việc với các phương thức:
\begin{itemize}
    \item \texttt{addCourse(dto)}
    \item \texttt{enroll(dto)}
    \item \texttt{submitRequest(dto)}
    \item \texttt{submitFeedback(dto)}
    \item \texttt{getCourses()}
\end{itemize}

\textbf{Các class bị giấu}\\
Facade che giấu các lớp nghiệp vụ sau:
\begin{itemize}
    \item \texttt{CourseService}
    \item \texttt{EnrollmentService}
    \item \texttt{RequestService}
    \item \texttt{FeedbackService}
\end{itemize}

\textbf{Lý do sử dụng Facade}\\
\begin{enumerate}
    \item Ẩn bớt sự phức tạp của nhiều tầng logic.
    \item Đơn giản hóa giao tiếp giữa client và hệ thống.
    \item Dễ bảo trì do chỉ cần thay đổi trong Facade khi logic thay đổi.
    \item Tăng tính bảo mật vì client không truy cập trực tiếp vào tầng nghiệp vụ.
    \item Hỗ trợ tích hợp giữa các nhóm phát triển dễ dàng hơn.
\end{enumerate}


\subsection{Mô tả phương thức}
\subsubsection{View Method}

\textbf{MeetingListUI}
\begin{itemize}
    \item displayMeetingList(meetings: List<Meeting>): void\\
    Hiển thị danh sách các buổi gặp mặt của người dùng.
    \item selectMeeting(meetingId: int): void\\
    Chọn một buổi gặp mặt để xem chi tiết hoặc thao tác (hủy).
\end{itemize}

\textbf{MeetingCancelUI}
\begin{itemize}
    \item displayCancellableMeetings(meetings: List<Meeting>): void\\
    Hiển thị danh sách các buổi gặp mặt có thể hủy (còn chưa diễn ra).
    \item showCancelForm(meetingId: int): void\\
    Hiển thị form nhập lý do hủy khi người dùng chọn nút hủy.
    \item submitCancelForm(meetingId: int, reason: String): void\\
    Gửi lý do hủy và thực hiện cập nhật trạng thái CANCELLED.
\end{itemize}

\textbf{AppointmentBookingUI}
\begin{itemize}
    \item showBookingForm(): void\\
    Hiển thị form đặt lịch hẹn.
    \item submitBookingForm(studentId: int, tutorId: int, date: String, startTime: String, topic: String): bool\\
    Gửi yêu cầu đặt lịch hẹn đến SchedulingAPI. Trả về true nếu thành công.
    \item displayAppointmentDetails(appointment: Appointment): void\\
    Hiển thị thông tin chi tiết của một lịch hẹn.
\end{itemize}

\textbf{AppointmentManagementUI}
\begin{itemize}
    \item displayPendingAppointments(tutorId: int): void\\
    Hiển thị danh sách các lịch hẹn đang chờ duyệt.
    \item approveAppointment(appointmentId: int): bool\\
    Duyệt một lịch hẹn, cập nhật trạng thái APPROVED và thêm vào Meeting list.
    \item rejectAppointment(appointmentId: int): bool\\
    Từ chối một lịch hẹn, cập nhật trạng thái REJECTED.
    \item showAppointmentDetails(appointmentId: int): void\\
    Hiển thị chi tiết thông tin lịch hẹn.
\end{itemize}

\textbf{ConsultationCreateUI}
\begin{itemize}
    \item showCreateForm(): void\\
    Hiển thị form tạo buổi tư vấn.
    \item submitConsultationForm(data: Map<String, Any>): bool\\
    Gửi dữ liệu tạo buổi tư vấn đến SchedulingAPI. Trả về true nếu thành công.
    \item selectRoom(): void\\
    Lấy danh sách phòng trống từ RoomAPI và chọn phòng (nếu offline).
    \item displayConsultationDetails(consultation: Consultation): void\\
    Hiển thị thông tin chi tiết của buổi tư vấn.
\end{itemize}

\textbf{ConsultationRegistrationUI}
\begin{itemize}
    \item showRegistrationForm(): void\\
    Hiển thị form đăng ký tham gia buổi tư vấn.
    \item submitRegistrationForm(consultationId: int, studentId: int): bool\\
    Gửi yêu cầu đăng ký buổi tư vấn. Trả về true nếu thành công.
    \item displayRegisteredConsultationDetails(consultation: Consultation): void\\
    Hiển thị thông tin chi tiết buổi tư vấn đã đăng ký.
\end{itemize}

\textbf{RoomUI}
\begin{itemize}
    \item displayAvailableRooms(date: String, startTime: String, endTime: String): void\\
    Hiển thị danh sách phòng trống theo thời gian.
    \item showRoomDetails(roomId): void\\
    Hiển thị chi tiết thông tin phòng cho người dùng.
\end{itemize}


\textbf{LoginUI}
\begin{itemize}
    \item \textbf{displayLoginForm(): void} \\
    Hiển thị form đăng nhập gồm username và password.
    \item \textbf{getUsernameInput(): String} \\
    Lấy giá trị username người dùng nhập.
    \item \textbf{getPasswordInput(): String} \\
    Lấy giá trị password người dùng nhập.
    \item \textbf{showLoginSuccessMessage(): void} \\
    Hiển thị thông báo đăng nhập thành công.
    \item \textbf{showLoginErrorMessage(msg: String): void} \\
    Hiển thị thông báo lỗi với nội dung \texttt{msg}.
    \item \textbf{showAccessDeniedMessage(): void} \\
    Hiển thị thông báo quyền truy cập bị từ chối.
    \item \textbf{onLoginButtonClick(callback: (username, password) $\rightarrow$ void): void} \\
    Gán callback khi người dùng nhấn nút đăng nhập.
\end{itemize}

\textbf{ForgotPasswordUI}
\begin{itemize}
    \item \textbf{displayForgotPasswordForm(): void} \\
    Hiển thị form nhập BKID để yêu cầu reset password.
    \item \textbf{getBKIDInput(): String} \\
    Lấy BKID người dùng nhập.
    \item \textbf{displayResetPasswordForm(token: String): void} \\
    Hiển thị form nhập mật khẩu mới kèm token.
    \item \textbf{getNewPasswordInput(): String} \\
    Lấy mật khẩu mới người dùng nhập.
    \item \textbf{showPasswordResetSuccessMessage(): void} \\
    Hiển thị thông báo reset password thành công.
    \item \textbf{showErrorMessage(msg: String): void} \\
    Hiển thị thông báo lỗi với nội dung \texttt{msg}.
    \item \textbf{onRequestResetClick(callback: (bkId) $\rightarrow$ void): void} \\
    Gán callback khi người dùng nhấn nút yêu cầu reset.
    \item \textbf{onResetPasswordClick(callback: (token, newPassword) $\rightarrow$ void): void} \\
    Gán callback khi người dùng nhấn nút xác nhận reset password.
\end{itemize}

\textbf{UserProfileUI}
\begin{itemize}
    \item \textbf{displayProfile(user: UserEntity, activities: List<ActivityEntity>, appointments: List<AppointmentEntity>): void} \\
    Hiển thị thông tin chi tiết user cùng lịch sử hoạt động và lịch hẹn.
    \item \textbf{showErrorMessage(msg: String): void} \\
    Hiển thị thông báo lỗi với nội dung \texttt{msg}.
    \item \textbf{onOpenProfile(callback: () $\rightarrow$ void): void} \\
    Gán callback khi người dùng mở profile.
\end{itemize}

\textbf{LogoutUI}
\begin{itemize}
    \item \textbf{displayLogoutConfirmation(): void} \\
    Hiển thị hộp thoại yêu cầu xác nhận đăng xuất.
    \item \textbf{showLogoutSuccessMessage(): void} \\
    Hiển thị thông báo đăng xuất thành công.
    \item \textbf{showErrorMessage(msg: String): void} \\
    Hiển thị thông báo lỗi với nội dung \texttt{msg}.
    \item \textbf{onConfirmLogout(callback: () $\rightarrow$ void): void} \\
    Gán callback khi người dùng xác nhận logout.
    \item \textbf{onCancelLogout(callback: () $\rightarrow$ void): void} \\
    Gán callback khi người dùng hủy thao tác logout.
\end{itemize}

\textbf{MaterialUI}
\begin{itemize}
    \item \textbf{displayMaterialList(materials: List<MaterialEntity>): void} \\
    Hiển thị danh sách các tài liệu.
    \item \textbf{displayMaterialContent(materialId: String, content: String): void} \\
    Hiển thị nội dung chi tiết của tài liệu.
    \item \textbf{showErrorMessage(msg: String): void} \\
    Hiển thị thông báo lỗi.
    \item \textbf{onSelectMaterial(callback: (materialId: String) $\rightarrow$ void): void} \\
    Gán callback khi người dùng chọn tài liệu.
    \item \textbf{onAdvancedSearch(callback: (criteria: SearchCriteria) $\rightarrow$ void): void} \\
    Gán callback khi người dùng thực hiện tìm kiếm nâng cao.
\end{itemize}

\textbf{TutorSlotCalendarView}
\begin{itemize}
    \item \textbf{renderCalendar(tutorID: String, slots: List): void} \\
    Hiển thị lịch của gia sư (tutorID) với các slot (slots) đã cung cấp.
    \item \textbf{showAddSlotModal(): void} \\
    Hiển thị cửa sổ (modal) để cho phép gia sư thêm một slot thời gian mới.
    \item \textbf{showEditSlotModal(slot: AvailableSlot): void} \\
    Hiển thị cửa sổ (modal) để chỉnh sửa thông tin một slot đã có.
    \item \textbf{enableDragAndDrop(): void} \\
    Kích hoạt chức năng kéo-thả trên lịch để di chuyển hoặc thay đổi thời lượng slot.
    \item \textbf{highlightAvailableSlots(): void} \\
    Làm nổi bật các slot thời gian còn trống trên lịch.
    \item \textbf{onSlotDelete(callback: (slotID: String) $\rightarrow$ void): void} \\
    Gán callback khi gia sư thực hiện hành động xóa một slot.
\end{itemize}

\textbf{StudentTutorSelectionView}
\begin{itemize}
    \item \textbf{renderTutorCard(tutorID: String, name: String, expertise: String): void} \\
    Hiển thị thông tin của một gia sư dưới dạng thẻ (card), bao gồm ID, tên, và chuyên môn.
    \item \textbf{onTutorClick(callback: (tutorID: String) $\rightarrow$ void): void} \\
    Gán callback khi sinh viên nhấp vào thẻ của một gia sư (để xem chi tiết hoặc lịch).
    \item \textbf{renderSlotList(slots: List): void} \\
    Hiển thị danh sách các slot thời gian (thường là của gia sư vừa được chọn).
    \item \textbf{disableBookedSlots(): void} \\
    Vô hiệu hóa (làm mờ hoặc không cho phép nhấp) các slot đã được đặt.
    \item \textbf{onSlotSelect(callback: (slotID: String) $\rightarrow$ void): void} \\
    Gán callback khi sinh viên chọn một slot thời gian (để đặt lịch).
\end{itemize}


\textbf{MaterialUploadUI}
\begin{itemize}
    \item \textbf{chooseFile(): void} \\
    Hiển thị cửa sổ/dialog để người dùng chọn tệp (file) từ máy tính.
    \item \textbf{fillForm(title: String, type: String, classId: String): void} \\
    Cho phép người dùng điền thông tin mô tả cho tài liệu (tiêu đề, loại, ID lớp).
    \item \textbf{submitUpload(tutorId: int): void} \\
    Gửi thông tin và tệp đã chọn lên hệ thống (thông qua API) để tải lên.
    \item \textbf{submitReviewRequest(materialId: int): void} \\
    Gửi yêu cầu xem xét (review) cho một tài liệu đã tải lên.
    \item \textbf{listMyUploads(tutorId: String): void} \\
    Hiển thị danh sách các tài liệu mà gia sư này đã tải lên.
    \item \textbf{showStatus(msg: String): void} \\
    Hiển thị thông báo trạng thái (ví dụ: "Tải lên thành công", "Lỗi") cho người dùng.
\end{itemize}



\textbf{CourseListUI}  
\begin{itemize}
    \item renderCourseList(courses)\\
    Hiển thị danh sách dạng card hoặc grid.
    \item showEnrollButton(courseID)\\
    Hiển thị nút đăng ký khi học viên chưa đăng ký.
\end{itemize}

\textbf{CourseDetailUI} 
\begin{itemize}
    \item renderCourseDetail(course)\\
    Hiển thị tiêu đề, mô tả, tutor và số lượng sinh viên.  
    \item displayTutorInfo(tutorID)\\
    Hiển thị thông tin chi tiết về giảng viên.
\end{itemize}

\textbf{RequestFormUI}  
\begin{itemize}
    \item showForm(tutors, courses)\\
    Tải danh sách tutor và khóa học lên giao diện.   
    \item onSubmit(requestDTO)\\
    Gửi yêu cầu khi người dùng nhấn nút \texttt{Gửi}.
\end{itemize}


\textbf{RequestListUI}
\begin{itemize}
    \item renderMyRequests(requests)\\
    Liệt kê các yêu cầu đã gửi của sinh viên.   
    \item onCancelRequest(requestID)\\
    Cho phép hủy yêu cầu nếu trạng thái đang là \texttt{PENDING}.
\end{itemize}



\textbf{TutorRequestListView}
\begin{itemize}
    \item renderPendingRequests(requests)\\
    Hiển thị danh sách yêu cầu cần duyệt.   
    \item onAccept(requestID)\\
    Chấp nhận yêu cầu và tạo buổi học mới.
    \item  onReject(requestID)\\
    Từ chối yêu cầu và gửi email thông báo lý do.
\end{itemize}


\textbf{NotificationUI}
\begin{itemize}
    \item load(): void\\
    Mở màn hình thông báo và nạp dữ liệu ban đầu (danh sách thông báo gần đây, số lượng chưa đọc).   
    \item render(): void\\
    Cập nhật và hiển thị giao diện danh sách thông báo theo dữ liệu hiện có.
    \item  handleManualSend(): void\\
    Xử lý thao tác gửi thông báo thủ công.
    \item handleAutoSend(): void\\
   Xử lý thao tác gửi thông báo tự động.
    \item markAsRead(userId: int, notificationId: int): void\\
    Đánh dấu một thông báo của người dùng là “đã đọc”.
\end{itemize}

\textbf{ReportUI}
\begin{itemize}
    \item  handleManualSend(report: Report, requestedBy: Department): void\\
    Gửi báo cáo thủ công đến phòng ban yêu cầu.
    \item handleAutoSend(report: Report, requestedBy: Department, periods: String): void\\
    Gửi báo cáo tự động theo chu kỳ đến phòng ban yêu cầu.
    \item displayReports(reports: List<Report>): void\\
    Hiển thị danh sách các báo cáo đã tạo.
    \item displayReport(report: Report): void\\
    Hiển thị chi tiết một báo cáo cụ thể.
    \item downloadReport(reportId: int, format: ReportFormat): void\\
    \item Tải về báo cáo ở định dạng mong muốn.
\end{itemize}

%%%%%%%%%%%%%%%%%%%%%%%
\subsubsection{Controller Method}

\textbf{TutorSchedulingAPI}
\begin{itemize}
    \item pending(tutorId: Long): List<Appointment>\\
    Trả về danh sách các Appointment đang ở trạng thái PENDING của tutor.

    \item approve(id: Long, req: ApproveRequest): String\\
    Tutor phê duyệt một Appointment theo ID. Trả về thông báo thành công hoặc lỗi.

    \item reject(id: Long, req: RejectRequest): String\\
    Tutor từ chối Appointment kèm lý do. Lý do phải không rỗng.

    \item official(tutorId: Long): List<Meeting>\\
    Trả về danh sách các Meeting đã được xác nhận (APPROVED) và còn hiệu lực của tutor.

    \item cancel(id: Long, req: CancelRequest): String\\
    Tutor hủy một Meeting kèm lý do hủy.

    \item returnSlot(tutorId: Long, meetingId: Long): String\\
    Tutor chọn trả hoặc không trả lại slot đã bị hủy vào lịch rảnh.

    \item detail(id: Long): Meeting\\
    Xem chi tiết một Meeting theo ID.
\end{itemize}
\textbf{StudentSchedulingAPI}
\begin{itemize}
    \item book(req: AppointmentRequest): String\\
    Sinh viên đặt lịch hẹn mới với tutor. Trả về thông báo thành công hoặc thất bại.

    \item getHistory(studentId: Long): List<Appointment>\\
    Trả về toàn bộ lịch sử Appointment của sinh viên.

    \item getCancelableMeetings(studentId: Long): List<Meeting>\\
    Trả về các Meeting sinh viên được phép hủy.

    \item cancel(id: Long, req: CancelRequest): String\\
    Sinh viên hủy một Meeting kèm lý do.

    \item getOfficial(studentId: Long): List<Meeting>\\
    Trả về danh sách Meeting đã được xác nhận và còn hiệu lực của sinh viên.

    \item viewSlots(tutorId: Long): List<FreeSlotResponse>\\
    Xem danh sách các slot rảnh của tutor để đặt lịch.

    \item detail(id: Long): Meeting\\
    Xem chi tiết một Meeting theo ID.
\end{itemize}

\textbf{RoomAPI}
\begin{itemize}
    \item getAvailableRooms(date: Date, startTime: Time, endTime: Time): ResponseEntity<List<Room>>\\
    Trả về danh sách phòng trống theo yêu cầu từ UI.
    \item  reserveRoom(Long roomId,  ReserveRequest request): ResponseEntity<RoomBooking>\\
    Cập nhật trạng thái phòng.
    \item getRoomDetails(roomId: int): ResponseEntity<Room>\\
    Lấy thông tin chi tiết của một phòng.
    \item releaseRoom(roomId, meetingId): ResponseEntity<Void>\\
    Hủy đặt phòng theo mã cuộc họp.
\end{itemize}

\textbf{LearningAPI}
\begin{itemize}
    \item addCourse(courseData: Course): Course\\
    Thêm khóa học mới.
    \item getAllCourses(): List<CourseDTO>\\
    Trả về danh sách khóa học ở dạng DTO.
    \item getCourseById(courseId: int): CourseDetail\\
    Lấy thông tin chi tiết khóa học.
    \item registerCourse(studentId: int, courseId: int): Enrollment\\
    Cho phép sinh viên đăng ký khóa học.
    \item getEnrollmentsByStudent(studentId: int): List<Enrollment>\\
    Trả về danh sách đăng ký của sinh viên.
    \item submitRequest(requestData: Request): Request\\
    Xử lý gửi yêu cầu tư vấn.
    \item getMyRequests(userId: int): List<Request>\\
    Trả về danh sách yêu cầu của người dùng hiện tại.
    \item getRequestsByTutor(tutorId: int): List<Request>\\
    Trả về danh sách yêu cầu dành cho tutor.
    \item updateRequestStatus(requestId: int, newStatus: String): bool\\
    Cập nhật trạng thái yêu cầu.
    \item submitFeedback(feedbackData: Feedback): bool\\
    Nhận phản hồi sau buổi học.
\end{itemize}


\textbf{LoginAPI}
\begin{itemize}
    \item \textbf{login(username: String, password: String): void} \\
    Gọi SSOService để xác thực, tạo session và thông báo kết quả lên view.
\end{itemize}

\textbf{ForgotPasswordAPI}
\begin{itemize}
    \item \textbf{requestPasswordReset(bkId: String): void} \\
    Gửi yêu cầu reset password và thông báo kết quả.
    \item \textbf{resetPassword(token: String, newPassword: String): void} \\
    Thực hiện reset password dựa trên token và mật khẩu mới.
\end{itemize}

\textbf{UserProfileManageAPI}
\begin{itemize}
    \item \textbf{viewProfile(): void} \\
    Lấy userId từ session, gọi ProfileService để lấy thông tin user, hoạt động, lịch hẹn, hiển thị lên ProfileView.
\end{itemize}

\textbf{LogoutAPI}
\begin{itemize}
    \item \textbf{logout(): void} \\
    Vô hiệu hóa session và thông báo kết quả lên view.
\end{itemize}

\textbf{MaterialAPI}
\begin{itemize}
    \item \textbf{viewMaterialList(): void} \\
    Lấy danh sách tài liệu từ MaterialService và hiển thị.
    \item \textbf{viewMaterial(materialId: String): void} \\
    Lấy nội dung chi tiết tài liệu và hiển thị.
    \item \textbf{searchMaterials(criteria: SearchCriteria): void} \\
    Thực hiện tìm kiếm tài liệu theo tiêu chí và hiển thị kết quả.
\end{itemize}
\textbf{SlotAPI}
\begin{itemize}
    \item \textbf{addSlot(tutorID: String, slotDTO: SlotDTO): ResponseEntity} \\
    Xử lý yêu cầu HTTP (ví dụ: POST) để thêm một slot mới cho gia sư. Gọi đến \texttt{AvailableSlotService}.
    \item \textbf{updateSlot(tutorID: String, slotID: String, slotDTO: SlotDTO): ResponseEntity} \\
    Xử lý yêu cầu HTTP (ví dụ: PUT) để cập nhật thông tin một slot đã có.
    \item \textbf{deleteSlot(tutorID: String, slotID: String): ResponseEntity} \\
    Xử lý yêu cầu HTTP (ví dụ: DELETE) để xóa một slot.
    \item \textbf{getTutorSlots(tutorID: String): ResponseEntity} \\
    Xử lý yêu cầu HTTP (ví dụ: GET) để lấy về danh sách các slot của một gia sư.
\end{itemize}

\textbf{MaterialUploadAPI}
\begin{itemize}
    \item \textbf{createDraft(tutorId: int, title: String, type: String, classId: String): Material} \\
    Xử lý yêu cầu HTTP tạo một bản nháp tài liệu, gọi \texttt{MaterialUploadService}.
    \item \textbf{attachFile(materialId: int, fileName: String, file: File): Material} \\
    Xử lý yêu cầu HTTP đính kèm tệp vào một bản nháp tài liệu đã có.
    \item \textbf{submitForReview(materialId: int): void} \\
    Xử lý yêu cầu HTTP gửi tài liệu để xem xét.
    \item \textbf{listMyUploads(tutorId: int): List<Material>} \\
    Xử lý yêu cầu HTTP lấy danh sách các tài liệu đã tải lên của gia sư.
\end{itemize}

\textbf{NotificationAPI}
\begin{itemize}
\item \textbf{notifyScheduleRequested(studentId: int, tutorId: int, status: String): void} \
Gửi thông báo về yêu cầu đặt lịch của sinh viên (tạo mới/được duyệt/bị từ chối) đến người nhận phù hợp.
\item \textbf{notifyMeeting(meeting: Meeting, studentId: int, tutorId: int): void} \
Thông báo liên quan tới buổi tư vấn: tạo mới, thay đổi thời gian/phòng hoặc hủy.
\item \textbf{notifyMaterial(material: Material, tutorId: int, studentId: int, status: String): void} \
Thông báo trạng thái tài liệu (đã duyệt/bị từ chối/được công bố) cho gia sư hoặc sinh viên tuỳ trường hợp.
\item \textbf{notifyCourseUpdated(course: CourseEntity, tutorId: int): void} \
Báo cho gia sư (và/hoặc lớp học liên quan) khi nội dung/kế hoạch khoá học được cập nhật.
\item \textbf{getMyNotifications(userId: String): List<Notification>} \
Lấy danh sách thông báo của người dùng để hiển thị trong hộp thư in-app.
\item \textbf{markAsRead(userId: String, notificationId: String): void} \
Đánh dấu một thông báo là đã đọc và cập nhật lại hiển thị/badge.
\end{itemize}


\textbf{ReportAPI}
\begin{itemize}
\item \textbf{sendManual(file: File, requestedBy: Department): void} \
Gửi báo cáo thủ công: nhận tệp đã chọn và gửi tới phòng ban được chỉ định (thường qua email).
\item \textbf{sendAuto(file: File, requestedBy: Department, periods: String): void} \
Gửi báo cáo tự động theo kỳ (ví dụ tháng/quý) đến phòng ban mặc định, dùng tham số kỳ để điền bộ lọc.
\end{itemize}
%%%%%%%%%%%%%%%%%%%%%%%%%%%%%%%%%%%%%%%%%%

\subsubsection{Service Method}
\textbf{StudentSchedulingService}
\begin{itemize}

    \item bookAppointment(studentId: Long, tutorId: Long, date: LocalDateTime, startTime: LocalDateTime, endTime: LocalDateTime, topic: String): bool \\
    Tạo một lịch hẹn mới giữa sinh viên và tutor. Logic bao gồm: kiểm tra tutor có trống hay không trong khoảng thời gian yêu cầu, kiểm tra slot rảnh từ FreeSlotService, lưu Appointment vào Repository, và cắt slot rảnh tương ứng. Trả về true nếu đặt thành công, false nếu không còn slot hoặc xảy ra lỗi.

    \item viewOfficialMeetings(studentId: Long): List<Meeting> \\
    Lấy danh sách các Meeting chính thức của sinh viên, bao gồm Appointment đã duyệt (APPROVED) và các loại Meeting khác. Dữ liệu lấy từ MeetingRepository.

    \item viewAppointmentHistory(studentId: Long): List<Appointment> \\
    Trả về lịch sử tất cả các Appointment của sinh viên, bao gồm PENDING, APPROVED, REJECTED, CANCELLED. Phục vụ mục xem lịch sử đặt lịch.

    \item cancelMeeting(meetingId: Long, reason: String): bool \\
    Sinh viên yêu cầu hủy một Meeting. Kiểm tra Meeting tồn tại, chưa bị hủy, thuộc về sinh viên. Cập nhật trạng thái thành CANCELLED, lưu lý do hủy, và nếu Appointment đã được duyệt thì trả lại slot vào FreeSlotService. Trả về true nếu hủy thành công.

    \item viewTutorAvailableSlots(tutorId: Long): List<FreeSlotResponse> \\
    Lấy danh sách các slot rảnh của tutor (dạng date + list of time ranges). Phục vụ UI để hiển thị lịch rảnh.

    \item findCancellableMeetings(studentId: Long): List<Meeting> \\
    Trả về danh sách Meeting mà sinh viên có quyền hủy: là các Meeting APPROVED với startTime nằm trong tương lai.

    \item viewMeetingDetails(meetingId: Long): Meeting \\
    Lấy chi tiết đầy đủ của một Meeting (bao gồm thông tin tutor, thời gian, online link, trạng thái...). Trả về null nếu MeetingId không tồn tại.

\end{itemize}
\textbf{TutorSchedulingService}
\begin{itemize}

    \item viewPendingAppointments(tutorId: Long): List<Appointment> \\
    Lấy danh sách Appointment đang ở trạng thái PENDING của tutor. Dùng cho UI để hiển thị các yêu cầu chờ duyệt.

    \item approveAppointment(appointmentId: Long, tutorId: Long): bool \\
    Tutor duyệt Appointment. Logic: kiểm tra quyền sở hữu, kiểm tra Appointment chưa được duyệt hoặc từ chối, cập nhật trạng thái thành APPROVED, tạo onlineLink, và lưu lại vào Repository. Trả về true nếu duyệt thành công.

    \item rejectAppointment(appointmentId: Long, tutorId: Long, reason: String): bool \\
    Tutor từ chối Appointment. Kiểm tra quyền hợp lệ, cập nhật trạng thái thành REJECTED, lưu lý do từ chối. Trả về true nếu thao tác thành công.

    \item viewOfficialMeetings(tutorId: Long): List<Meeting> \\
    Trả về Meetings chính thức của tutor, bao gồm các Appointment đã duyệt và các Meeting khác có trong Repository.

    \item cancelMeeting(tutorId: Long, meetingId: Long, reason: String): bool \\
    Tutor hủy buổi Meeting. Kiểm tra Meeting thuộc tutor, chưa bị hủy, cập nhật trạng thái CANCELLED, lưu lý do. Trả về true nếu hủy thành công.

    \item tutorReturnCancelledSlot(tutorId: Long, meetingId: Long): bool \\
    Sau khi hủy Meeting, tutor có tùy chọn trả slot vào lịch rảnh. Kiểm tra Meeting đúng tutor, đúng trạng thái CANCELLED, rồi gọi FreeSlotService để trả lại slot. Trả về true nếu slot được trả, false nếu tutor chọn không trả.

    \item findCancellableMeetings(tutorId: Long): List<Meeting> \\
    Trả về các Meeting mà tutor có quyền hủy (APPROVED và xảy ra trong tương lai).

    \item viewMeetingDetails(meetingId: Long): Meeting \\
    Lấy thông tin chi tiết của một Meeting theo ID. Trả về null nếu không tồn tại.

    \item createOnlineLink(appointment: Appointment): String \\
    Tạo đường link học online cho Appointment (ví dụ: tạo link Google Meet). Được gọi khi tutor approve lịch. Trả về chuỗi URL.
    
    \item viewAppointmentDetails(appointmentId: Long): Appointment\\
    Lấy thông tin chi tiết của một cuộc hẹn (Appointment). Nếu ID không tồn tại hoặc Meeting không phải là Appointment thì trả về null.

\textbf{RoomService}
\begin{itemize}

    \item getAvailableRooms(date: Date, startTime: Time, endTime: Time): List<Room>\\
    Trả về danh sách phòng trống theo yêu cầu.

    \item isRoomAvailable(roomId: int, date: Date, startTime: Time, endTime: Time): boolean\\
    Kiểm tra phòng có trống không.

    \item reserveRoom(roomId: int, date: Date, startTime: Time, endTime: Time, meetingId: int): boolean\\
    Đặt phòng cho một cuộc họp. Trả về true nếu đặt thành công.

    \item releaseRoom(roomId: int, meetingId: int): boolean\\
    Hủy đặt phòng theo mã cuộc họp.

    \item getRoomDetails(roomId: int): Room\\
    Lấy thông tin chi tiết của phòng.

\end{itemize}


\textbf{CourseService}
\begin{itemize}
    \item addCourse(title: String, description: String, instructorID: int): bool\\
    Dùng để thêm khóa học mới, có kiểm tra trùng tiêu đề và trả về \texttt{true} nếu thêm thành công.
    \item getAllCourses(): List<CourseEntity>\\
    Trả về danh sách tất cả các \texttt{CourseEntity} trong cơ sở dữ liệu.
    \item getCourseById(courseID: int): CourseEntity\\
    Dùng để tìm khóa học theo ID và trả về \texttt{null} nếu không tồn tại.
\end{itemize}



\textbf{EnrollmentService} 
\begin{itemize}
    \item register(studentID: int, courseID: int): bool\\
    Cho phép sinh viên đăng ký khóa học nếu chưa đăng ký trước đó.
    \item getEnrollmentsByStudent(studentID: int): List<Course>\\
    Trả về danh sách các khóa học mà sinh viên đã đăng ký.
\end{itemize}



\textbf{RequestService} 
\begin{itemize}
    \item createRequest(studentID: int, tutorID: int, courseID: int, topic: String): int\\
    Tạo yêu cầu mới, gửi email đến tutor và trả về \texttt{requestID}.
    \item getRequestsByStudent(studentID: int): List<Request>\\
    Trả về danh sách yêu cầu của sinh viên.
    \item getRequestsByTutor(tutorID: int): List<Request>\\
    Trả về danh sách yêu cầu đang chờ tutor duyệt.
    \item updateStatus(requestID: int, status: String): bool\\
    Cập nhật trạng thái yêu cầu từ \texttt{PENDING} sang \texttt{APPROVED} hoặc \texttt{REJECTED}.
\end{itemize}



\textbf{FeedbackService} 
\begin{itemize}
    \item submitFeedback(requestID: int, rating: int, comment: String): bool\\
    Lưu lại đánh giá và trả về \texttt{true} nếu thành công.
\end{itemize}

\textbf{SSOService}
\begin{itemize}
    \item \textbf{authenticate(username: String, password: String): UserEntity} \\
    Kiểm tra username/password, trả về UserEntity nếu thành công.
    \item \textbf{sendResetEmail(bkId: String): void} \\
    Gửi email reset password.
    \item \textbf{resetPassword(token: String, newPassword: String): void} \\
    Reset password dựa trên token.
\end{itemize}

\textbf{DatacoreService}
\begin{itemize}
    \item \textbf{getUserInfo(username: String): UserEntity} \\
    Lấy thông tin user từ Datacore.
\end{itemize}

\textbf{MailService}
\begin{itemize}
    \item \textbf{sendEmail(to: String, content: String): void} \\
    Gửi email đơn giản.
    \item \textbf{sendPasswordResetEmail(to: String, resetLink: String): void} \\
    Gửi link reset password.
    \item \textbf{sendNotificationEmail(to: String, subject: String, content: String): void} \\
    Gửi email thông báo.
\end{itemize}

\textbf{ProfileService}
\begin{itemize}
    \item \textbf{getProfile(userId: String): UserEntity} \\
    Trả về thông tin user.
    \item \textbf{getActivityHistory(userId: String): List<ActivityEntity>} \\
    Trả về danh sách hoạt động của user.
    \item \textbf{getAppointmentHistory(userId: String): List<AppointmentEntity>} \\
    Trả về danh sách lịch hẹn của user.
\end{itemize}

\textbf{SessionManager}
\begin{itemize}
    \item \textbf{createSession(user: UserEntity): String} \\
    Tạo session mới và trả về sessionId.
    \item \textbf{getSession(sessionId: String): SessionEntity} \\
    Lấy thông tin session.
    \item \textbf{invalidateSession(sessionId: String): void} \\
    Vô hiệu hóa session.
    \item \textbf{isValid(sessionId: String): Boolean} \\
    Kiểm tra session còn hợp lệ.
\end{itemize}

\textbf{MaterialService}
\begin{itemize}
    \item \textbf{getMaterials(): List<MaterialEntity>} \\
    Lấy danh sách tài liệu.
    \item \textbf{getMaterialContent(materialId: String): String} \\
    Lấy nội dung chi tiết tài liệu.
    \item \textbf{searchMaterials(criteria: SearchCriteria): List<MaterialEntity>} \\
    Tìm kiếm tài liệu theo tiêu chí.
\end{itemize}
\textbf{AvailableSlotService}
\begin{itemize}
    \item \textbf{addAvailableSlot(tutorID: String, start: LocalDateTime, end: LocalDateTime): boolean} \\
    Thêm một slot trống mới cho gia sư. Kiểm tra logic nghiệp vụ (ví dụ: không trùng lặp). Trả về true nếu thành công.
    \item \textbf{updateAvailableSlot(tutorID: String, slotID: String, start: LocalDateTime, end: LocalDateTime): boolean} \\
    Cập nhật thời gian bắt đầu/kết thúc của một slot. Trả về true nếu thành công.
    \item \textbf{removeAvailableSlot(tutorID: String, slotID: String): boolean} \\
    Xóa một slot khỏi danh sách (ví dụ: gia sư hủy slot). Trả về true nếu thành công.
    \item \textbf{getAvailableSlotsByTutor(tutorID: String): List<AvailableSlot>} \\
    Lấy tất cả các slot (cả trống và đã đặt) của một gia sư.
    \item \textbf{getAvailableSlotsByTutorAndRange(tutorID: String, start: LocalDateTime, end: LocalDateTime): List<AvailableSlot>} \\
    Lấy các slot của gia sư trong một khoảng thời gian cụ thể.
    \item \textbf{findAvailable(tutorID: String, start: LocalDateTime, end: LocalDateTime): boolean} \\
    Kiểm tra xem gia sư có rảnh (available) trong khoảng thời gian được yêu cầu hay không.
    \item \textbf{markSlotAsBooked(tutorID: String, slotID: String): boolean} \\
    Đánh dấu một slot là đã được đặt (trạng thái: BOOKED). Trả về true nếu thành công.
    \item \textbf{markSlotAsAvailable(tutorID: String, slotID: String): boolean} \\
    Đánh dấu một slot là còn trống (trạng thái: AVAILABLE). Trả về true nếu thành công.
\end{itemize}
\textbf{MaterialUploadService}
\begin{itemize}
    \item \textbf{createDraft(tutorId: String, title: String, type: String, classId: String): Material} \\
    Logic nghiệp vụ tạo một bản nháp (draft) tài liệu trong cơ sở dữ liệu.
    \item \textbf{attachFile(materialId: String, fileName: String, bytes: byte[]): Material} \\
    Logic nghiệp vụ xử lý tệp (bytes) và liên kết nó với bản nháp tài liệu.
    \item \textbf{submitForReview(materialId: String): void} \\
    Logic nghiệp vụ cập nhật trạng thái tài liệu thành "Đang chờ xem xét" (Pending Review).
    \item \textbf{listMyUploads(tutorId: String): List<Material>} \\
    Truy vấn cơ sở dữ liệu (thông qua repository) để lấy danh sách tài liệu của gia sư.
\end{itemize}


\textbf{NotificationService}
\begin{itemize}
\item \textbf{sendToUser(o: Object, userId: int, type: String, title: String, content: String): void} \
Tạo một thông báo và gửi trực tiếp cho \textit{một} người dùng xác định bởi \texttt{userId}. Tham số \texttt{type/title/content} mô tả loại, tiêu đề và nội dung thông báo; \texttt{o} là dữ liệu ngữ cảnh đi kèm (nếu cần).
\item \textbf{sendToUsers(o: Object, userIds: List\textless int\textgreater, type: String, title: String, content: String): void} \
Gửi cùng một thông báo đến \textit{nhiều} người dùng trong danh sách \texttt{userIds}. Dùng cho các tình huống broadcast (ví dụ gửi cho cả lớp/khoa).
\end{itemize}


\textbf{ReportService}
\begin{itemize}
\item \textbf{sendManual(report: Report, requestedBy: Department): void} \
Gửi báo cáo theo yêu cầu thủ công: nhận đối tượng \texttt{report} (tệp/siêu dữ liệu báo cáo đã chuẩn bị) và gửi đến phòng ban được chỉ định trong \texttt{requestedBy}.
\item \textbf{sendAuto(report: Report, requestedBy: Department, periods: String): void} \
Gửi báo cáo tự động theo kỳ (ví dụ tháng/quý) đến phòng ban. Chuỗi \texttt{periods} mô tả khoảng thời gian cần áp dụng cho báo cáo.
\item \textbf{generateReport(data: Map\textless key,value\textgreater, report: Report): File} \
Sinh tệp báo cáo từ dữ liệu đầu vào \texttt{data} theo mẫu/định nghĩa trong \texttt{report}, trả về tệp để dùng cho việc gửi hoặc tải xuống.
\end{itemize}

%%%%%%%%%%%%%%%%%%%%%%%%




\subsubsection{Repository Method (Implementation)}
\textbf{RoomRepository}
\begin{itemize}
    \item listAvailableRooms(date: Date, startTime: Time, endTime: Time): List<Room>\\
    Trả về danh sách phòng trống trong khoảng thời gian nhất định.
    \item updateRoomStatus(roomId: int, status: String): bool\\
    Cập nhật trạng thái phòng (AVAILABLE, OCCUPIED, MAINTENANCE). Trả về true nếu thành công.
    \item getRoomInfo(roomId: int): String\\
    Lấy thông tin chi tiết của phòng theo roomId.
\end{itemize}

\textbf{RoomBookingRepository}
\begin{itemize}
    \item findBookings(date: Date): List<RoomBooking>\\
    Trả về tất cả các lịch đặt phòng theo ngày.

    \item save(booking: RoomBooking): void\\
    Lưu một lịch đặt phòng mới.

    \item findConflicts(roomId: int, date: Date, startTime: Time, endTime: Time): boolean\\
    Kiểm tra phòng có xung đột lịch hay không.
\end{itemize}

\textbf{MeetingRepository}
\begin{itemize}
    \item findById(meetingId: Long): Meeting\\
    Tìm và trả về Meeting theo \texttt{meetingId}. Trả về \texttt{null} nếu không tồn tại.
    
    \item save(meeting: Meeting): void\\
    Lưu một Meeting mới vào danh sách nội bộ.

    \item update(meeting: Meeting): void\\
    Cập nhật Meeting theo \texttt{meetingId}. Ném ngoại lệ nếu không tìm thấy.

    \item findPendingAppointmentsByTutor(tutorId: Long): List<Appointment>\\
    Trả về danh sách các Appointment ở trạng thái \texttt{PENDING} của tutor.

    \item findApprovedAppointmentsByTutor(tutorId: Long): List<Appointment>\\
    Trả về danh sách Appointment ở trạng thái \texttt{APPROVED} của tutor.

    \item findAllAppointmentsByStudent(studentId: Long): List<Appointment>\\
    Lấy toàn bộ Appointment của một student (mọi trạng thái).

    \item findApprovedAppointmentsByStudent(studentId: Long): List<Appointment>\\
    Trả về Appointment đã \texttt{APPROVED} của student.

    \item findApprovedMeetingByStudent(studentId: Long): List<Meeting>\\
    Trả về Meeting đã \texttt{APPROVED} của student.

    \item findApprovedMeetingByTutor(tutorId: Long): List<Meeting>\\
    Trả về Meeting đã \texttt{APPROVED} của tutor.

    \item findOfficialMeetingsByStudent(studentId: Long): List<Meeting>\\
    Trả về Meeting \texttt{APPROVED} và chưa bị hủy (\texttt{!isCancelled()}) của student.

    \item findOfficialMeetingsByTutor(tutorId: Long): List<Meeting>\\
    Trả về Meeting \texttt{APPROVED} và chưa bị hủy của tutor.
\end{itemize}


\textbf{CourseRepository}
\begin{itemize}
    \item save(course: Course): Course\\
    Dùng để lưu hoặc cập nhật khoá học.
    \item findById(courseID: int): Course\\
    Tìm khóa học theo ID.
    \item findAll(): List<Course>\\
    Trả về tất cả khóa học.
\end{itemize}


\textbf{EnrollmentRepository}
\begin{itemize}
    \item save(enrollment: Enrollment): Enrollment\\
    Lưu đăng ký.
    \item findByStudentID(studentID: int): List<Enrollment>\\
    Trả về danh sách đăng ký của học viên.
\end{itemize}

\textbf{RequestRepository}
\begin{itemize}
    \item save(request: Request): Request\\
    Lưu yêu cầu.
    \item findByStudentID(studentID: int): List<Request>\\
    Trả về danh sách yêu cầu của sinh viên.
    \item findByTutorID(tutorID: int): List<Request>\\
    Trả về danh sách yêu cầu cần tutor duyệt.
    \item updateStatus(requestID: int, status: String): bool\\
    Cập nhật trạng thái yêu cầu.
\end{itemize}


\textbf{FeedbackRepository}
\begin{itemize}
    \item save(feedback: Feedback): Feedback\\
    Lưu phản hồi sau buổi học.
    \item findByRequestID(requestID: int): Feedback\\
    Tìm phản hồi theo mã yêu cầu.
\end{itemize}

\textbf{UserRepository}
\begin{itemize}
    \item \textbf{getUserById(userId: String): UserEntity} \\
    Lấy user theo ID.
    \item \textbf{saveUser(user: UserEntity): void} \\
    Lưu user mới.
    \item \textbf{updateUser(user: UserEntity): void} \\
    Cập nhật user.
\end{itemize}

\textbf{ActivityRepository}
\begin{itemize}
    \item \textbf{getActivitiesByUserId(userId: String): List<ActivityEntity>} \\
    Lấy danh sách hoạt động theo userId.
    \item \textbf{saveActivity(activity: ActivityEntity): void} \\
    Lưu hoạt động mới.
\end{itemize}

\textbf{AppointmentRepository}
\begin{itemize}
    \item \textbf{getAppointmentsByUserId(userId: String): List<AppointmentEntity>} \\
    Lấy danh sách lịch hẹn theo userId.
    \item \textbf{saveAppointment(appointment: AppointmentEntity): void} \\
    Lưu lịch hẹn mới.
\end{itemize}

\textbf{MaterialRepository}
\begin{itemize}
    \item \textbf{getAllMaterials(): List<MaterialEntity>} \\
    Lấy toàn bộ tài liệu.
    \item \textbf{getMaterialById(materialId: String): MaterialEntity} \\
    Lấy chi tiết tài liệu theo ID.
    \item \textbf{searchMaterials(criteria: SearchCriteria): List<MaterialEntity>} \\
    Tìm kiếm tài liệu theo tiêu chí.
\end{itemize}

\textbf{SessionRepository}
\begin{itemize}
    \item \textbf{createSession(user: UserEntity): String} \\
    Tạo session mới.
    \item \textbf{getSession(sessionId: String): SessionEntity} \\
    Lấy session theo ID.
    \item \textbf{save(sessionId: String): void} \\
    Lưu session.
    \item \textbf{update(sessionId: String): Boolean} \\
    Cập nhật session, trả về true nếu thành công.
\end{itemize}

\textbf{AvailableSlotRepository}
\begin{itemize}
    \item \textbf{findByTutorId(id: String): List<AvailableSlot>} \\
    Tìm và trả về danh sách các \texttt{AvailableSlot} từ cơ sở dữ liệu dựa trên \texttt{tutorId}.
    \item \textbf{save(slot: AvailableSlot): AvailableSlot} \\
    Lưu (thêm mới hoặc cập nhật) một đối tượng \texttt{AvailableSlot} vào cơ sở dữ liệu.
\end{itemize}


\textbf{NotificationRepository}
\begin{itemize}
\item \textbf{save(n: Notification): Notification} \
Lưu một thông báo mới vào kho dữ liệu và trả về bản ghi đã lưu.
\item \textbf{saveAll(list: List\textless Notification\textgreater): void} \
Lưu hàng loạt thông báo (ví dụ gửi cho cả lớp/khoa) trong một lần thao tác.
\item \textbf{getNotifications(userId: int): List\textless Notification\textgreater} \
Lấy toàn bộ thông báo của một người dùng để hiển thị trong hộp thư.
\item \textbf{markAsRead(userId: int, notificationId: int): void} \
Đánh dấu một thông báo của người dùng là đã đọc.
\end{itemize}


\textbf{DataFetcherRepository}
\begin{itemize}
\item \textbf{fetchData(userId: int, key: List\textless String\textgreater): Map\textless key,value\textgreater} \
Truy vấn dữ liệu thô phục vụ sinh báo cáo, dựa trên người yêu cầu và các khóa tham số (\textit{key}) cụ thể.
\end{itemize}



\textbf{Exporter}
\begin{itemize}
\item \textbf{exportPDF(data: Map\textless key,value\textgreater, report: Report): File} \
Xuất dữ liệu đã render thành tệp \texttt{PDF} theo mẫu/định nghĩa của báo cáo, trả về tệp để đính kèm hoặc tải xuống.
\item \textbf{exportCSV(data: Map\textless key,value\textgreater, report: Report): File} \
Xuất dữ liệu thành tệp \texttt{CSV} (bảng dữ liệu), phù hợp cho phân tích hoặc nhập vào bảng tính.
\end{itemize}




\subsubsection{Repository Interface Method}

\textbf{IRoomRepository}
\begin{itemize}
    \item listAvailableRooms(date: Date, startTime: Time, endTime: Time): List<Room>\\
    Trả về danh sách phòng trống trong khoảng thời gian nhất định.
    \item updateRoomStatus(roomId: int, status: String): bool\\
    Cập nhật trạng thái phòng (AVAILABLE, OCCUPIED, MAINTENANCE). Trả về true nếu thành công.
    \item getRoomInfo(roomId: int): String\\
    Lấy thông tin chi tiết của phòng theo roomId.
\end{itemize}

\textbf{IRoomBookingRepository}
\begin{itemize}
    \item findBookings(date: Date): List<RoomBooking>\\
    Trả về tất cả các lịch đặt phòng theo ngày.

    \item save(booking: RoomBooking): void\\
    Lưu một lịch đặt phòng mới.

    \item findConflicts(roomId: int, date: Date, startTime: Time, endTime: Time): boolean\\
    Kiểm tra phòng có xung đột lịch hay không.
    \item updateStatus(roomId: int, status: String): void\\
\end{itemize}

\textbf{IMeetingRepository}
\begin{itemize}
    \item findById(meetingId: Long): Meeting\\
    Tìm và trả về Meeting theo \texttt{meetingId}. Trả về \texttt{null} nếu không tồn tại.
    
    \item save(meeting: Meeting): void\\
    Lưu một Meeting mới vào danh sách nội bộ.

    \item update(meeting: Meeting): void\\
    Cập nhật Meeting theo \texttt{meetingId}. Ném ngoại lệ nếu không tìm thấy.

    \item findPendingAppointmentsByTutor(tutorId: Long): List<Appointment>\\
    Trả về danh sách các Appointment ở trạng thái \texttt{PENDING} của tutor.

    \item findApprovedAppointmentsByTutor(tutorId: Long): List<Appointment>\\
    Trả về danh sách Appointment ở trạng thái \texttt{APPROVED} của tutor.

    \item findAllAppointmentsByStudent(studentId: Long): List<Appointment>\\
    Lấy toàn bộ Appointment của một student (mọi trạng thái).

    \item findApprovedAppointmentsByStudent(studentId: Long): List<Appointment>\\
    Trả về Appointment đã \texttt{APPROVED} của student.

    \item findApprovedMeetingByStudent(studentId: Long): List<Meeting>\\
    Trả về Meeting đã \texttt{APPROVED} của student.

    \item findApprovedMeetingByTutor(tutorId: Long): List<Meeting>\\
    Trả về Meeting đã \texttt{APPROVED} của tutor.

    \item findOfficialMeetingsByStudent(studentId: Long): List<Meeting>\\
    Trả về Meeting \texttt{APPROVED} và chưa bị hủy (\texttt{!isCancelled()}) của student.

    \item findOfficialMeetingsByTutor(tutorId: Long): List<Meeting>\\
    Trả về Meeting \texttt{APPROVED} và chưa bị hủy của tutor.
\end{itemize}


\textbf{ISSOServices}
\begin{itemize}
    \item \textbf{authenticate(username: String, password: String): UserEntity} \\
    Xác thực username/password, trả về UserEntity nếu thành công.
    \item \textbf{sendResetEmail(bkId: String): void} \\
    Gửi email reset password.
    \item \textbf{resetPassword(token: String, newPassword: String): void} \\
    Reset password dựa trên token.
\end{itemize}

\textbf{IMailService}
\begin{itemize}
    \item \textbf{sendEmail(to: String, content: String): void} \\
    Gửi email.
    \item \textbf{sendPasswordResetEmail(to: String, resetLink: String): void} \\
    Gửi link reset password.
    \item \textbf{sendNotificationEmail(to: String, subject: String, content: String): void} \\
    Gửi email thông báo.
\end{itemize}

\textbf{IDatacoreService}
\begin{itemize}
    \item \textbf{getUserInfo(username: String): UserEntity} \\
    Lấy thông tin user từ Datacore.
\end{itemize}

\textbf{IProfileService}
\begin{itemize}
    \item \textbf{getProfile(userId: String): UserEntity} \\
    Trả về thông tin user.
    \item \textbf{getActivityHistory(userId: String): List<ActivityEntity>} \\
    Trả về danh sách hoạt động của user.
    \item \textbf{getAppointmentHistory(userId: String): List<AppointmentEntity>} \\
    Trả về danh sách lịch hẹn của user.
\end{itemize}

\textbf{ISessionManager}
\begin{itemize}
    \item \textbf{createSession(user: UserEntity): String} \\
    Tạo session mới.
    \item \textbf{getSession(sessionId: String): SessionEntity} \\
    Lấy session theo ID.
    \item \textbf{invalidateSession(sessionId: String): void} \\
    Vô hiệu hóa session.
    \item \textbf{isValid(sessionId: String): Boolean} \\
    Kiểm tra session còn hợp lệ.
\end{itemize}

\textbf{IMaterialService}
\begin{itemize}
    \item \textbf{getMaterials(): List<MaterialEntity>} \\
    Lấy danh sách tài liệu.
    \item \textbf{getMaterialContent(materialId: String): String} \\
    Lấy nội dung chi tiết tài liệu.
    \item \textbf{searchMaterials(criteria: SearchCriteria): List<MaterialEntity>} \\
    Tìm kiếm tài liệu theo tiêu chí.
\end{itemize}
\textbf{IAvailableSlotRepository}
\begin{itemize}
    \item \textbf{findByTutorId(id: String): List<AvailableSlot>} \\
    Định nghĩa phương thức tìm kiếm các slot dựa trên \texttt{tutorId}.
    \item \textbf{save(slot: AvailableSlot): AvailableSlot} \\
    Định nghĩa phương thức lưu (thêm mới hoặc cập nhật) một \texttt{AvailableSlot}.
\end{itemize}
\textbf{IStudentSchedulingService}
\begin{itemize}

    \item bookAppointment(studentId: Long, tutorId: Long, date: LocalDateTime, startTime: LocalDateTime, endTime: LocalDateTime, topic: String): bool \\
    Tạo một lịch hẹn mới giữa sinh viên và tutor. Logic bao gồm: kiểm tra tutor có trống hay không trong khoảng thời gian yêu cầu, kiểm tra slot rảnh từ FreeSlotService, lưu Appointment vào Repository, và cắt slot rảnh tương ứng. Trả về true nếu đặt thành công, false nếu không còn slot hoặc xảy ra lỗi.

    \item viewOfficialMeetings(studentId: Long): List<Meeting> \\
    Lấy danh sách các Meeting chính thức của sinh viên, bao gồm Appointment đã duyệt (APPROVED) và các loại Meeting khác. Dữ liệu lấy từ MeetingRepository.

    \item viewAppointmentHistory(studentId: Long): List<Appointment> \\
    Trả về lịch sử tất cả các Appointment của sinh viên, bao gồm PENDING, APPROVED, REJECTED, CANCELLED. Phục vụ mục xem lịch sử đặt lịch.

    \item cancelMeeting(meetingId: Long, reason: String): bool \\
    Sinh viên yêu cầu hủy một Meeting. Kiểm tra Meeting tồn tại, chưa bị hủy, thuộc về sinh viên. Cập nhật trạng thái thành CANCELLED, lưu lý do hủy, và nếu Appointment đã được duyệt thì trả lại slot vào FreeSlotService. Trả về true nếu hủy thành công.

    \item viewTutorAvailableSlots(tutorId: Long): List<FreeSlotResponse> \\
    Lấy danh sách các slot rảnh của tutor (dạng date + list of time ranges). Phục vụ UI để hiển thị lịch rảnh.

    \item findCancellableMeetings(studentId: Long): List<Meeting> \\
    Trả về danh sách Meeting mà sinh viên có quyền hủy: là các Meeting APPROVED với startTime nằm trong tương lai.

    \item viewMeetingDetails(meetingId: Long): Meeting \\
    Lấy chi tiết đầy đủ của một Meeting (bao gồm thông tin tutor, thời gian, online link, trạng thái...). Trả về null nếu MeetingId không tồn tại.

\end{itemize}
\textbf{ITutorSchedulingService}
\begin{itemize}

    \item viewPendingAppointments(tutorId: Long): List<Appointment> \\
    Lấy danh sách Appointment đang ở trạng thái PENDING của tutor. Dùng cho UI để hiển thị các yêu cầu chờ duyệt.

    \item approveAppointment(appointmentId: Long, tutorId: Long): bool \\
    Tutor duyệt Appointment. Logic: kiểm tra quyền sở hữu, kiểm tra Appointment chưa được duyệt hoặc từ chối, cập nhật trạng thái thành APPROVED, tạo onlineLink, và lưu lại vào Repository. Trả về true nếu duyệt thành công.

    \item rejectAppointment(appointmentId: Long, tutorId: Long, reason: String): bool \\
    Tutor từ chối Appointment. Kiểm tra quyền hợp lệ, cập nhật trạng thái thành REJECTED, lưu lý do từ chối. Trả về true nếu thao tác thành công.

    \item viewOfficialMeetings(tutorId: Long): List<Meeting> \\
    Trả về Meetings chính thức của tutor, bao gồm các Appointment đã duyệt và các Meeting khác có trong Repository.

    \item cancelMeeting(tutorId: Long, meetingId: Long, reason: String): bool \\
    Tutor hủy buổi Meeting. Kiểm tra Meeting thuộc tutor, chưa bị hủy, cập nhật trạng thái CANCELLED, lưu lý do. Trả về true nếu hủy thành công.

    \item tutorReturnCancelledSlot(tutorId: Long, meetingId: Long): bool \\
    Sau khi hủy Meeting, tutor có tùy chọn trả slot vào lịch rảnh. Kiểm tra Meeting đúng tutor, đúng trạng thái CANCELLED, rồi gọi FreeSlotService để trả lại slot. Trả về true nếu slot được trả, false nếu tutor chọn không trả.

    \item findCancellableMeetings(tutorId: Long): List<Meeting> \\
    Trả về các Meeting mà tutor có quyền hủy (APPROVED và xảy ra trong tương lai).

    \item viewMeetingDetails(meetingId: Long): Meeting \\
    Lấy thông tin chi tiết của một Meeting theo ID. Trả về null nếu không tồn tại.

    \item createOnlineLink(appointment: Appointment): String \\
    Tạo đường link học online cho Appointment (ví dụ: tạo link Google Meet). Được gọi khi tutor approve lịch. Trả về chuỗi URL.
    
    \item viewAppointmentDetails(appointmentId: Long): Appointment\\
    Lấy thông tin chi tiết của một cuộc hẹn (Appointment). Nếu ID không tồn tại hoặc Meeting không phải là Appointment thì trả về null.


%%%%%%%%%%%%%%%%%%%%%%%%%%%%%%%%%%%%%%%%%%
\subsubsection{Entity Method}

\textbf{Meeting (abstract)}
\begin{itemize}
    \item cancel(userId, reason): bool\\
    Hủy buổi gặp mặt, lưu lý do.
    \item updateStatus(): void\\
    Cập nhật trạng thái theo thời gian (SCHEDULED → ONGOING → COMPLETED).
    \item Getter/Setter cho: meetingId, tutorId, date, startTime, endTime, topic, cancelled, cancellationReason.
\end{itemize}

\textbf{Appointment (extends Meeting)}
\begin{itemize}
    \item approve(tutorId): bool\\
    Duyệt lịch hẹn PENDING → APPROVED.
    \item reject(tutorId): bool\\
    Từ chối lịch hẹn.
    \item cancel(userId, reason): bool\\
    Override phương thức từ Meeting.
    \item Getter/Setter cho: studentId, status (PENDING/APPROVED/REJECTED).
\end{itemize}

\textbf{Consultation (extends Meeting)}
\begin{itemize}
    \item register(studentId): bool\\
    Thêm student vào participants.
    \item cancel(userId, reason): bool\\
    Override phương thức từ Meeting.
    \item Getter/Setter cho: title, mode, room, onlineLink, maxParticipants, participants, status (SCHEDULED/ONGOING/COMPLETED/CANCELLED).
\end{itemize}

\textbf{Room}
\begin{itemize}
    \item getId(): int\\
    Trả về mã phòng.

    \item getStatus(): String\\
    Trả về trạng thái phòng.

    \item setStatus(status: String): void\\
    Cập nhật trạng thái phòng.
\end{itemize}
\textbf{RoomBooking}
\begin{itemize}
    \item getRoomId(): int\\
    Trả về mã phòng.

    \item getDate(): Date\\
    Trả về ngày.

    \item getStartTime(): Time\\
    Trả về giờ bắt đầu.

    \item getEndTime(): Time\\
    Trả về giờ kết thúc.

    \item getMeetingId(): int\\
    Trả về mã cuộc họp.
\end{itemize}

\textbf{User}
\begin{itemize}
    \item \textbf{login(): bool} \\
    Thực hiện đăng nhập.
    \item \textbf{logout(): bool} \\
    Thực hiện đăng xuất.
    \item \textbf{viewProfile(): void} \\
    Xem thông tin cá nhân.
    \item \textbf{viewMaterials(): void} \\
    Xem danh sách tài liệu.
    \item \textbf{forgotPassword(): void} \\
    Yêu cầu đặt lại mật khẩu.
\end{itemize}

\textbf{Student}
\begin{itemize}
    \item \textbf{requestTutor(tutorId: int): bool} \\
    Yêu cầu tư vấn từ tutor.
    \item \textbf{bookAppointment(tutorId: int, datetime): bool} \\
    Đặt lịch hẹn với tutor.
    \item \textbf{giveFeedback(sessionId: int, rating: int, comment: String): bool} \\
    Gửi phản hồi về buổi tư vấn.
    \item \textbf{registerCounselingSession(tutorId: int, datetime): bool} \\
    Đăng ký buổi tư vấn với tutor.
\end{itemize}

\textbf{Tutor}
\begin{itemize}
    \item \textbf{acceptRequest(requestId: int): bool} \\
    Chấp nhận yêu cầu tư vấn.
    \item \textbf{createCounselingSession(studentId: int, datetime): bool} \\
    Tạo buổi tư vấn cho student.
    \item \textbf{cancelAppointment(appointmentId: int, reason: String): bool} \\
    Hủy một buổi hẹn.
    \item \textbf{recordStudentProgress(studentId: int, summary: String): bool} \\
    Ghi lại tiến độ học tập của student.
    \item \textbf{setAvailability(startTime: datetime, endTime: datetime): bool} \\
    Cập nhật khung thời gian có sẵn.
    \item \textbf{uploadMaterials(materialId: int, content: String): bool} \\
    Tải lên tài liệu học tập.
    \item \textbf{deleteMaterials(materialId: int): bool} \\
    Xóa tài liệu học tập.
    \item \textbf{processAppointmentRequest(requestId: int): bool} \\
    Xử lý yêu cầu đặt lịch hẹn.
\end{itemize}
\textbf{AvailableSlot}
\begin{itemize}
    \item \textbf{getStatus(): SlotStatus} \\
    Trả về trạng thái hiện tại của slot (ví dụ: AVAILABLE, BOOKED).
    \item \textbf{setStatus(status: SlotStatus): void} \\
    Cập nhật trạng thái của slot.
    \item \textbf{isOverlapping(other: AvailableSlot): boolean} \\
    Kiểm tra xem slot này có bị trùng lặp (chồng chéo) thời gian với một slot \texttt{other} hay không.
    \item \textbf{isAvailable(): boolean} \\
    Kiểm tra nhanh xem trạng thái của slot có phải là \texttt{AVAILABLE} hay không.
\end{itemize}

\textbf{Notification}
\begin{itemize}
\item \textbf{id: int} \
ID thông báo.
\item \textbf{userId: int} \
ID người nhận thông báo.
\item \textbf{type: String} \
Loại thông báo, dùng để phân nhóm xử lý.
\item \textbf{content: String} \
Nội dung tóm tắt hiển thị cho người dùng.
\item \textbf{createdAt: datetime} \
Thời điểm hệ thống tạo thông báo.
\item \textbf{readAt: datetime} \
Thời điểm người dùng đánh dấu đã đọc.
\end{itemize}

\bigskip

\textbf{Report (Entity)}
\begin{itemize}
\item \textbf{reportId: int} \
ID báo cáo.
\item \textbf{name: string} \
Tên của mẫu báo cáo.
\item \textbf{description: string} \
Mô tả ngắn về nội dung của báo cáo.
\item \textbf{format: ReportFormat} \
Định dạng xuất (\texttt{PDF}, \texttt{CSV}).
\item \textbf{title: string} \
Tiêu đề hiển thị.
\item \textbf{sendTo: Department} \
Phòng ban nhận báo cáo.
\item \textbf{createAt: datetime} \
Thời điểm tạo báo cáo.
\end{itemize}


\subsection{Test case}
\newpage
\subsubsection{Đăng nhập}
\begin{longtable}{|l|p{12cm}|}
\hline
\textbf{Test case} &
UC01-TC01 - Đăng nhập thành công qua HCMUT\_SSO \\
\hline
\textbf{Test description} &
User có tài khoản hợp lệ và có quyền sử dụng hệ thống, đăng nhập thành công thông qua HCMUT\_SSO, phiên làm việc được tạo và được chuyển đến trang chính tương ứng với vai trò. \\
\hline
\textbf{Related screens} &
- Trang chủ Tutor Support System với nút ``Đăng nhập qua HCMUT\_SSO''. \\
& - Trang đăng nhập HCMUT\_SSO (form BKNetID + mật khẩu). \\
& - Trang chính sau đăng nhập (dashboard tương ứng role: Sinh viên / Tutor / Điều phối viên). \\
\hline
\textbf{Pre-conditions} &
1. User có tài khoản hợp lệ trong hệ thống HCMUT\_SSO (BKNetID + mật khẩu đúng). \\
& 2. Tài khoản thuộc nhóm được phép dùng Tutor Support System (role hợp lệ). \\
& 3. HCMUT\_SSO và HCMUT\_DATACORE đang hoạt động bình thường. \\
& 4. User chưa đăng nhập vào Tutor Support System trên trình duyệt hiện tại. \\
\hline
\textbf{Actions} &
1. User truy cập trang chủ Tutor Support System. \\
& 2. User nhấn nút ``Đăng nhập qua HCMUT\_SSO''. \\
& 3. Hệ thống chuyển hướng sang trang đăng nhập HCMUT\_SSO. \\
& 4. User nhập đúng BKNetID và mật khẩu, sau đó nhấn nút đăng nhập. \\
& 5. HCMUT\_SSO xác thực thông tin đăng nhập thành công và trả về token xác thực cho Tutor Support System. \\
& 6. Tutor Support System dùng token truy vấn HCMUT\_DATACORE để lấy thông tin cơ bản (họ tên, email, role, trạng thái tài khoản, \ldots). \\
& 7. Hệ thống kiểm tra role và trạng thái tài khoản, tạo phiên làm việc cho user. \\
& 8. Hệ thống chuyển hướng user đến trang chính tương ứng với vai trò (dashboard Sinh viên / Tutor / Điều phối viên). \\
\hline
\textbf{Inputs} &
- BKNetID: tài khoản hợp lệ \\
& - Mật khẩu: mật khẩu tương ứng, hợp lệ. \\
\hline
\textbf{Expected Outputs} &
- User được xác thực thành công từ HCMUT\_SSO. \\
& - Dữ liệu người dùng được lấy thành công từ HCMUT\_DATACORE. \\
& - Phiên làm việc (session) của user được tạo trong Tutor Support System. \\
& - User được chuyển tới đúng trang chính theo vai trò, không có thông báo lỗi. \\
\hline
\textbf{Testing environment} &
Web \\
\hline
\end{longtable}


\newpage
% ================== UC01_TC02: SAI THÔNG TIN ĐĂNG NHẬP (E1) ==================
\begin{longtable}{|l|p{12cm}|}
\hline
\textbf{Test case} &
UC01-TC02 - Sai thông tin đăng nhập (mật khẩu/BKNetID không đúng) \\
\hline
\textbf{Test description} &
User nhập sai BKNetID hoặc mật khẩu tại HCMUT\_SSO, hệ thống hiển thị thông báo lỗi trên form và không đăng nhập vào Tutor Support System. \\
\hline
\textbf{Related screens} &
- Trang chủ Tutor Support System. \\
& - Trang đăng nhập HCMUT\_SSO (form đăng nhập). \\
\hline
\textbf{Pre-conditions} &
1. HCMUT\_SSO đang hoạt động bình thường. \\
& 2. User chưa đăng nhập vào Tutor Support System. \\
\hline
\textbf{Actions} &
1. User truy cập trang chủ Tutor Support System và nhấn ``Đăng nhập qua HCMUT\_SSO''. \\
& 2. Hệ thống chuyển sang trang đăng nhập HCMUT\_SSO. \\
& 3. User nhập BKNetID hoặc mật khẩu không chính xác, nhấn đăng nhập. \\
\hline
\textbf{Inputs} &
- BKNetID(đúng) hoặc một mã bất kỳ. \\
& - Mật khẩu không khớp mật khẩu thật \\
\hline
\textbf{Expected Outputs} &
- HCMUT\_SSO hiển thị thông báo lỗi trực tiếp trên form, ví dụ: ``Thông tin đăng nhập không chính xác.''. \\
& - User vẫn ở trên trang đăng nhập HCMUT\_SSO, không được chuyển hướng về Tutor Support System. \\
& - Không có phiên làm việc nào được tạo ở Tutor Support System . \\
\hline
\textbf{Testing environment} &
Web\\
\hline
\end{longtable}


\newpage
% ================== UC01_TC03: TÀI KHOẢN BỊ KHÓA / HẾT HẠN / YÊU CẦU ĐỔI MẬT KHẨU (E2) ==================
\begin{longtable}{|l|p{12cm}|}
\hline
\textbf{Test case} &
UC01-TC03 - Tài khoản bị khóa / hết hạn / yêu cầu đổi mật khẩu \\
\hline
\textbf{Test description} &
User nhập đúng BKNetID và mật khẩu nhưng tài khoản ở trạng thái bị khóa, hết hạn hoặc bắt buộc đổi mật khẩu; HCMUT\_SSO hiển thị thông báo tương ứng và user không đăng nhập được vào Tutor Support System. \\
\hline
\textbf{Related screens} &
Trang đăng nhập HCMUT\_SSO (form đăng nhập). \\
\hline
\textbf{Pre-conditions} &
1. Tài khoản BKNetID tồn tại nhưng đang ở trạng thái: bị khóa / hết hạn / yêu cầu đổi mật khẩu. \\
& 2. HCMUT\_SSO hoạt động bình thường. \\
& 3. User chưa có phiên đăng nhập Tutor Support System. \\
\hline
\textbf{Actions} &
1. User từ trang chủ Tutor Support System nhấn ``Đăng nhập qua HCMUT\_SSO'' để mở trang đăng nhập SSO. \\
& 2. User nhập đúng BKNetID và mật khẩu của tài khoản đang bị khóa/hết hạn/yêu cầu đổi mật khẩu. \\
& 3. User nhấn đăng nhập. \\
\hline
\textbf{Inputs} &
- BKNetID: tài khoản tồn tại nhưng bị khóa/hết hạn. \\
& - Mật khẩu: hợp lệ với tài khoản đó. \\
\hline
\textbf{Expected Outputs} &
- HCMUT\_SSO hiển thị thông báo lỗi tương ứng trên form đăng nhập (E2.1). \\
& - User không được chuyển hướng về Tutor Support System, không có token xác thực được trả về. \\
& - Use case kết thúc không thành công tại SSO, không tạo phiên làm việc ở Tutor Support System (E2.2). \\
\hline
\textbf{Testing environment} &
Web \\
\hline
\end{longtable}


\newpage
% ================== UC01_TC04: KHÔNG CÓ QUYỀN TRUY CẬP HỆ THỐNG (E3) ==================
\begin{longtable}{|l|p{12cm}|}
\hline
\textbf{Test case} &
UC01-TC04- Đăng nhập bằng tài khoản không có quyền sử dụng Tutor Support System \\
\hline
\textbf{Test description} &
User đăng nhập SSO thành công nhưng role không thuộc nhóm được phép truy cập Tutor Support System; hệ thống hiển thị thông báo ``Bạn không có quyền truy cập hệ thống.'' và không tạo phiên làm việc. \\
\hline
\textbf{Related screens} &
- Trang chủ Tutor Support System. \\
& - Trang đăng nhập HCMUT\_SSO. \\
& - Trang thông báo lỗi quyền truy cập của Tutor Support System. \\
\hline
\textbf{Pre-conditions} &
1. User có tài khoản hợp lệ trong HCMUT\_SSO. \\
& 2. Role của user \textbf{không} nằm trong tập role được phép dùng Tutor Support System. \\
& 3. HCMUT\_SSO và HCMUT\_DATACORE hoạt động bình thường. \\
\hline
\textbf{Actions} &
1. User từ trang chủ Tutor Support System nhấn ``Đăng nhập qua HCMUT\_SSO''. \\
& 2. User đăng nhập thành công trên HCMUT\_SSO (đúng BKNetID và mật khẩu). \\
& 3. Tutor Support System nhận token, truy vấn HCMUT\_DATACORE lấy thông tin user. \\
& 4. Hệ thống kiểm tra role, phát hiện user không thuộc nhóm được phép sử dụng. \\
\hline
\textbf{Inputs} &
- BKNetID + mật khẩu của user có role không hợp lệ (ví dụ thuộc hệ thống khác). \\
\hline
\textbf{Expected Outputs} &
- Tutor Support System hiển thị thông báo: ``Bạn không có quyền truy cập hệ thống.'' . \\
& - Phiên đăng nhập không được tạo; user không truy cập được vào bất kỳ trang chức năng nào. \\
& - Use case kết thúc không thành công, nhưng phiên SSO có thể vẫn tồn tại độc lập. \\
\hline
\textbf{Testing environment} &
Web \\
\hline
\end{longtable}


\newpage
% ================== UC01_TC05: LỖI KẾT NỐI ĐẾN HCMUT_SSO (E4) ==================
\begin{longtable}{|l|p{12cm}|}
\hline
\textbf{Test case} &
UC01-TC05 - Lỗi kết nối đến HCMUT\_SSO \\
\hline
\textbf{Test description} &
Tutor Support System không thể kết nối đến dịch vụ HCMUT\_SSO khi user yêu cầu đăng nhập; hệ thống báo lỗi và use case kết thúc không thành công. \\
\hline
\textbf{Related screens} &
Trang chủ / trang đăng nhập của Tutor Support System (nơi hiển thị nút ``Đăng nhập qua HCMUT\_SSO''). \\
\hline
\textbf{Pre-conditions} &
1. HCMUT\_SSO bị ngừng hoạt động hoặc kết nối mạng từ Tutor Support System đến SSO bị lỗi. \\
& 2. User chưa đăng nhập vào Tutor Support System. \\
\hline
\textbf{Actions} &
1. User truy cập trang chủ Tutor Support System. \\
& 2. User nhấn nút ``Đăng nhập qua HCMUT\_SSO''. \\
& 3. Hệ thống cố gắng chuyển hướng/kết nối tới HCMUT\_SSO nhưng bị lỗi kết nối. \\
\hline
\textbf{Inputs} &
Không có dữ liệu nhập từ user ngoài thao tác nhấn nút đăng nhập SSO. \\
\hline
\textbf{Expected Outputs} &
- Tutor Support System hiển thị thông báo: ``Không thể kết nối đến dịch vụ xác thực, vui lòng thử lại sau.'' . \\
& - Không có phiên làm việc được tạo, user vẫn ở trang login/landing của Tutor Support System. \\
\hline
\textbf{Testing environment} &
Web. \\
\hline
\end{longtable}


\newpage
% ================== UC01_TC06: LỖI ĐỒNG BỘ DỮ LIỆU HCMUT_DATACORE (E5) ==================
\begin{longtable}{|l|p{12cm}|}
\hline
\textbf{Test case} &
UC01-TC06 - Lỗi đồng bộ dữ liệu từ HCMUT\_DATACORE \\
\hline
\textbf{Test description} &
User đăng nhập SSO thành công, nhưng Tutor Support System không lấy được thông tin người dùng từ HCMUT\_DATACORE; hệ thống hiển thị thông báo lỗi, hủy phiên đăng nhập và không cho truy cập hệ thống. \\
\hline
\textbf{Related screens} &
- Trang đăng nhập HCMUT\_SSO. \\
& - Trang thông báo lỗi của Tutor Support System sau khi nhận token. \\
\hline
\textbf{Pre-conditions} &
1. User có tài khoản hợp lệ và đăng nhập được vào HCMUT\_SSO. \\
& 2. Dịch vụ HCMUT\_DATACORE gặp lỗi (không truy cập được, trả về lỗi, hoặc dữ liệu không đầy đủ). \\
\hline
\textbf{Actions} &
1. User từ trang chủ Tutor Support System chọn ``Đăng nhập qua HCMUT\_SSO'' và đăng nhập thành công trên SSO. \\
& 2. Tutor Support System nhận token từ HCMUT\_SSO. \\
& 3. Hệ thống dùng token truy vấn HCMUT\_DATACORE để lấy thông tin người dùng. \\
& 4. Việc truy vấn thất bại hoặc trả về dữ liệu lỗi/thiếu. \\
\hline
\textbf{Inputs} &
- BKNetID + mật khẩu hợp lệ (đăng nhập SSO thành công). \\
\hline
\textbf{Expected Outputs} &
- Tutor Support System hiển thị thông báo: ``Không thể tải thông tin người dùng, vui lòng thử lại sau.'' . \\
& - Hệ thống hủy phiên đăng nhập hiện tại, không cho user truy cập bất kỳ chức năng nào . \\
& - Use case kết thúc không thành công (E5.3). \\
\hline
\textbf{Testing environment} &
Web. \\
\hline
\end{longtable}


\newpage
\subsubsection{Đăng xuất}
% ================== UC02_TC01: ĐĂNG XUẤT THÀNH CÔNG ==================
\begin{longtable}{|l|p{12cm}|}
\hline
\textbf{Test case} &
UC02-TC01 - Đăng xuất thành công \\
\hline
\textbf{Test description} &
User đang đăng nhập và chủ động chọn chức năng Đăng xuất; hệ thống kết thúc phiên làm việc, vô hiệu hóa token và chuyển về trang đăng nhập với thông báo thành công. \\
\hline
\textbf{Related screens} &
- Trang bất kỳ trong Tutor Support System có hiển thị menu người dùng / biểu tượng tài khoản. \\
& - Hộp thoại xác nhận ``Bạn có chắc chắn muốn đăng xuất?''. \\
& - Trang đăng nhập Tutor Support System. \\
\hline
\textbf{Pre-conditions} &
1. User (Sinh viên / Tutor / Điều phối viên) đã đăng nhập thành công vào hệ thống. \\
& 2. Phiên làm việc của user đang hoạt động, token xác thực còn hiệu lực. \\
\hline
\textbf{Actions} &
1. User nhấn vào biểu tượng tài khoản hoặc menu người dùng ở góc trên. \\
& 2. Hệ thống hiển thị menu với tùy chọn ``Đăng xuất''. \\
& 3. User chọn mục ``Đăng xuất''. \\
& 4. Hệ thống hiển thị hộp thoại xác nhận ``Bạn có chắc chắn muốn đăng xuất?'' với hai nút: ``Đăng xuất'' và ``Hủy''. \\
& 5. User chọn nút xác nhận Đăng xuất. \\
& 6. Hệ thống vô hiệu hóa token xác thực hiện tại. \\
& 7. Hệ thống xóa thông tin phiên làm việc (session, cookie liên quan). \\
& 8. Hệ thống chuyển hướng user về trang đăng nhập. \\
& 9. Hệ thống hiển thị thông báo ``Đăng xuất thành công!'' trên trang đăng nhập. \\
\hline
\textbf{Inputs} &
Không có dữ liệu nhập dạng text; chỉ có thao tác bấm menu và xác nhận Đăng xuất. \\
\hline
\textbf{Expected Outputs} &
- Phiên làm việc của user bị xóa; token xác thực bị vô hiệu hóa. \\
& - User được chuyển về trang đăng nhập, không thể truy cập lại các trang nội bộ nếu không đăng nhập lại. \\
& - Thông báo ``Đăng xuất thành công!'' được hiển thị rõ ràng. \\
\hline
\textbf{Testing environment} &
Web \\
\hline
\end{longtable}


\newpage
% ================== UC02_TC02: HỦY ĐĂNG XUẤT (ALTERNATIVE FLOW 5a) ==================
\begin{longtable}{|l|p{12cm}|}
\hline
\textbf{Test case} &
UC02-TC02 - Hủy thao tác đăng xuất \\
\hline
\textbf{Test description} &
User chọn Đăng xuất nhưng bấm ``Hủy'' trong hộp thoại xác nhận; hệ thống đóng hộp thoại và user tiếp tục sử dụng hệ thống như bình thường. \\
\hline
\textbf{Related screens} &
Trang bất kỳ trong hệ thống + hộp thoại xác nhận Đăng xuất. \\
\hline
\textbf{Pre-conditions} &
1. User đã đăng nhập và phiên làm việc đang hoạt động. \\
\hline
\textbf{Actions} &
1. User mở menu người dùng và chọn ``Đăng xuất''. \\
& 2. Hệ thống hiển thị hộp thoại xác nhận ``Bạn có chắc chắn muốn đăng xuất?''. \\
& 3. User chọn nút ``Hủy'' trên hộp thoại xác nhận. \\
\hline
\textbf{Inputs} &
Thao tác bấm vào nút ``Hủy'' trên hộp thoại xác nhận Đăng xuất. \\
\hline
\textbf{Expected Outputs} &
- Hệ thống đóng hộp thoại xác nhận (không còn hiển thị trên màn hình). \\
& - Phiên làm việc và token xác thực của user vẫn giữ nguyên, không bị xóa. \\
& - User vẫn ở lại trang hiện tại và tiếp tục sử dụng hệ thống bình thường. \\
\hline
\textbf{Testing environment} &
Chrome / Windows 10; hệ thống backend hoạt động bình thường. \\
\hline
\end{longtable}


\newpage
% ================== UC02_TC03: LỖI VÔ HIỆU HÓA TOKEN (EXCEPTION E1) ==================
\begin{longtable}{|l|p{12cm}|}
\hline
\textbf{Test case} &
UC02-TC03 - Lỗi khi vô hiệu hóa token đăng nhập \\
\hline
\textbf{Test description} &
Trong quá trình đăng xuất, hệ thống gặp lỗi khi gọi backend để vô hiệu hóa token; hệ thống ghi log lỗi, xóa thông tin phiên làm việc phía client và chuyển user về trang đăng nhập. \\
\hline
\textbf{Related screens} &
- Trang bất kỳ trong hệ thống. \\
& - Hộp thoại xác nhận Đăng xuất. \\
& - Trang đăng nhập. \\
\hline
\textbf{Pre-conditions} &
1. User đã đăng nhập và phiên làm việc đang hoạt động. \\
& 2. Môi trường kiểm thử được cấu hình mô phỏng lỗi ở bước vô hiệu hóa token trên server (backend trả lỗi). \\
\hline
\textbf{Actions} &
1. User mở menu người dùng và chọn ``Đăng xuất''. \\
& 2. Hệ thống hiển thị hộp thoại xác nhận, user chọn xác nhận Đăng xuất. \\
& 3. Hệ thống gửi yêu cầu vô hiệu hóa token đến backend và nhận lỗi. \\
\hline
\textbf{Inputs} &
Thao tác xác nhận Đăng xuất; không có input text. \\
\hline
\textbf{Expected Outputs} &
- Backend ghi lại log lỗi vô hiệu hóa token (E1.1). \\
& - Ở phía client, hệ thống vẫn xóa thông tin phiên làm việc hiện tại (session, cookie) . \\
& - User được chuyển hướng về trang đăng nhập; từ đó không thể truy cập lại trang nội bộ nếu chưa đăng nhập lại . \\
& - Có thể (tuỳ thiết kế) hiển thị thông báo chung ``Đã đăng xuất, nhưng có lỗi nội bộ, vui lòng đăng nhập lại sau.''. \\
\hline
\textbf{Testing environment} &
Web \\
\hline
\end{longtable}


\newpage
% ================== UC02_TC04: PHIÊN ĐÃ HẾT HẠN TRƯỚC KHI ĐĂNG XUẤT (EXCEPTION E2) ==================
\begin{longtable}{|l|p{12cm}|}
\hline
\textbf{Test case} &
UC02-TC04 - Phiên làm việc đã hết hạn trước khi người dùng chọn Đăng xuất \\
\hline
\textbf{Test description} &
Phiên làm việc của user đã hết hạn do timeout; khi user thao tác Đăng xuất, hệ thống phát hiện session hết hạn và chuyển trực tiếp về trang đăng nhập với thông báo tương ứng. \\
\hline
\textbf{Related screens} &
Trang bất kỳ trong hệ thống (có thể đã cũ) và trang đăng nhập. \\
\hline
\textbf{Pre-conditions} &
1. User đã từng đăng nhập vào hệ thống. \\
& 2. Thời gian không hoạt động vượt quá thời gian timeout phiên; session trên server đã bị hủy. \\
& 3. Trình duyệt của user vẫn còn trang hệ thống mở (UI chưa refresh). \\
\hline
\textbf{Actions} &
1. User trên trang cũ nhấn vào menu người dùng và chọn ``Đăng xuất''. \\
& 2. Hệ thống gửi yêu cầu đăng xuất, nhưng phát hiện phiên làm việc đã hết hạn. \\
\hline
\textbf{Inputs} &
Thao tác chọn ``Đăng xuất'' trên UI sau khi phiên đã hết hạn. \\
\hline
\textbf{Expected Outputs} &
- Hệ thống không xử lý như đăng xuất thông thường mà nhận diện phiên đã hết hạn. \\
& - User được chuyển hướng ngay về trang đăng nhập. \\
& - Trên trang đăng nhập hiển thị thông báo, ví dụ: ``Phiên làm việc đã hết hạn.''. \\
& - User không còn truy cập được các trang nội bộ nếu không đăng nhập lại. \\
\hline
\textbf{Testing environment} &
Web \\
\hline
\end{longtable}


\newpage
\subsubsection{Đăng ký Tutor}

\begin{longtable}{|l|p{12cm}|}
\hline
\textbf{Test case} & UC07-TC01 — Đăng ký Tutor thành công \\ \hline
\textbf{Test description} & Sinh viên đăng ký Tutor với đầy đủ thông tin hợp lệ \\ \hline
\textbf{Related screens} & Màn hình Đăng ký Tutor \\ \hline
\textbf{Pre-conditions} & 1. Sinh viên đã đăng nhập \\ 
& 2. Chương trình Tutor đang mở đăng ký \\ \hline
\textbf{Actions} & 1. Sinh viên mở chức năng “Đăng ký Tutor” \\
& 2. Hệ thống hiển thị giao diện chọn môn học/lĩnh vực \\
& 3. Sinh viên chọn môn học Nguyên lý ngôn ngữ lập trình và bấm “Tiếp theo” \\
& 4. Hệ thống hiển thị danh sách tutor phù hợp \\
& 5. Sinh viên chọn đăng ký tutor Nguyễn Thị B và bấm xác nhận \\
& 6. Hệ thống hiển thị màn hình Chờ duyệt 12h với nút “Hủy đăng ký” \\ \hline
\textbf{Inputs} & Môn học: Nguyên lý ngôn ngữ lập trình; Tutor: Nguyễn Thị B \\ \hline
\textbf{Expected Outputs} & 1. Hệ thống hiển thị danh sách tutor phù hợp \\
& 2. Hệ thống hiển thị trạng thái Chờ duyệt 12h \\
& 3. Yêu cầu đăng ký được lưu trong hệ thống \\
& 4. Nút Hủy đăng ký khả dụng \\ \hline
\textbf{Testing environment} & Web – Windows 11 \\ \hline
\end{longtable}

\newpage
\begin{longtable}{|l|p{12cm}|}
\hline
\textbf{Test case} & UC07-TC02 — Bỏ trống thông tin môn học \\ \hline
\textbf{Test description} & Sinh viên không chọn môn học ở bước 1 \\ \hline
\textbf{Related screens} & Màn hình nhập thông tin \\ \hline
\textbf{Pre-conditions} & 1. Sinh viên đã đăng nhập \\ 
& 2. Chương trình Tutor đang mở đăng ký \\ \hline
\textbf{Actions} & 1. Sinh viên mở chức năng “Đăng ký Tutor” \\
& 2. Hệ thống hiển thị giao diện chọn môn học \\
& 3. Sinh viên bỏ trống môn học \\
& 4. Sinh viên Bấm “Tiếp theo” \\ \hline
\textbf{Inputs} & Môn học: (trống) \\ \hline
\textbf{Expected Outputs} & 1. Hệ thống hiển thị thông báo lỗi “Vui lòng nhập đầy đủ thông tin” \\
& 2. Không chuyển sang màn hình hiển thị danh sách tutor. \\ \hline
\textbf{Testing environment} & Web – Windows 11 \\ \hline
\end{longtable}
\newpage
\begin{longtable}{|l|p{12cm}|}
\hline
\textbf{Test case} & UC07-TC03 — Hủy đăng ký trong khi chờ duyệt và đăng ký mới \\ \hline
\textbf{Test description} & Sinh viên hủy yêu cầu đăng ký tutor đang chờ duyệt và ngay lập tức thực hiện đăng ký mới \\ \hline
\textbf{Related screens} & Màn hình Chờ duyệt 12h \\ \hline
\textbf{Pre-conditions} & 1. Sinh viên đã đăng nhập \\
& 2. Yêu cầu đăng ký tutor đang ở trạng thái Chờ duyệt 12h \\ \hline
\textbf{Actions} & 1. Sinh viên mở chức năng “Đăng ký Tutor” \\
& 2. Hệ thống hiển thị màn hình Chờ duyệt 12h \\
& 3. Nhấn nút Hủy đăng ký \\
& 4. Xác nhận hủy \\
& 5. Hệ thống hiển thị thông báo “Hủy thành công” \\
& 6. Sinh viên chọn Đăng kí mới \\
& 7. Hệ thống quay về hiển thị danh sách tutor phù hợp \\
& 8. Sinh viên chọn tutor mới và bấm “Xác nhận” \\ \hline
\textbf{Inputs} & Tutor: Nguyễn Thị B \\ \hline
\textbf{Expected Outputs} & 1. Hệ thống hiển thị “Hủy thành công” \\& 2. Cập nhật trạng thái slot tutor cũ \\
& 3. Nếu chọn đăng ký mới → hệ thống hiển thị danh sách tutor \\
& 4. Sinh viên có thể chọn tutor mới và hoàn tất đăng ký \\ \hline
\textbf{Testing environment} & Web – Windows 11 \\ \hline
\end{longtable}

\newpage
\subsubsection{Đặt lịch hẹn}

\begin{longtable}{|l|p{12cm}|}
\hline

\textbf{Test case} & UC08-TC01 — Đặt lịch hẹn thành công \\ \hline
\textbf{Test description} & Sinh viên đặt lịch hẹn với tutor dựa trên lịch rảnh và nội dung hỗ trợ hợp lệ \\ \hline
\textbf{Related screens} & Màn hình Đặt lịch hẹn \\ \hline
\textbf{Pre-conditions} & 1. Sinh viên đã đăng nhập \newline
2. Sinh viên đã chọn tutor \newline
3. Tutor đã cung cấp lịch rảnh \\ \hline
\textbf{Actions} & 1. Sinh viên mở chức năng “Đặt lịch hẹn” \newline
2. Hệ thống hiển thị lịch rảnh của tutor \newline
3. Sinh viên chọn khoảng thời gian phù hợp và nhập nội dung hỗ trợ \newline
4. Nhấn “Xác nhận” \\ \hline
\textbf{Inputs} & Tutor: Nguyễn Thị B, Ngày giờ: 15/11/2025 14:00-15:00, Nội dung: Hướng dẫn bài tập Nguyên lý ngôn ngữ lập trình \\ \hline
\textbf{Expected Outputs} & 1. Hệ thống lưu yêu cầu lịch hẹn ở trạng thái Chờ duyệt \newline
2. Tutor nhận thông báo yêu cầu mới \\ \hline
\textbf{Testing environment} & Web – Windows 11 \\ \hline
\end{longtable}

\newpage
\begin{longtable}{|l|p{12cm}|}
\hline
\textbf{Test case} & UC08-TC02 — Đặt lịch hẹn thiếu thông tin \\ \hline
\textbf{Test description} & Sinh viên không nhập nội dung \\ \hline
\textbf{Related screens} & Màn hình Đặt lịch hẹn \\ \hline
\textbf{Pre-conditions} & 1. Sinh viên đã đăng nhập \newline
2. Sinh viên đã chọn tutor \newline
3. Tutor đã cung cấp lịch rảnh \\ \hline
\textbf{Actions} & 1. Sinh viên mở chức năng “Đặt lịch hẹn” \newline
2. Không chọn thời gian hoặc không nhập nội dung \newline
3. Nhấn “Xác nhận” \\ \hline
\textbf{Inputs} & Ngày giờ: 15/11/2025 14:00-15:00; Nội dung: (trống) \\ \hline
\textbf{Expected Outputs} & 1. Hệ thống hiển thị thông báo lỗi và yêu cầu nhập lại \newline
2. Quay lại màn hình nhập thông tin \\ \hline
\textbf{Testing environment} & Web – Windows 11 \\ \hline
\end{longtable}
\newpage
\subsubsection{Phản hồi chất lượng buổi học}
% ================== TEST CASE 1 ==================
\begin{longtable}{|l|p{12cm}|}
\hline
\textbf{Test case} &
UC09-TC01 - Gửi phản hồi chất lượng buổi học thành công \\
\hline
\textbf{Test description} &
Sinh viên gửi phản hồi hợp lệ cho một buổi học đã diễn ra, hệ thống lưu phản hồi, cập nhật lịch sử \\
\hline
\textbf{Related screens} &
- Tab ``Phản hồi chất lượng''. \\
& - Màn hình ``Gửi phản hồi''. \\
\hline
\textbf{Pre-conditions} &
1. Sinh viên đã có ít nhất một buổi hẹn đã tham gia. \\
& 2. Tab ``Phản hồi chất lượng'' đang hiển thị và có buổi hẹn mà sinh viên đã tham gia \\
\hline
\textbf{Actions} &
1. Sinh viên nhấn nút ``Phản hồi chất lượng'' của buổi hẹn. \\
& 2. Hệ thống mở màn hình ``Gửi phản hồi'' cho buổi hẹn đó. \\
& 3. Sinh viên đánh giá số sao \\
& 4. Sinh viên nhập nội dung vào ô ``Nhận xét'', ví dụ: ``Giải bài mẫu chi tiết, dễ hiểu.''. \\
& 5. Sinh viên nhấn nút ``Gửi''. \\
& 6. Hệ thống kiểm tra dữ liệu hợp lệ (đã chọn sao và nhận xét không rỗng). \\
& 7. Hệ thống lưu phản hồi vào cơ sở dữ liệu. \\
& 8. Hệ thống cập nhật khung ``Lịch sử phản hồi'' với dòng phản hồi mới. \\
& 9. Hệ thống hiển thị thông báo gửi phản hồi thành công. \\
\hline
\textbf{Inputs} &
- Mức đánh giá: 5/5 sao. \\
& - Nhận xét: ``Giải bài mẫu chi tiết, dễ hiểu.''. \\
\hline
\textbf{Expected Outputs} &
- Phản hồi mới được lưu đúng sinh viên, đúng buổi hẹn \\
& - Khung ``Lịch sử phản hồi'' hiển thị nội dung đã đánh giá, ví dụ: ngày 2025-10-26, điểm 5/5, nội dung ``Giải bài mẫu chi tiết, dễ hiểu.''. \\
& - Sinh viên thấy thông báo gửi thành công, không có thông báo lỗi. \\
\hline
\textbf{Testing environment} &
Web \\
\hline
\end{longtable}
\newpage
% ================== TEST CASE 2 ==================
\begin{longtable}{|l|p{12cm}|}
\hline
\textbf{Test case} &
UC09-TC02 - Gửi phản hồi thiếu nội dung nhận xét \\
\hline
\textbf{Test description} &
Sinh viên chỉ chọn số sao, bỏ trống ô nhận xét rồi nhấn ``Gửi''; hệ thống phát hiện thiếu thông tin, hiển thị thông báo lỗi và không lưu phản hồi. \\
\hline
\textbf{Related screens} &
- Tab ``Phản hồi chất lượng''. \\
& - Màn hình ``Gửi phản hồi''. \\
\hline
\textbf{Pre-conditions} &
1. Sinh viên đã có ít nhất một buổi hẹn đã tham gia. \\
& 2. Tab ``Phản hồi chất lượng'' đang hiển thị và có buổi hẹn mà sinh viên đã tham gia \\
\hline
\textbf{Actions} &
1. Sinh viên nhấn ``Phản hồi chất lượng'' cho buổi hẹn. \\
& 2. Màn hình ``Gửi phản hồi'' hiển thị. \\
& 3. Sinh viên chọn số sao. \\
& 4. Sinh viên để trống ô ``Nhận xét''. \\
& 5. Sinh viên nhấn nút ``Gửi''. \\
\hline
\textbf{Inputs} &
- Mức đánh giá: 4/5 sao. \\
& - Nhận xét: để trống. \\
\hline
\textbf{Expected Outputs} &
- Hệ thống không lưu bất kỳ phản hồi nào vào cơ sở dữ liệu. \\
& - Hệ thống hiển thị thông báo lỗi, ví dụ: ``Vui lòng nhập nội dung phản hồi trước khi gửi.''. \\
& - Màn hình ``Gửi phản hồi'' vẫn giữ nguyên: 4 sao vẫn được chọn, ô ``Nhận xét'' vẫn trống để sinh viên nhập lại. \\
& - Khung ``Lịch sử phản hồi'' không thay đổi. \\
\hline
\textbf{Testing environment} &
Web \\
\hline
\end{longtable}

\newpage
% ================== TEST CASE 3 ==================
\begin{longtable}{|l|p{12cm}|}
\hline
\textbf{Test case} &
UC09-TC03 - Hủy gửi phản hồi chất lượng \\
\hline
\textbf{Test description} &
Sinh viên mở màn hình ``Gửi phản hồi'' nhưng nhấn ``Hủy'' thay vì ``Gửi''; hệ thống không lưu dữ liệu và quay về danh sách buổi hẹn. \\
\hline
\textbf{Related screens} &
- Tab ``Phản hồi chất lượng''. \\
& - Màn hình ``Gửi phản hồi''. \\
\hline
\textbf{Pre-conditions} &
1. Sinh viên đã có ít nhất một buổi hẹn đã tham gia. \\
& 2. Tab ``Phản hồi chất lượng'' đang hiển thị và có buổi hẹn mà sinh viên đã tham gia \\
\hline
\textbf{Actions} &
1. Sinh viên nhấn ``Phản hồi chất lượng'' cho buổi hẹn. \\
& 2. Màn hình ``Gửi phản hồi'' hiển thị. \\
& 3. Sinh viên có thể chọn số sao và/hoặc nhập vài từ vào ô ``Nhận xét'' (tuỳ ý). \\
& 4. Sinh viên nhấn nút ``Hủy'' thay vì ``Gửi''. \\
\hline
\textbf{Inputs} &
Có thể đã nhập một số dữ liệu tạm (ví dụ chọn 3/5 sao, nhận xét ngắn), nhưng chưa thực hiện gửi. \\
\hline
\textbf{Expected Outputs} &
- Hệ thống không lưu bất kỳ phản hồi nào vào cơ sở dữ liệu. \\
& - Hệ thống đóng màn hình ``Gửi phản hồi'' và quay lại tab ``Phản hồi chất lượng'' (danh sách buổi hẹn). \\
& - Khi mở lại màn hình ``Gửi phản hồi'' cho cùng buổi hẹn, khung ``Lịch sử phản hồi'' không có phản hồi mới. \\
\hline
\textbf{Testing environment} &
Web \\
\hline
\end{longtable}

\newpage
% ================== TEST CASE 4 ==================
\begin{longtable}{|l|p{12cm}|}
\hline
\textbf{Test case} &
UC09-TC04 - Lỗi hệ thống khi lưu phản hồi \\
\hline
\textbf{Test description} &
Trong lúc sinh viên gửi phản hồi hợp lệ, hệ thống gặp lỗi cơ sở dữ liệu hoặc lỗi backend; hệ thống báo thất bại, ghi log lỗi và không làm mất dữ liệu đã nhập trên form. \\
\hline
\textbf{Related screens} &
- Tab ``Phản hồi chất lượng''. \\
& - Màn hình ``Gửi phản hồi''. \\
\hline
\textbf{Pre-conditions} &
1. Tài khoản sinh viên tồn tại và đang hoạt động, đã đăng nhập. \\
& 2. Có buổi hẹn đã tham gia và hiển thị trong tab ``Phản hồi chất lượng''. \\
& 3. Môi trường kiểm thử được cấu hình mô phỏng lỗi khi lưu vào cơ sở dữ liệu. \\
\hline
\textbf{Actions} &
1. Sinh viên nhấn ``Phản hồi chất lượng'' cho buổi hẹn ngày 26/10/2025. \\
& 2. Màn hình ``Gửi phản hồi'' hiển thị. \\
& 3. Sinh viên chọn 5 sao. \\
& 4. Sinh viên nhập nhận xét, ví dụ: ``Buổi học rất hữu ích.''. \\
& 5. Sinh viên nhấn nút ``Gửi''. \\
& 6. Backend cố gắng lưu dữ liệu nhưng gặp lỗi cơ sở dữ liệu. \\
\hline
\textbf{Inputs} &
- Mức đánh giá: 5/5 sao. \\
& - Nhận xét: ``Buổi học rất hữu ích.''. \\
\hline
\textbf{Expected Outputs} &
- Hệ thống không tạo bản ghi phản hồi mới trong cơ sở dữ liệu. \\
& - Hệ thống hiển thị thông báo lỗi, ví dụ: ``Gửi phản hồi thất bại, vui lòng thử lại sau hoặc liên hệ bộ phận kỹ thuật.''. \\
& - Dữ liệu trên form không bị mất: 5 sao vẫn được chọn, nội dung nhận xét vẫn còn để sinh viên có thể thử gửi lại. \\
& - Lỗi được ghi log ở phía server để phục vụ việc theo dõi và khắc phục. \\
\hline
\textbf{Testing environment} &
Web \\
\hline
\end{longtable}
\newpage
% ================== TEST CASE: KHÔNG CHỌN SAO ==================
\begin{longtable}{|l|p{12cm}|}
\hline
\textbf{Test case} &
UC09-TC05 - Gửi phản hồi không chọn mức sao đánh giá \\
\hline
\textbf{Test description} &
Sinh viên chỉ nhập nội dung nhận xét, không chọn sao đánh giá rồi nhấn ``Gửi''; hệ thống phát hiện thiếu mức đánh giá, hiển thị thông báo lỗi và không lưu phản hồi. \\
\hline
\textbf{Related screens} &
- Tab ``Phản hồi chất lượng''. \\
& - Màn hình ``Gửi phản hồi''. \\
\hline
\textbf{Pre-conditions} &
1. Tài khoản sinh viên tồn tại, đang hoạt động và đã đăng nhập. \\
& 2. Có ít nhất một buổi hẹn đã tham gia, hiển thị trong tab ``Phản hồi chất lượng'' với nút ``Phản hồi chất lượng''. \\
\hline
\textbf{Actions} &
1. Tại tab ``Phản hồi chất lượng'', sinh viên nhấn nút ``Phản hồi chất lượng'' của buổi hẹn. \\
& 2. Hệ thống mở màn hình ``Gửi phản hồi'' cho buổi hẹn đó. \\
& 3. Sinh viên không chọn bất kỳ sao nào (mức đánh giá để trống). \\
& 4. Sinh viên nhập nhận xét vào ô ``Nhận xét'', ví dụ: ``Buổi học khá ổn.''. \\
& 5. Sinh viên nhấn nút ``Gửi''. \\
\hline
\textbf{Inputs} &
- Mức đánh giá: không chọn (trống). \\
& - Nhận xét: ``Buổi học khá ổn.''. \\
\hline
\textbf{Expected Outputs} &
- Hệ thống không lưu phản hồi vào cơ sở dữ liệu. \\
& - Hệ thống hiển thị thông báo lỗi, ví dụ: ``Vui lòng chọn mức đánh giá trước khi gửi.''. \\
& - Màn hình ``Gửi phản hồi'' vẫn giữ lại nội dung đã nhập trong ô ``Nhận xét'' để sinh viên không phải gõ lại. \\
& - Sau lỗi, nếu quay lại tab ``Phản hồi chất lượng'', khung ``Lịch sử phản hồi'' của buổi hẹn không có thêm phản hồi mới. \\
\hline
\textbf{Testing environment} &
Web \\
\hline
\end{longtable}
\newpage
\subsubsection{Xem danh sách buổi gặp mặt}
\begin{longtable}{|l|p{12cm}|}
\hline
\textbf{Test case} & UC-10-TC01 — Xem danh sách buổi gặp mặt \\ \hline
\textbf{Test description} & Sinh viên hoặc tutor xem danh sách các buổi gặp mặt đã đăng ký thành công hoặc đã được duyệt \\ \hline
\textbf{Related screens} & Màn hình Danh sách buổi gặp mặt \\ \hline
\textbf{Pre-conditions} & 1. Sinh viên/Tutor đã đăng nhập \newline
2. Nếu là sinh viên thì phải đã đăng ký tutor \newline
3. Có ít nhất một buổi gặp đã đăng ký / được tạo \\ \hline
\textbf{Actions} & 1. Mở chức năng “Danh sách buổi gặp mặt” \newline
2. Hệ thống hiển thị danh sách các buổi gặp \newline
3. Nhấn vào một buổi để xem chi tiết \newline
4. Hệ thống hiển thị thông tin chi tiết buổi đó \newline
5. Người dùng nhấn Quay lại / Back \newline
6. Hệ thống quay về danh sách buổi gặp mặt \\ \hline
\textbf{Inputs} & Chọn buổi gặp với Chủ đề: Công nghệ phần mềm \\ \hline
\textbf{Expected Outputs} & 1. Hệ thống hiển thị chi tiết buổi gặp \newline
2. Quay lại danh sách, các buổi gặp vẫn hiển thị đúng \\ \hline
\textbf{Testing environment} & Web – Windows 11 \\ \hline
\end{longtable}
\newpage
\subsubsection{Hủy buổi gặp mặt}
\begin{longtable}{|l|p{12cm}|}
\hline
\textbf{Test case} & UC-11-TC01 — Hủy buổi gặp mặt với lý do \\ \hline
\textbf{Test description} & Actor (Sinh viên/Tutor) hủy buổi gặp đã lên lịch và nhập lý do hủy \\ \hline
\textbf{Related screens} & Màn hình hủy đăng ký buổi gặp mặt / hủy buổi gặp mặt \\ \hline
\textbf{Pre-conditions} & 1. Actor đã đăng nhập \newline
2. Buổi gặp chưa bắt đầu \newline
3. Buổi gặp đã được đăng ký, lên lịch \\ \hline
\textbf{Actions} & 1. Actor vào mục hủy đăng ký buổi gặp mặt / hủy buổi gặp mặt \newline
2. Hệ thống hiển thị danh sách buổi gặp đã đăng ký / đã lên lịch \newline
3. Actor chọn một buổi gặp cụ thể \newline
4. Hệ thống hiển thị form nhập lý do hủy \newline
5. Actor nhập lý do và bấm Xác nhận \newline
6. Hệ thống cập nhật slot rảnh của Tutor và trạng thái buổi gặp \newline
7. Hệ thống gửi thông báo xác nhận hủy: \newline
a. Nếu actor là Sinh viên → gửi thông báo cho Tutor \newline
b. Nếu actor là Tutor → gửi thông báo cho Sinh viên và Tutor \\ \hline
\textbf{Inputs} & Actor: Sinh viên/Tutor, Buổi gặp: 15/11/2025 14:00-15:00 , Lý do hủy: “Bận đột xuất” \\ \hline
\textbf{Expected Outputs} & 1. Hệ thống hiển thị “Hủy thành công” \newline
2. Cập nhật trạng thái slot: \newline
• Nếu actor là Sinh viên → mở slot Tutor, buổi gặp bị hủy cho sinh viên \newline
• Nếu actor là Tutor → buổi gặp bị hủy cho cả Tutor và Sinh viên \newline
3. Gửi thông báo hủy cho actor liên quan \\ \hline
\textbf{Testing environment} & Web – Windows 11 \\ \hline
\end{longtable}
\newpage
\subsubsection{Xử lý yêu cầu đặt lịch hẹn}

\begin{longtable}{|l|p{12cm}|}
\hline
\textbf{Test case} & UC-18-TC01 - Tutor đồng ý lịch hẹn\\ \hline
\textbf{Test description} & Tutor chấp nhận (đồng ý) một yêu cầu đặt lịch hẹn đang chờ xử lý từ sinh viên. \\ \hline
\textbf{Related screens} & - Giao diện danh sách yêu cầu lịch hẹn\newline 
- Giao diện xem chi tiết lịch hẹn\\ \hline
\textbf{Pre-conditions} & - Tutor đã đăng nhập vào hệ thống \newline
- Giao diện danh sách lịch hẹn đang hiển thị\newline 
- Tồn tại ít nhất một lịch hẹn với trạng thái đang chờ xử lí \\ \hline
\textbf{Actions} & - Từ giao diện danh sách yêu cầu lịch hẹn, tutor chọn một lịch hẹn để xem chi tiết \newline
- Hệ thống hiển thị thông tin chi tiết của lịch hẹn bao gồm họ tên sinh viên, nội dung, thời gian, và hình thức\newline
- Tutor nhấn nút "Phê duyệt"\newline 
- Hệ thống xử lí yêu cầu và hiển thị thông báo "Yêu cầu đã được phê duyệt"\newline
- Tutor chọn "Quay lại" hệ thống quay lại giao diện danh sách lịch hẹn\\ \hline
\textbf{Inputs} & - Danh sách yêu cầu lịch hẹn\\ \hline
\textbf{Expected Outputs} & Hệ thống xác nhận "Yêu cầu đã được phê duyệt" và gửi thông báo đến sinh viên đó\\ \hline
\textbf{Testing environment} & Giao diện web chạy trên trình duyệt, và vai trò người dùng là Tutor \\ \hline
\end{longtable}
\newpage
\begin{longtable}{|l|p{12cm}|}
\hline
\textbf{Test case} & UC-18-TC02 - Tutor từ chối lịch hẹn\\ \hline
\textbf{Test description} & Tutor từ chối một yêu cầu đặt lịch hẹn đang chờ xử lý từ sinh viên. \\ \hline
\textbf{Related screens} & - Giao diện danh sách yêu cầu lịch hẹn\newline 
- Giao diện xem chi tiết lịch hẹn\newline
- Pop-up "Lý do từ chối"\\ \hline
\textbf{Pre-conditions} & - Tutor đã đăng nhập vào hệ thống \newline
- Giao diện danh sách lịch hẹn đang hiển thị\newline 
- Tồn tại ít nhất một lịch hẹn với trạng thái đang chờ xử lí \\ \hline
\textbf{Actions} & - Từ giao diện danh sách yêu cầu lịch hẹn, tutor chọn một lịch hẹn để xem chi tiết \newline
- Hệ thống hiển thị thông tin chi tiết của lịch hẹn bao gồm họ tên sinh viên, nội dung, thời gian, và hình thức\newline
- Tutor nhấn nút "Từ chối"\newline 
- Hệ thống hiển thị Pop-up "Lý do từ chối", yêu cầu nhập lý do.\newline
- Tutor nhập lý do vào ô text và chọn "Xác nhận từ chối"\newline
- Hệ thống xử lý yêu cầu và thông báo "Đã từ chối lịch hẹn" \newline
- Tutor chọn "Quay lại" hệ thống quay lại giao diện danh sách lịch hẹn\\ \hline
\textbf{Inputs} & - Danh sách yêu cầu lịch hẹn\\ \hline
\textbf{Expected Outputs} & Hệ thống xác nhận "Đã từ chối lịch hẹn" và gửi thông báo đến sinh viên đó\\ \hline
\textbf{Testing environment} & Giao diện web chạy trên trình duyệt, và vai trò người dùng là Tutor \\ \hline
\end{longtable}

\end{document}


