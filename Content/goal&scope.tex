\subsubsection{Mục tiêu}
\setlength{\parindent}{2em}  % thụt đầu dòng 2em cho mỗi đoạn

Dự án Tutor/Mentor tại HCMUT được khởi xướng với mục tiêu xây dựng một hệ thống phần mềm hiện đại, đồng bộ và thân thiện, đóng vai trò là nền tảng trung tâm để quản lý, điều phối và nâng cao chất lượng chương trình. Thay vì vận hành rời rạc và thủ công, hệ thống này hướng đến số hóa toàn diện, từ khâu quản lý hồ sơ, đăng ký tham gia cho đến quản lý lịch hẹn, phản hồi và đánh giá. Qua đó, không chỉ giảm tải đáng kể khối lượng công việc hành chính cho cán bộ mà còn mang lại trải nghiệm thuận tiện, minh bạch và chủ động hơn cho sinh viên cũng như đội ngũ Tutor.

Bên cạnh chức năng quản lý cốt lõi, dự án còn nhấn mạnh việc khai thác dữ liệu thông minh để hỗ trợ các đơn vị chức năng như Phòng Đào tạo, Phòng Công tác Sinh viên theo dõi tiến độ, đánh giá hiệu quả và tối ưu hóa nguồn lực. Các dữ liệu này cũng trở thành căn cứ quan trọng để cộng điểm rèn luyện, xét học bổng hay triển khai chính sách hỗ trợ sinh viên phù hợp. Điều này không chỉ giúp cải thiện chất lượng hoạt động Tutor/Mentor mà còn nâng cao toàn diện trải nghiệm học tập và phúc lợi cho người học.

Trên nền tảng dữ liệu thống nhất và công nghệ hiện đại, dự án còn tăng cường việc phát triển một môi trường học tập thông minh và năng động hơn thông qua các tính năng mở rộng như AI gợi ý ghép cặp, cộng đồng học tập trực tuyến, tích hợp các chương trình học thuật và phi học thuật cùng khả năng hỗ trợ cá nhân hóa học tập. Những tính năng này sẽ giúp sinh viên chủ động hơn trong việc học và kết nối. Đây cũng là nền tảng để nhà trường tối ưu và phát triển thêm các dịch vụ học tập số, hướng tới một hệ sinh thái giáo dục số bền vững và linh hoạt, đáp ứng nhu cầu ngày càng đa dạng của người học.

Như vậy, mục tiêu của dự án không chỉ dừng lại ở việc tạo ra một công cụ quản lý đơn thuần mà còn là một bước tiến quan trọng trong chiến lược chuyển đổi số của HCMUT. Hệ thống sẽ trở thành cầu nối giữa sinh viên, tutor và nhà trường, thúc đẩy tương tác, lan tỏa tri thức và định hình một môi trường học tập hiện đại, sáng tạo và lấy người học làm trung tâm.


\subsubsection{Phạm vi}
Trong phạm vi \textit{(scope)} của dự án, hệ thống Tutor Support sẽ tập trung vào các chức năng cốt lõi để hỗ trợ hiệu quả cho cả sinh viên và tutor. Cụ thể, hệ thống cho phép quản lý hồ sơ tutor và sinh viên thông qua việc đồng bộ dữ liệu cá nhân cơ bản (họ tên, MSSV/Mã cán bộ, khoa/chuyên ngành, email học vụ…) từ HCMUT\_DATACORE, giúp giảm thiểu nhập liệu thủ công và đảm bảo tính chính xác. Sinh viên có thể đăng nhập bằng tài khoản SSO để đăng ký tham gia chương trình. Ngoài ra, hệ thống hỗ trợ việc ghép cặp tutor – sinh viên theo hai cách: sinh viên tự chọn hoặc được gợi ý dựa trên nhu cầu chuyên môn cần hỗ trợ và lịch rảnh Sau khi ghép cặp, sinh viên và tutor có thể chủ động đặt lịch, hủy hoặc đổi lịch, đồng thời hệ thống sẽ tự động gửi thông báo và nhắc nhở để cả hai bên có thể nhận được những thay đổi mới nhất. Các buổi gặp có thể diễn ra trực tiếp hoặc trực tuyến. Bên cạnh đó, hệ thống cho phép sinh viên gửi phản hồi, đánh giá chất lượng buổi học, còn tutor có thể ghi nhận tiến bộ của sinh viên; các dữ liệu này sẽ được tổng hợp thành báo cáo cho khoa/bộ môn và các phòng ban liên quan nhằm theo dõi hiệu quả của chương trình.

Về ranh giới \textit{(boundary)} với các hệ thống khác, Tutor Support sẽ giao tiếp trực tiếp với HCMUT\_SSO để thực hiện xác thực đăng nhập, đồng bộ dữ liệu cá nhân từ HCMUT\_DATACORE, và kết nối với HCMUT\_LIBRARY để sinh viên và tutor có thể truy cập vào nguồn tài liệu học thuật. Đây là các tích hợp nhằm đảm bảo tính đồng bộ, chính thống và an toàn dữ liệu. Dự án cũng xác định rõ những chức năng nằm ngoài phạm vi: không xử lý các vấn đề liên quan đến thanh toán học phí, không thay thế hoàn toàn hệ thống LMS của trường, không can thiệp vào việc quản lý chương trình giảng dạy, lịch học hay điểm số. Tutor Support chỉ giữ vai trò hỗ trợ bổ sung, đồng thời tạo giá trị riêng trong việc kết nối sinh viên và tutor một cách thuận tiện hơn.
 
