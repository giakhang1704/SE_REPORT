\subsubsection{Bối cảnh dự án}
Trong môi trường giáo dục đại học hiện nay, việc hỗ trợ sinh viên không chỉ dừng lại ở giảng dạy kiến thức trên lớp mà còn bao gồm tư vấn, kèm cặp và phát triển kỹ năng. Nhằm đáp ứng nhu cầu đó, Trường Đại học Bách Khoa – ĐHQG TP.HCM (HCMUT) đã triển khai chương trình Tutor/Mentor, trong đó các tutor là giảng viên, nghiên cứu sinh hoặc sinh viên năm trên có thành tích học tập tốt, được phân công để đồng hành và hỗ trợ các nhóm sinh viên trong quá trình học tập.  

Để quản lý chương trình hiệu quả hơn, mở rộng quy mô và tránh quản lý thủ công, rườm rà và dễ sai sót ,  nhà trường mong muốn xây dựng một hệ thống phần mềm để quản lý và
vận hành chương trình Tutor một cách hiệu quả. Hệ thống sẽ hỗ trợ quản lý hồ sơ sinh viên và tutor, đăng ký và ghép cặp, tổ chức và quản lý lịch tư vấn, đồng thời cung cấp công cụ thông báo, nhắc nhở và đánh giá. Bên cạnh đó, các khoa, phòng ban có thể khai thác dữ liệu tổng hợp để giám sát chất lượng đào tạo, phân bổ nguồn lực, cộng điểm rèn luyện hoặc xét học bổng cho sinh viên. 


Hệ thống sẽ được tích hợp với các hạ tầng công nghệ của HCMUT như \textbf{HCMUT\_SSO} (đăng nhập tập trung), \textbf{HCMUT\_DATACORE} (đồng bộ dữ liệu cá nhân và phân quyền tự động) và \textbf{HCMUT\_LIBRARY} (truy cập, chia sẻ học liệu chính thống). Việc tích hợp này giúp đảm bảo an toàn, đồng bộ dữ liệu và tạo điều kiện thuận lợi cho sinh viên, tutor cũng như các phòng ban trong quá trình sử dụng.  
Ngoài các chức năng cốt lõi, hệ thống cũng có thể mở rộng với các tính năng nâng cao như:  
Ghép cặp tutor – sinh viên thông minh dựa trên AI,  
     Xây dựng cộng đồng trực tuyến cho tutor và mentee,  
     Tổ chức các chương trình hỗ trợ học thuật và phi học thuật,  
     Cung cấp dịch vụ hỗ trợ học tập cá nhân hóa.  