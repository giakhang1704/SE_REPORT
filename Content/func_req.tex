\subsubsection{Đối với sinh viên}
\begin{itemize}
    \item Đăng nhập: 
    .\begin{itemize}
    \item Hệ thống phải cho phép sinh viên đăng nhập bằng tài khoản đã được cấp thông qua HCMUT\_SSO.
    \item Hệ thống phải kiểm tra tính hợp lệ của tên đăng nhập và mật khẩu.
    \item Sau khi xác thực thành công, hệ thống phải gán quyền “sinh viên” để truy cập các chức năng phù hợp.
    \item Nếu xác thực thất bại, hệ thống phải hiển thị thông báo lỗi và yêu cầu sinh viên nhập lại thông tin.
\end{itemize}

    \item Đăng xuất: \newline
    Hệ thống phải cho phép sinh viên đăng xuất để kết thúc phiên làm việc hiện tại. Sau khi đăng xuất, hệ thống phải đưa sinh viên về trang đăng nhập và vô hiệu hóa quyền truy cập còn lại.
    
    \item Xem hồ sơ cá nhân: \newline
Hệ thống phải cho phép sinh viên xem thông tin cá nhân của mình được đồng bộ từ HCMUT\_DATACORE, bao gồm họ tên, mã số sinh viên (MSSV), khoa, chuyên ngành, email học vụ và các thông tin liên quan đến học tập.  

    \item Xem tài liệu:\begin{itemize}
        \item Hệ thống phải cho phép sinh viên truy cập và xem các tài liệu học tập được cung cấp từ HCMUT\_LIBRARY.   
        \item Hệ thống phải cho phép sinh viên xem trực tuyến hoặc tải xuống tài liệu để sử dụng.  
    \end{itemize}
    \item Đăng ký Tutor: \newline
    Hệ thống phải cho phép sinh viên chọn tutor phù hợp dựa trên chuyên môn, lĩnh vực hỗ trợ hoặc gợi ý thông minh. Sau khi sinh viên gửi yêu cầu đăng ký, hệ thống phải lưu yêu cầu ở trạng thái “Chờ duyệt” trong 12 giờ và cho phép sinh viên hủy yêu cầu trong thời gian này.  

    \item Đăng kí buổi tư vấn: \newline
    Hệ thống phải cho phép sinh viên chọn và đăng ký buổi tư vấn còn chỗ trống do tutor tổ chức. 
    \item  Hủy đăng kí buổi gặp mặt: \newline   
    Hệ thống phải cho phép sinh viên huỷ các buổi tư vấn và buổi hẹn đã đăng ký trong thời gian cho phép (trước khi buổi gặp mặt bắt đầu).
    \item Đặt lịch hẹn: \newline
    Hệ thống phải cho phép sinh viên đặt lịch hẹn riêng với tutor dựa trên lịch rảnh do tutor cung cấp và nhập nội dung cần được hỗ trợ cho buổi hẹn.  
 
    \item Phản hồi chất lượng buổi học:
    \begin{itemize}
    \item Hệ thống phải cung cấp form phản hồi để sinh viên đánh giá chất lượng buổi học sau mỗi buổi tư vấn.  
        \item Hệ thống phải cho phép sinh viên xem lại lịch sử phản hồi của mình.  
        \item Form phản hồi phải bao gồm:  
        \begin{itemize}
            \item Điểm số đánh giá (1 đến 5).  
            \item Nhận xét ngắn gọn về nội dung buổi học, thái độ và khả năng truyền đạt của tutor.  
        \end{itemize}
    \end{itemize}  
\end{itemize}

\subsubsection{Đối với Tutor}

\begin{itemize}
    \item Đăng nhập:
    \begin{itemize}
        \item Hệ thống phải cho phép tutor đăng nhập bằng tài khoản HCMUT\_SSO.  
        \item Hệ thống phải kiểm tra tính hợp lệ của tên đăng nhập và mật khẩu.  
        \item Sau khi xác thực thành công, hệ thống phải gán quyền “tutor” để truy cập các chức năng tương ứng.  
        \item Nếu xác thực thất bại, hệ thống phải hiển thị thông báo lỗi và yêu cầu tutor nhập lại thông tin.  
    \end{itemize}

    \item Đăng xuất: 
     Hệ thống phải cho phép tutor đăng xuất để kết thúc phiên làm việc hiện tại. Sau khi đăng xuất, hệ thống phải đưa tutor về trang đăng nhập và vô hiệu hóa quyền truy cập còn lại.
    \item Xem hồ sơ cá nhân:
    Hệ thống phải cho phép tutor xem thông tin cá nhân của mình được đồng bộ từ HCMUT\_DATACORE, bao gồm họ tên, mã định danh, email, số điện thoại và thông tin liên hệ khác.   

    \item Thiết lập lịch rảnh: \newline
    Hệ thống phải cho phép tutor thiết lập lịch rảnh để sinh viên tham khảo khi đặt lịch hẹn. Lịch rảnh bao gồm thông tin về ngày, giờ. Khi tutor thay đổi hoặc hủy lịch rảnh, hệ thống phải tự động cập nhật lại.

    \item Tạo buổi tư vấn :  \newline
    Hệ thống cho phép tutor tạo các buổi tư vấn với thông tin về ngày, giờ, chủ đề và hình thức (online hoặc offline). Sau khi tạo, hệ thống phải gửi thông báo buổi tư vấn đến các sinh viên đã chọn tutor. 
    \item Xử lý yêu cầu đặt lịch hẹn:  
Hệ thống phải cho phép tutor xem các yêu cầu đặt buổi gặp từ sinh viên và duyệt hoặc từ chối từng yêu cầu.  

    \item Hủy buổi gặp mặt: \newline
    Hệ thống cho phép tutor hủy các buổi tư vấn đã tạo, buổi hẹn đã duyệt và nhập lý do . 
    \item Xem tài liệu: 
    \begin{itemize}
        \item Hệ thống phải cho phép tutor truy cập và xem các tài liệu học tập được cung cấp từ HCMUT\_LIBRARY.   
        \item Hệ thống phải cho phép tutor xem trực tuyến hoặc tải xuống tài liệu để sử dụng.  
    \end{itemize}
    \item Đăng tải tài liệu: \newline
    Hệ thống phải cho phép tutor đăng tải tài liệu học tập với thông tin cơ bản (file, tiêu đề, mô tả, loại tài liệu). Tài liệu được lưu ở trạng thái “Chờ duyệt” và chỉ hiển thị cho sinh viên sau khi được quản trị viên phê duyệt.
    \item Xóa tài liện: \newline
     Hệ thống phải cho phép tutor xóa các tài liệu đã đăng tải nếu không còn sử dụng hoặc muốn thay thế.
    \item Ghi nhận tiến độ học tập của sinh viên: 
    \begin{itemize}
        \item Hệ thống phải cho phép tutor ghi nhận tiến độ và phản hồi về quá trình học tập của sinh viên.  
        \item Hệ thống phải cho phép lưu trữ dữ liệu này để sử dụng trong các báo cáo thống kê.  
    \end{itemize}
\end{itemize}

\subsubsection{Đối với Điều phối viên – Coordinator}
\begin{itemize}
    \item Quản lý thông tin tutor và sinh viên:\newline
    Hệ thống phải cho phép điều phối viên  xem và quản lý thông tin của các tutor và sinh viên trong chương trình. Điều này bao gồm việc theo dõi các hồ sơ cá nhân, chuyên môn, và nhu cầu hỗ trợ của sinh viên.
     \item Hệ thống phải cho phép điều phối viên đăng nhập/đăng xuất: Gọi API SSO, gán quyền “Coordinator”
    \item Gửi báo cáo cho các phòng ban: \newline
    Hệ thống phải cho phép điều phối viên gửi các báo cáo đã được hệ thống tạo sẵn tới các phòng ban và khoa/bộ môn. Điều phối viên có thể chọn báo cáo, định dạng xuất báo cáo, và phương thức gửi.
    \item Điều phối khung chương trình chung cho các tutor: \newline
    Hệ thống phải cho phép điều phối viên kiểm soát khung chương trình chung cho các tutor, từ đó các tutor thực hiện và đảm bảo tính đồng bộ trong các buổi hẹn. Các buổi tư vấn diễn ra theo một chương trình đã được chuẩn bị sẵn, với các chủ đề, thời gian và tài liệu học tập được thống nhất.
    
\end{itemize}

\subsubsection{Đối với Ban quản lý (Phòng Đào tạo, Phòng Công tác Sinh viên, Khoa/Bộ môn)} 

\begin{itemize}
    \item	Hệ thống phải cung cấp báo cáo tổng quan về chương trình: số lượng Tutor/sinh viên đăng kí tham gia, phòng học sử dụng, số buổi đã tổ chức, tỷ lệ tham gia/hoàn thành,…
    \item  Hỗ thống phải cho phép xuất báo cáo (PDF/Excel/CSV) để lưu trữ hoặc tích hợp hệ thống khác
    \item Hệ thống cho phép gửi báo cáo định kỳ tự động theo tuần/tháng hoặc theo chu kỳ do quản trị viên cấu hình.
    \item Hệ thống cho phép gửi báo cáo thủ công khi cần: điều phối viên là người chính, quản trị viên có thể can thiệp khi cần.
    \item Hệ thống cho phép các phòng ban nhận báo cáo  qua email mà không cần tài khoản đăng nhập hệ thống Tutor.
\end{itemize}

\subsubsection{Đối với HCMUT\_DATACORE} 
\begin{itemize} 
    \item Hệ thống phải cho phép cung cấp dữ liệu cá nhân của sinh viên và tutor (họ tên, MSSV, mã tutor, khoa/chuyên ngành, email học vụ, trạng thái học tập/giảng dạy, \ldots) cho các hệ thống khác để phục vụ quá trình quản lý và học tập.
    \item Hệ thống phải cho phép đồng bộ dữ liệu cá nhân với các hệ thống liên quan để đảm bảo tính chính xác và nhất quán.
    \item Hệ thống phải cho phép chia sẻ dữ liệu người dùng, tình trạng học tập và điểm số giữa các hệ thống HCMUT (DATACORE, SSO và Library) nhằm hỗ trợ tra cứu và quản lý.
\end{itemize} 

\subsubsection{Đối với HCMUT\_SSO} 
\begin{itemize} 
    \item Hệ thống phải cho phép người dùng đăng nhập tập trung một lần để truy cập các ứng dụng và dịch vụ khác của HCMUT mà không cần đăng nhập lại.
    \item Hệ thống phải cho phép người dùng đổi mật khẩu và khôi phục tài khoản để đảm bảo an toàn và bảo mật thông tin cá nhân.
    \item Hệ thống phải cho phép cấp quyền truy cập theo vai trò (sinh viên, tutor, điều phối viên, \ldots) để giới hạn phạm vi chức năng phù hợp với từng loại người dùng.
\end{itemize} 

\subsubsection{Đối với HCMUT\_Library} 
\begin{itemize} 
    \item Hệ thống phải cho phép liên kết các khóa học với tài liệu số (giáo trình, bài giảng, tài liệu tham khảo, \ldots) để sinh viên và tutor có thể truy cập hoặc tải về phục vụ học tập.
    \item Hệ thống phải cho phép tutor đăng tải và xóa tài liệu học tập cho sinh viên thông qua hệ thống thư viện số.
    \item Hệ thống phải cho phép ghi nhận và thống kê lượt truy cập hoặc tải tài liệu để hỗ trợ công tác quản lý và cải thiện chất lượng học liệu.
\end{itemize}