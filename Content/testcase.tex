\newpage
\subsubsection{Đăng nhập}
\begin{longtable}{|l|p{12cm}|}
\hline
\textbf{Test case} &
UC01-TC01 - Đăng nhập thành công qua HCMUT\_SSO \\
\hline
\textbf{Test description} &
User có tài khoản hợp lệ và có quyền sử dụng hệ thống, đăng nhập thành công thông qua HCMUT\_SSO, phiên làm việc được tạo và được chuyển đến trang chính tương ứng với vai trò. \\
\hline
\textbf{Related screens} &
- Trang chủ Tutor Support System với nút ``Đăng nhập qua HCMUT\_SSO''. \\
& - Trang đăng nhập HCMUT\_SSO (form BKNetID + mật khẩu). \\
& - Trang chính sau đăng nhập (dashboard tương ứng role: Sinh viên / Tutor / Điều phối viên). \\
\hline
\textbf{Pre-conditions} &
1. User có tài khoản hợp lệ trong hệ thống HCMUT\_SSO (BKNetID + mật khẩu đúng). \\
& 2. Tài khoản thuộc nhóm được phép dùng Tutor Support System (role hợp lệ). \\
& 3. HCMUT\_SSO và HCMUT\_DATACORE đang hoạt động bình thường. \\
& 4. User chưa đăng nhập vào Tutor Support System trên trình duyệt hiện tại. \\
\hline
\textbf{Actions} &
1. User truy cập trang chủ Tutor Support System. \\
& 2. User nhấn nút ``Đăng nhập qua HCMUT\_SSO''. \\
& 3. Hệ thống chuyển hướng sang trang đăng nhập HCMUT\_SSO. \\
& 4. User nhập đúng BKNetID và mật khẩu, sau đó nhấn nút đăng nhập. \\
& 5. HCMUT\_SSO xác thực thông tin đăng nhập thành công và trả về token xác thực cho Tutor Support System. \\
& 6. Tutor Support System dùng token truy vấn HCMUT\_DATACORE để lấy thông tin cơ bản (họ tên, email, role, trạng thái tài khoản, \ldots). \\
& 7. Hệ thống kiểm tra role và trạng thái tài khoản, tạo phiên làm việc cho user. \\
& 8. Hệ thống chuyển hướng user đến trang chính tương ứng với vai trò (dashboard Sinh viên / Tutor / Điều phối viên). \\
\hline
\textbf{Inputs} &
- BKNetID: tài khoản hợp lệ \\
& - Mật khẩu: mật khẩu tương ứng, hợp lệ. \\
\hline
\textbf{Expected Outputs} &
- User được xác thực thành công từ HCMUT\_SSO. \\
& - Dữ liệu người dùng được lấy thành công từ HCMUT\_DATACORE. \\
& - Phiên làm việc (session) của user được tạo trong Tutor Support System. \\
& - User được chuyển tới đúng trang chính theo vai trò, không có thông báo lỗi. \\
\hline
\textbf{Testing environment} &
Web \\
\hline
\end{longtable}


\newpage
% ================== UC01_TC02: SAI THÔNG TIN ĐĂNG NHẬP (E1) ==================
\begin{longtable}{|l|p{12cm}|}
\hline
\textbf{Test case} &
UC01-TC02 - Sai thông tin đăng nhập (mật khẩu/BKNetID không đúng) \\
\hline
\textbf{Test description} &
User nhập sai BKNetID hoặc mật khẩu tại HCMUT\_SSO, hệ thống hiển thị thông báo lỗi trên form và không đăng nhập vào Tutor Support System. \\
\hline
\textbf{Related screens} &
- Trang chủ Tutor Support System. \\
& - Trang đăng nhập HCMUT\_SSO (form đăng nhập). \\
\hline
\textbf{Pre-conditions} &
1. HCMUT\_SSO đang hoạt động bình thường. \\
& 2. User chưa đăng nhập vào Tutor Support System. \\
\hline
\textbf{Actions} &
1. User truy cập trang chủ Tutor Support System và nhấn ``Đăng nhập qua HCMUT\_SSO''. \\
& 2. Hệ thống chuyển sang trang đăng nhập HCMUT\_SSO. \\
& 3. User nhập BKNetID hoặc mật khẩu không chính xác, nhấn đăng nhập. \\
\hline
\textbf{Inputs} &
- BKNetID(đúng) hoặc một mã bất kỳ. \\
& - Mật khẩu không khớp mật khẩu thật \\
\hline
\textbf{Expected Outputs} &
- HCMUT\_SSO hiển thị thông báo lỗi trực tiếp trên form, ví dụ: ``Thông tin đăng nhập không chính xác.''. \\
& - User vẫn ở trên trang đăng nhập HCMUT\_SSO, không được chuyển hướng về Tutor Support System. \\
& - Không có phiên làm việc nào được tạo ở Tutor Support System . \\
\hline
\textbf{Testing environment} &
Web\\
\hline
\end{longtable}


\newpage
% ================== UC01_TC03: TÀI KHOẢN BỊ KHÓA / HẾT HẠN / YÊU CẦU ĐỔI MẬT KHẨU (E2) ==================
\begin{longtable}{|l|p{12cm}|}
\hline
\textbf{Test case} &
UC01-TC03 - Tài khoản bị khóa / hết hạn / yêu cầu đổi mật khẩu \\
\hline
\textbf{Test description} &
User nhập đúng BKNetID và mật khẩu nhưng tài khoản ở trạng thái bị khóa, hết hạn hoặc bắt buộc đổi mật khẩu; HCMUT\_SSO hiển thị thông báo tương ứng và user không đăng nhập được vào Tutor Support System. \\
\hline
\textbf{Related screens} &
Trang đăng nhập HCMUT\_SSO (form đăng nhập). \\
\hline
\textbf{Pre-conditions} &
1. Tài khoản BKNetID tồn tại nhưng đang ở trạng thái: bị khóa / hết hạn / yêu cầu đổi mật khẩu. \\
& 2. HCMUT\_SSO hoạt động bình thường. \\
& 3. User chưa có phiên đăng nhập Tutor Support System. \\
\hline
\textbf{Actions} &
1. User từ trang chủ Tutor Support System nhấn ``Đăng nhập qua HCMUT\_SSO'' để mở trang đăng nhập SSO. \\
& 2. User nhập đúng BKNetID và mật khẩu của tài khoản đang bị khóa/hết hạn/yêu cầu đổi mật khẩu. \\
& 3. User nhấn đăng nhập. \\
\hline
\textbf{Inputs} &
- BKNetID: tài khoản tồn tại nhưng bị khóa/hết hạn. \\
& - Mật khẩu: hợp lệ với tài khoản đó. \\
\hline
\textbf{Expected Outputs} &
- HCMUT\_SSO hiển thị thông báo lỗi tương ứng trên form đăng nhập (E2.1). \\
& - User không được chuyển hướng về Tutor Support System, không có token xác thực được trả về. \\
& - Use case kết thúc không thành công tại SSO, không tạo phiên làm việc ở Tutor Support System (E2.2). \\
\hline
\textbf{Testing environment} &
Web \\
\hline
\end{longtable}


\newpage
% ================== UC01_TC04: KHÔNG CÓ QUYỀN TRUY CẬP HỆ THỐNG (E3) ==================
\begin{longtable}{|l|p{12cm}|}
\hline
\textbf{Test case} &
UC01-TC04- Đăng nhập bằng tài khoản không có quyền sử dụng Tutor Support System \\
\hline
\textbf{Test description} &
User đăng nhập SSO thành công nhưng role không thuộc nhóm được phép truy cập Tutor Support System; hệ thống hiển thị thông báo ``Bạn không có quyền truy cập hệ thống.'' và không tạo phiên làm việc. \\
\hline
\textbf{Related screens} &
- Trang chủ Tutor Support System. \\
& - Trang đăng nhập HCMUT\_SSO. \\
& - Trang thông báo lỗi quyền truy cập của Tutor Support System. \\
\hline
\textbf{Pre-conditions} &
1. User có tài khoản hợp lệ trong HCMUT\_SSO. \\
& 2. Role của user \textbf{không} nằm trong tập role được phép dùng Tutor Support System. \\
& 3. HCMUT\_SSO và HCMUT\_DATACORE hoạt động bình thường. \\
\hline
\textbf{Actions} &
1. User từ trang chủ Tutor Support System nhấn ``Đăng nhập qua HCMUT\_SSO''. \\
& 2. User đăng nhập thành công trên HCMUT\_SSO (đúng BKNetID và mật khẩu). \\
& 3. Tutor Support System nhận token, truy vấn HCMUT\_DATACORE lấy thông tin user. \\
& 4. Hệ thống kiểm tra role, phát hiện user không thuộc nhóm được phép sử dụng. \\
\hline
\textbf{Inputs} &
- BKNetID + mật khẩu của user có role không hợp lệ (ví dụ thuộc hệ thống khác). \\
\hline
\textbf{Expected Outputs} &
- Tutor Support System hiển thị thông báo: ``Bạn không có quyền truy cập hệ thống.'' . \\
& - Phiên đăng nhập không được tạo; user không truy cập được vào bất kỳ trang chức năng nào. \\
& - Use case kết thúc không thành công, nhưng phiên SSO có thể vẫn tồn tại độc lập. \\
\hline
\textbf{Testing environment} &
Web \\
\hline
\end{longtable}


\newpage
% ================== UC01_TC05: LỖI KẾT NỐI ĐẾN HCMUT_SSO (E4) ==================
\begin{longtable}{|l|p{12cm}|}
\hline
\textbf{Test case} &
UC01-TC05 - Lỗi kết nối đến HCMUT\_SSO \\
\hline
\textbf{Test description} &
Tutor Support System không thể kết nối đến dịch vụ HCMUT\_SSO khi user yêu cầu đăng nhập; hệ thống báo lỗi và use case kết thúc không thành công. \\
\hline
\textbf{Related screens} &
Trang chủ / trang đăng nhập của Tutor Support System (nơi hiển thị nút ``Đăng nhập qua HCMUT\_SSO''). \\
\hline
\textbf{Pre-conditions} &
1. HCMUT\_SSO bị ngừng hoạt động hoặc kết nối mạng từ Tutor Support System đến SSO bị lỗi. \\
& 2. User chưa đăng nhập vào Tutor Support System. \\
\hline
\textbf{Actions} &
1. User truy cập trang chủ Tutor Support System. \\
& 2. User nhấn nút ``Đăng nhập qua HCMUT\_SSO''. \\
& 3. Hệ thống cố gắng chuyển hướng/kết nối tới HCMUT\_SSO nhưng bị lỗi kết nối. \\
\hline
\textbf{Inputs} &
Không có dữ liệu nhập từ user ngoài thao tác nhấn nút đăng nhập SSO. \\
\hline
\textbf{Expected Outputs} &
- Tutor Support System hiển thị thông báo: ``Không thể kết nối đến dịch vụ xác thực, vui lòng thử lại sau.'' . \\
& - Không có phiên làm việc được tạo, user vẫn ở trang login/landing của Tutor Support System. \\
\hline
\textbf{Testing environment} &
Web. \\
\hline
\end{longtable}


\newpage
% ================== UC01_TC06: LỖI ĐỒNG BỘ DỮ LIỆU HCMUT_DATACORE (E5) ==================
\begin{longtable}{|l|p{12cm}|}
\hline
\textbf{Test case} &
UC01-TC06 - Lỗi đồng bộ dữ liệu từ HCMUT\_DATACORE \\
\hline
\textbf{Test description} &
User đăng nhập SSO thành công, nhưng Tutor Support System không lấy được thông tin người dùng từ HCMUT\_DATACORE; hệ thống hiển thị thông báo lỗi, hủy phiên đăng nhập và không cho truy cập hệ thống. \\
\hline
\textbf{Related screens} &
- Trang đăng nhập HCMUT\_SSO. \\
& - Trang thông báo lỗi của Tutor Support System sau khi nhận token. \\
\hline
\textbf{Pre-conditions} &
1. User có tài khoản hợp lệ và đăng nhập được vào HCMUT\_SSO. \\
& 2. Dịch vụ HCMUT\_DATACORE gặp lỗi (không truy cập được, trả về lỗi, hoặc dữ liệu không đầy đủ). \\
\hline
\textbf{Actions} &
1. User từ trang chủ Tutor Support System chọn ``Đăng nhập qua HCMUT\_SSO'' và đăng nhập thành công trên SSO. \\
& 2. Tutor Support System nhận token từ HCMUT\_SSO. \\
& 3. Hệ thống dùng token truy vấn HCMUT\_DATACORE để lấy thông tin người dùng. \\
& 4. Việc truy vấn thất bại hoặc trả về dữ liệu lỗi/thiếu. \\
\hline
\textbf{Inputs} &
- BKNetID + mật khẩu hợp lệ (đăng nhập SSO thành công). \\
\hline
\textbf{Expected Outputs} &
- Tutor Support System hiển thị thông báo: ``Không thể tải thông tin người dùng, vui lòng thử lại sau.'' . \\
& - Hệ thống hủy phiên đăng nhập hiện tại, không cho user truy cập bất kỳ chức năng nào . \\
& - Use case kết thúc không thành công (E5.3). \\
\hline
\textbf{Testing environment} &
Web. \\
\hline
\end{longtable}


\newpage
\subsubsection{Đăng xuất}
% ================== UC02_TC01: ĐĂNG XUẤT THÀNH CÔNG ==================
\begin{longtable}{|l|p{12cm}|}
\hline
\textbf{Test case} &
UC02-TC01 - Đăng xuất thành công \\
\hline
\textbf{Test description} &
User đang đăng nhập và chủ động chọn chức năng Đăng xuất; hệ thống kết thúc phiên làm việc, vô hiệu hóa token và chuyển về trang đăng nhập với thông báo thành công. \\
\hline
\textbf{Related screens} &
- Trang bất kỳ trong Tutor Support System có hiển thị menu người dùng / biểu tượng tài khoản. \\
& - Hộp thoại xác nhận ``Bạn có chắc chắn muốn đăng xuất?''. \\
& - Trang đăng nhập Tutor Support System. \\
\hline
\textbf{Pre-conditions} &
1. User (Sinh viên / Tutor / Điều phối viên) đã đăng nhập thành công vào hệ thống. \\
& 2. Phiên làm việc của user đang hoạt động, token xác thực còn hiệu lực. \\
\hline
\textbf{Actions} &
1. User nhấn vào biểu tượng tài khoản hoặc menu người dùng ở góc trên. \\
& 2. Hệ thống hiển thị menu với tùy chọn ``Đăng xuất''. \\
& 3. User chọn mục ``Đăng xuất''. \\
& 4. Hệ thống hiển thị hộp thoại xác nhận ``Bạn có chắc chắn muốn đăng xuất?'' với hai nút: ``Đăng xuất'' và ``Hủy''. \\
& 5. User chọn nút xác nhận Đăng xuất. \\
& 6. Hệ thống vô hiệu hóa token xác thực hiện tại. \\
& 7. Hệ thống xóa thông tin phiên làm việc (session, cookie liên quan). \\
& 8. Hệ thống chuyển hướng user về trang đăng nhập. \\
& 9. Hệ thống hiển thị thông báo ``Đăng xuất thành công!'' trên trang đăng nhập. \\
\hline
\textbf{Inputs} &
Không có dữ liệu nhập dạng text; chỉ có thao tác bấm menu và xác nhận Đăng xuất. \\
\hline
\textbf{Expected Outputs} &
- Phiên làm việc của user bị xóa; token xác thực bị vô hiệu hóa. \\
& - User được chuyển về trang đăng nhập, không thể truy cập lại các trang nội bộ nếu không đăng nhập lại. \\
& - Thông báo ``Đăng xuất thành công!'' được hiển thị rõ ràng. \\
\hline
\textbf{Testing environment} &
Web \\
\hline
\end{longtable}


\newpage
% ================== UC02_TC02: HỦY ĐĂNG XUẤT (ALTERNATIVE FLOW 5a) ==================
\begin{longtable}{|l|p{12cm}|}
\hline
\textbf{Test case} &
UC02-TC02 - Hủy thao tác đăng xuất \\
\hline
\textbf{Test description} &
User chọn Đăng xuất nhưng bấm ``Hủy'' trong hộp thoại xác nhận; hệ thống đóng hộp thoại và user tiếp tục sử dụng hệ thống như bình thường. \\
\hline
\textbf{Related screens} &
Trang bất kỳ trong hệ thống + hộp thoại xác nhận Đăng xuất. \\
\hline
\textbf{Pre-conditions} &
1. User đã đăng nhập và phiên làm việc đang hoạt động. \\
\hline
\textbf{Actions} &
1. User mở menu người dùng và chọn ``Đăng xuất''. \\
& 2. Hệ thống hiển thị hộp thoại xác nhận ``Bạn có chắc chắn muốn đăng xuất?''. \\
& 3. User chọn nút ``Hủy'' trên hộp thoại xác nhận. \\
\hline
\textbf{Inputs} &
Thao tác bấm vào nút ``Hủy'' trên hộp thoại xác nhận Đăng xuất. \\
\hline
\textbf{Expected Outputs} &
- Hệ thống đóng hộp thoại xác nhận (không còn hiển thị trên màn hình). \\
& - Phiên làm việc và token xác thực của user vẫn giữ nguyên, không bị xóa. \\
& - User vẫn ở lại trang hiện tại và tiếp tục sử dụng hệ thống bình thường. \\
\hline
\textbf{Testing environment} &
Chrome / Windows 10; hệ thống backend hoạt động bình thường. \\
\hline
\end{longtable}


\newpage
% ================== UC02_TC03: LỖI VÔ HIỆU HÓA TOKEN (EXCEPTION E1) ==================
\begin{longtable}{|l|p{12cm}|}
\hline
\textbf{Test case} &
UC02-TC03 - Lỗi khi vô hiệu hóa token đăng nhập \\
\hline
\textbf{Test description} &
Trong quá trình đăng xuất, hệ thống gặp lỗi khi gọi backend để vô hiệu hóa token; hệ thống ghi log lỗi, xóa thông tin phiên làm việc phía client và chuyển user về trang đăng nhập. \\
\hline
\textbf{Related screens} &
- Trang bất kỳ trong hệ thống. \\
& - Hộp thoại xác nhận Đăng xuất. \\
& - Trang đăng nhập. \\
\hline
\textbf{Pre-conditions} &
1. User đã đăng nhập và phiên làm việc đang hoạt động. \\
& 2. Môi trường kiểm thử được cấu hình mô phỏng lỗi ở bước vô hiệu hóa token trên server (backend trả lỗi). \\
\hline
\textbf{Actions} &
1. User mở menu người dùng và chọn ``Đăng xuất''. \\
& 2. Hệ thống hiển thị hộp thoại xác nhận, user chọn xác nhận Đăng xuất. \\
& 3. Hệ thống gửi yêu cầu vô hiệu hóa token đến backend và nhận lỗi. \\
\hline
\textbf{Inputs} &
Thao tác xác nhận Đăng xuất; không có input text. \\
\hline
\textbf{Expected Outputs} &
- Backend ghi lại log lỗi vô hiệu hóa token (E1.1). \\
& - Ở phía client, hệ thống vẫn xóa thông tin phiên làm việc hiện tại (session, cookie) . \\
& - User được chuyển hướng về trang đăng nhập; từ đó không thể truy cập lại trang nội bộ nếu chưa đăng nhập lại . \\
& - Có thể (tuỳ thiết kế) hiển thị thông báo chung ``Đã đăng xuất, nhưng có lỗi nội bộ, vui lòng đăng nhập lại sau.''. \\
\hline
\textbf{Testing environment} &
Web \\
\hline
\end{longtable}


\newpage
% ================== UC02_TC04: PHIÊN ĐÃ HẾT HẠN TRƯỚC KHI ĐĂNG XUẤT (EXCEPTION E2) ==================
\begin{longtable}{|l|p{12cm}|}
\hline
\textbf{Test case} &
UC02-TC04 - Phiên làm việc đã hết hạn trước khi người dùng chọn Đăng xuất \\
\hline
\textbf{Test description} &
Phiên làm việc của user đã hết hạn do timeout; khi user thao tác Đăng xuất, hệ thống phát hiện session hết hạn và chuyển trực tiếp về trang đăng nhập với thông báo tương ứng. \\
\hline
\textbf{Related screens} &
Trang bất kỳ trong hệ thống (có thể đã cũ) và trang đăng nhập. \\
\hline
\textbf{Pre-conditions} &
1. User đã từng đăng nhập vào hệ thống. \\
& 2. Thời gian không hoạt động vượt quá thời gian timeout phiên; session trên server đã bị hủy. \\
& 3. Trình duyệt của user vẫn còn trang hệ thống mở (UI chưa refresh). \\
\hline
\textbf{Actions} &
1. User trên trang cũ nhấn vào menu người dùng và chọn ``Đăng xuất''. \\
& 2. Hệ thống gửi yêu cầu đăng xuất, nhưng phát hiện phiên làm việc đã hết hạn. \\
\hline
\textbf{Inputs} &
Thao tác chọn ``Đăng xuất'' trên UI sau khi phiên đã hết hạn. \\
\hline
\textbf{Expected Outputs} &
- Hệ thống không xử lý như đăng xuất thông thường mà nhận diện phiên đã hết hạn. \\
& - User được chuyển hướng ngay về trang đăng nhập. \\
& - Trên trang đăng nhập hiển thị thông báo, ví dụ: ``Phiên làm việc đã hết hạn.''. \\
& - User không còn truy cập được các trang nội bộ nếu không đăng nhập lại. \\
\hline
\textbf{Testing environment} &
Web \\
\hline
\end{longtable}


\newpage
\subsubsection{Đăng ký Tutor}

\begin{longtable}{|l|p{12cm}|}
\hline
\textbf{Test case} & UC07-TC01 — Đăng ký Tutor thành công \\ \hline
\textbf{Test description} & Sinh viên đăng ký Tutor với đầy đủ thông tin hợp lệ \\ \hline
\textbf{Related screens} & Màn hình Đăng ký Tutor \\ \hline
\textbf{Pre-conditions} & 1. Sinh viên đã đăng nhập \\ 
& 2. Chương trình Tutor đang mở đăng ký \\ \hline
\textbf{Actions} & 1. Sinh viên mở chức năng “Đăng ký Tutor” \\
& 2. Hệ thống hiển thị giao diện chọn môn học/lĩnh vực \\
& 3. Sinh viên chọn môn học Nguyên lý ngôn ngữ lập trình và bấm “Tiếp theo” \\
& 4. Hệ thống hiển thị danh sách tutor phù hợp \\
& 5. Sinh viên chọn đăng ký tutor Nguyễn Thị B và bấm xác nhận \\
& 6. Hệ thống hiển thị màn hình Chờ duyệt 12h với nút “Hủy đăng ký” \\ \hline
\textbf{Inputs} & Môn học: Nguyên lý ngôn ngữ lập trình; Tutor: Nguyễn Thị B \\ \hline
\textbf{Expected Outputs} & 1. Hệ thống hiển thị danh sách tutor phù hợp \\
& 2. Hệ thống hiển thị trạng thái Chờ duyệt 12h \\
& 3. Yêu cầu đăng ký được lưu trong hệ thống \\
& 4. Nút Hủy đăng ký khả dụng \\ \hline
\textbf{Testing environment} & Web – Windows 11 \\ \hline
\end{longtable}

\newpage
\begin{longtable}{|l|p{12cm}|}
\hline
\textbf{Test case} & UC07-TC02 — Bỏ trống thông tin môn học \\ \hline
\textbf{Test description} & Sinh viên không chọn môn học ở bước 1 \\ \hline
\textbf{Related screens} & Màn hình nhập thông tin \\ \hline
\textbf{Pre-conditions} & 1. Sinh viên đã đăng nhập \\ 
& 2. Chương trình Tutor đang mở đăng ký \\ \hline
\textbf{Actions} & 1. Sinh viên mở chức năng “Đăng ký Tutor” \\
& 2. Hệ thống hiển thị giao diện chọn môn học \\
& 3. Sinh viên bỏ trống môn học \\
& 4. Sinh viên Bấm “Tiếp theo” \\ \hline
\textbf{Inputs} & Môn học: (trống) \\ \hline
\textbf{Expected Outputs} & 1. Hệ thống hiển thị thông báo lỗi “Vui lòng nhập đầy đủ thông tin” \\
& 2. Không chuyển sang màn hình hiển thị danh sách tutor. \\ \hline
\textbf{Testing environment} & Web – Windows 11 \\ \hline
\end{longtable}
\newpage
\begin{longtable}{|l|p{12cm}|}
\hline
\textbf{Test case} & UC07-TC03 — Hủy đăng ký trong khi chờ duyệt và đăng ký mới \\ \hline
\textbf{Test description} & Sinh viên hủy yêu cầu đăng ký tutor đang chờ duyệt và ngay lập tức thực hiện đăng ký mới \\ \hline
\textbf{Related screens} & Màn hình Chờ duyệt 12h \\ \hline
\textbf{Pre-conditions} & 1. Sinh viên đã đăng nhập \\
& 2. Yêu cầu đăng ký tutor đang ở trạng thái Chờ duyệt 12h \\ \hline
\textbf{Actions} & 1. Sinh viên mở chức năng “Đăng ký Tutor” \\
& 2. Hệ thống hiển thị màn hình Chờ duyệt 12h \\
& 3. Nhấn nút Hủy đăng ký \\
& 4. Xác nhận hủy \\
& 5. Hệ thống hiển thị thông báo “Hủy thành công” \\
& 6. Sinh viên chọn Đăng kí mới \\
& 7. Hệ thống quay về hiển thị danh sách tutor phù hợp \\
& 8. Sinh viên chọn tutor mới và bấm “Xác nhận” \\ \hline
\textbf{Inputs} & Tutor: Nguyễn Thị B \\ \hline
\textbf{Expected Outputs} & 1. Hệ thống hiển thị “Hủy thành công” \\& 2. Cập nhật trạng thái slot tutor cũ \\
& 3. Nếu chọn đăng ký mới → hệ thống hiển thị danh sách tutor \\
& 4. Sinh viên có thể chọn tutor mới và hoàn tất đăng ký \\ \hline
\textbf{Testing environment} & Web – Windows 11 \\ \hline
\end{longtable}

\newpage
\subsubsection{Đặt lịch hẹn}

\begin{longtable}{|l|p{12cm}|}
\hline

\textbf{Test case} & UC08-TC01 — Đặt lịch hẹn thành công \\ \hline
\textbf{Test description} & Sinh viên đặt lịch hẹn với tutor dựa trên lịch rảnh và nội dung hỗ trợ hợp lệ \\ \hline
\textbf{Related screens} & Màn hình Đặt lịch hẹn \\ \hline
\textbf{Pre-conditions} & 1. Sinh viên đã đăng nhập \newline
2. Sinh viên đã chọn tutor \newline
3. Tutor đã cung cấp lịch rảnh \\ \hline
\textbf{Actions} & 1. Sinh viên mở chức năng “Đặt lịch hẹn” \newline
2. Hệ thống hiển thị lịch rảnh của tutor \newline
3. Sinh viên chọn khoảng thời gian phù hợp và nhập nội dung hỗ trợ \newline
4. Nhấn “Xác nhận” \\ \hline
\textbf{Inputs} & Tutor: Nguyễn Thị B, Ngày giờ: 15/11/2025 14:00-15:00, Nội dung: Hướng dẫn bài tập Nguyên lý ngôn ngữ lập trình \\ \hline
\textbf{Expected Outputs} & 1. Hệ thống lưu yêu cầu lịch hẹn ở trạng thái Chờ duyệt \newline
2. Tutor nhận thông báo yêu cầu mới \\ \hline
\textbf{Testing environment} & Web – Windows 11 \\ \hline
\end{longtable}

\newpage
\begin{longtable}{|l|p{12cm}|}
\hline
\textbf{Test case} & UC08-TC02 — Đặt lịch hẹn thiếu thông tin \\ \hline
\textbf{Test description} & Sinh viên không nhập nội dung \\ \hline
\textbf{Related screens} & Màn hình Đặt lịch hẹn \\ \hline
\textbf{Pre-conditions} & 1. Sinh viên đã đăng nhập \newline
2. Sinh viên đã chọn tutor \newline
3. Tutor đã cung cấp lịch rảnh \\ \hline
\textbf{Actions} & 1. Sinh viên mở chức năng “Đặt lịch hẹn” \newline
2. Không chọn thời gian hoặc không nhập nội dung \newline
3. Nhấn “Xác nhận” \\ \hline
\textbf{Inputs} & Ngày giờ: 15/11/2025 14:00-15:00; Nội dung: (trống) \\ \hline
\textbf{Expected Outputs} & 1. Hệ thống hiển thị thông báo lỗi và yêu cầu nhập lại \newline
2. Quay lại màn hình nhập thông tin \\ \hline
\textbf{Testing environment} & Web – Windows 11 \\ \hline
\end{longtable}
\newpage
\subsubsection{Phản hồi chất lượng buổi học}
% ================== TEST CASE 1 ==================
\begin{longtable}{|l|p{12cm}|}
\hline
\textbf{Test case} &
UC09-TC01 - Gửi phản hồi chất lượng buổi học thành công \\
\hline
\textbf{Test description} &
Sinh viên gửi phản hồi hợp lệ cho một buổi học đã diễn ra, hệ thống lưu phản hồi, cập nhật lịch sử \\
\hline
\textbf{Related screens} &
- Tab ``Phản hồi chất lượng''. \\
& - Màn hình ``Gửi phản hồi''. \\
\hline
\textbf{Pre-conditions} &
1. Sinh viên đã có ít nhất một buổi hẹn đã tham gia. \\
& 2. Tab ``Phản hồi chất lượng'' đang hiển thị và có buổi hẹn mà sinh viên đã tham gia \\
\hline
\textbf{Actions} &
1. Sinh viên nhấn nút ``Phản hồi chất lượng'' của buổi hẹn. \\
& 2. Hệ thống mở màn hình ``Gửi phản hồi'' cho buổi hẹn đó. \\
& 3. Sinh viên đánh giá số sao \\
& 4. Sinh viên nhập nội dung vào ô ``Nhận xét'', ví dụ: ``Giải bài mẫu chi tiết, dễ hiểu.''. \\
& 5. Sinh viên nhấn nút ``Gửi''. \\
& 6. Hệ thống kiểm tra dữ liệu hợp lệ (đã chọn sao và nhận xét không rỗng). \\
& 7. Hệ thống lưu phản hồi vào cơ sở dữ liệu. \\
& 8. Hệ thống cập nhật khung ``Lịch sử phản hồi'' với dòng phản hồi mới. \\
& 9. Hệ thống hiển thị thông báo gửi phản hồi thành công. \\
\hline
\textbf{Inputs} &
- Mức đánh giá: 5/5 sao. \\
& - Nhận xét: ``Giải bài mẫu chi tiết, dễ hiểu.''. \\
\hline
\textbf{Expected Outputs} &
- Phản hồi mới được lưu đúng sinh viên, đúng buổi hẹn \\
& - Khung ``Lịch sử phản hồi'' hiển thị nội dung đã đánh giá, ví dụ: ngày 2025-10-26, điểm 5/5, nội dung ``Giải bài mẫu chi tiết, dễ hiểu.''. \\
& - Sinh viên thấy thông báo gửi thành công, không có thông báo lỗi. \\
\hline
\textbf{Testing environment} &
Web \\
\hline
\end{longtable}
\newpage
% ================== TEST CASE 2 ==================
\begin{longtable}{|l|p{12cm}|}
\hline
\textbf{Test case} &
UC09-TC02 - Gửi phản hồi thiếu nội dung nhận xét \\
\hline
\textbf{Test description} &
Sinh viên chỉ chọn số sao, bỏ trống ô nhận xét rồi nhấn ``Gửi''; hệ thống phát hiện thiếu thông tin, hiển thị thông báo lỗi và không lưu phản hồi. \\
\hline
\textbf{Related screens} &
- Tab ``Phản hồi chất lượng''. \\
& - Màn hình ``Gửi phản hồi''. \\
\hline
\textbf{Pre-conditions} &
1. Sinh viên đã có ít nhất một buổi hẹn đã tham gia. \\
& 2. Tab ``Phản hồi chất lượng'' đang hiển thị và có buổi hẹn mà sinh viên đã tham gia \\
\hline
\textbf{Actions} &
1. Sinh viên nhấn ``Phản hồi chất lượng'' cho buổi hẹn. \\
& 2. Màn hình ``Gửi phản hồi'' hiển thị. \\
& 3. Sinh viên chọn số sao. \\
& 4. Sinh viên để trống ô ``Nhận xét''. \\
& 5. Sinh viên nhấn nút ``Gửi''. \\
\hline
\textbf{Inputs} &
- Mức đánh giá: 4/5 sao. \\
& - Nhận xét: để trống. \\
\hline
\textbf{Expected Outputs} &
- Hệ thống không lưu bất kỳ phản hồi nào vào cơ sở dữ liệu. \\
& - Hệ thống hiển thị thông báo lỗi, ví dụ: ``Vui lòng nhập nội dung phản hồi trước khi gửi.''. \\
& - Màn hình ``Gửi phản hồi'' vẫn giữ nguyên: 4 sao vẫn được chọn, ô ``Nhận xét'' vẫn trống để sinh viên nhập lại. \\
& - Khung ``Lịch sử phản hồi'' không thay đổi. \\
\hline
\textbf{Testing environment} &
Web \\
\hline
\end{longtable}

\newpage
% ================== TEST CASE 3 ==================
\begin{longtable}{|l|p{12cm}|}
\hline
\textbf{Test case} &
UC09-TC03 - Hủy gửi phản hồi chất lượng \\
\hline
\textbf{Test description} &
Sinh viên mở màn hình ``Gửi phản hồi'' nhưng nhấn ``Hủy'' thay vì ``Gửi''; hệ thống không lưu dữ liệu và quay về danh sách buổi hẹn. \\
\hline
\textbf{Related screens} &
- Tab ``Phản hồi chất lượng''. \\
& - Màn hình ``Gửi phản hồi''. \\
\hline
\textbf{Pre-conditions} &
1. Sinh viên đã có ít nhất một buổi hẹn đã tham gia. \\
& 2. Tab ``Phản hồi chất lượng'' đang hiển thị và có buổi hẹn mà sinh viên đã tham gia \\
\hline
\textbf{Actions} &
1. Sinh viên nhấn ``Phản hồi chất lượng'' cho buổi hẹn. \\
& 2. Màn hình ``Gửi phản hồi'' hiển thị. \\
& 3. Sinh viên có thể chọn số sao và/hoặc nhập vài từ vào ô ``Nhận xét'' (tuỳ ý). \\
& 4. Sinh viên nhấn nút ``Hủy'' thay vì ``Gửi''. \\
\hline
\textbf{Inputs} &
Có thể đã nhập một số dữ liệu tạm (ví dụ chọn 3/5 sao, nhận xét ngắn), nhưng chưa thực hiện gửi. \\
\hline
\textbf{Expected Outputs} &
- Hệ thống không lưu bất kỳ phản hồi nào vào cơ sở dữ liệu. \\
& - Hệ thống đóng màn hình ``Gửi phản hồi'' và quay lại tab ``Phản hồi chất lượng'' (danh sách buổi hẹn). \\
& - Khi mở lại màn hình ``Gửi phản hồi'' cho cùng buổi hẹn, khung ``Lịch sử phản hồi'' không có phản hồi mới. \\
\hline
\textbf{Testing environment} &
Web \\
\hline
\end{longtable}

\newpage
% ================== TEST CASE 4 ==================
\begin{longtable}{|l|p{12cm}|}
\hline
\textbf{Test case} &
UC09-TC04 - Lỗi hệ thống khi lưu phản hồi \\
\hline
\textbf{Test description} &
Trong lúc sinh viên gửi phản hồi hợp lệ, hệ thống gặp lỗi cơ sở dữ liệu hoặc lỗi backend; hệ thống báo thất bại, ghi log lỗi và không làm mất dữ liệu đã nhập trên form. \\
\hline
\textbf{Related screens} &
- Tab ``Phản hồi chất lượng''. \\
& - Màn hình ``Gửi phản hồi''. \\
\hline
\textbf{Pre-conditions} &
1. Tài khoản sinh viên tồn tại và đang hoạt động, đã đăng nhập. \\
& 2. Có buổi hẹn đã tham gia và hiển thị trong tab ``Phản hồi chất lượng''. \\
& 3. Môi trường kiểm thử được cấu hình mô phỏng lỗi khi lưu vào cơ sở dữ liệu. \\
\hline
\textbf{Actions} &
1. Sinh viên nhấn ``Phản hồi chất lượng'' cho buổi hẹn ngày 26/10/2025. \\
& 2. Màn hình ``Gửi phản hồi'' hiển thị. \\
& 3. Sinh viên chọn 5 sao. \\
& 4. Sinh viên nhập nhận xét, ví dụ: ``Buổi học rất hữu ích.''. \\
& 5. Sinh viên nhấn nút ``Gửi''. \\
& 6. Backend cố gắng lưu dữ liệu nhưng gặp lỗi cơ sở dữ liệu. \\
\hline
\textbf{Inputs} &
- Mức đánh giá: 5/5 sao. \\
& - Nhận xét: ``Buổi học rất hữu ích.''. \\
\hline
\textbf{Expected Outputs} &
- Hệ thống không tạo bản ghi phản hồi mới trong cơ sở dữ liệu. \\
& - Hệ thống hiển thị thông báo lỗi, ví dụ: ``Gửi phản hồi thất bại, vui lòng thử lại sau hoặc liên hệ bộ phận kỹ thuật.''. \\
& - Dữ liệu trên form không bị mất: 5 sao vẫn được chọn, nội dung nhận xét vẫn còn để sinh viên có thể thử gửi lại. \\
& - Lỗi được ghi log ở phía server để phục vụ việc theo dõi và khắc phục. \\
\hline
\textbf{Testing environment} &
Web \\
\hline
\end{longtable}
\newpage
% ================== TEST CASE: KHÔNG CHỌN SAO ==================
\begin{longtable}{|l|p{12cm}|}
\hline
\textbf{Test case} &
UC09-TC05 - Gửi phản hồi không chọn mức sao đánh giá \\
\hline
\textbf{Test description} &
Sinh viên chỉ nhập nội dung nhận xét, không chọn sao đánh giá rồi nhấn ``Gửi''; hệ thống phát hiện thiếu mức đánh giá, hiển thị thông báo lỗi và không lưu phản hồi. \\
\hline
\textbf{Related screens} &
- Tab ``Phản hồi chất lượng''. \\
& - Màn hình ``Gửi phản hồi''. \\
\hline
\textbf{Pre-conditions} &
1. Tài khoản sinh viên tồn tại, đang hoạt động và đã đăng nhập. \\
& 2. Có ít nhất một buổi hẹn đã tham gia, hiển thị trong tab ``Phản hồi chất lượng'' với nút ``Phản hồi chất lượng''. \\
\hline
\textbf{Actions} &
1. Tại tab ``Phản hồi chất lượng'', sinh viên nhấn nút ``Phản hồi chất lượng'' của buổi hẹn. \\
& 2. Hệ thống mở màn hình ``Gửi phản hồi'' cho buổi hẹn đó. \\
& 3. Sinh viên không chọn bất kỳ sao nào (mức đánh giá để trống). \\
& 4. Sinh viên nhập nhận xét vào ô ``Nhận xét'', ví dụ: ``Buổi học khá ổn.''. \\
& 5. Sinh viên nhấn nút ``Gửi''. \\
\hline
\textbf{Inputs} &
- Mức đánh giá: không chọn (trống). \\
& - Nhận xét: ``Buổi học khá ổn.''. \\
\hline
\textbf{Expected Outputs} &
- Hệ thống không lưu phản hồi vào cơ sở dữ liệu. \\
& - Hệ thống hiển thị thông báo lỗi, ví dụ: ``Vui lòng chọn mức đánh giá trước khi gửi.''. \\
& - Màn hình ``Gửi phản hồi'' vẫn giữ lại nội dung đã nhập trong ô ``Nhận xét'' để sinh viên không phải gõ lại. \\
& - Sau lỗi, nếu quay lại tab ``Phản hồi chất lượng'', khung ``Lịch sử phản hồi'' của buổi hẹn không có thêm phản hồi mới. \\
\hline
\textbf{Testing environment} &
Web \\
\hline
\end{longtable}
\newpage
\subsubsection{Xem danh sách buổi gặp mặt}
\begin{longtable}{|l|p{12cm}|}
\hline
\textbf{Test case} & UC-10-TC01 — Xem danh sách buổi gặp mặt \\ \hline
\textbf{Test description} & Sinh viên hoặc tutor xem danh sách các buổi gặp mặt đã đăng ký thành công hoặc đã được duyệt \\ \hline
\textbf{Related screens} & Màn hình Danh sách buổi gặp mặt \\ \hline
\textbf{Pre-conditions} & 1. Sinh viên/Tutor đã đăng nhập \newline
2. Nếu là sinh viên thì phải đã đăng ký tutor \newline
3. Có ít nhất một buổi gặp đã đăng ký / được tạo \\ \hline
\textbf{Actions} & 1. Mở chức năng “Danh sách buổi gặp mặt” \newline
2. Hệ thống hiển thị danh sách các buổi gặp \newline
3. Nhấn vào một buổi để xem chi tiết \newline
4. Hệ thống hiển thị thông tin chi tiết buổi đó \newline
5. Người dùng nhấn Quay lại / Back \newline
6. Hệ thống quay về danh sách buổi gặp mặt \\ \hline
\textbf{Inputs} & Chọn buổi gặp với Chủ đề: Công nghệ phần mềm \\ \hline
\textbf{Expected Outputs} & 1. Hệ thống hiển thị chi tiết buổi gặp \newline
2. Quay lại danh sách, các buổi gặp vẫn hiển thị đúng \\ \hline
\textbf{Testing environment} & Web – Windows 11 \\ \hline
\end{longtable}
\newpage
\subsubsection{Hủy buổi gặp mặt}
\begin{longtable}{|l|p{12cm}|}
\hline
\textbf{Test case} & UC-11-TC01 — Hủy buổi gặp mặt với lý do \\ \hline
\textbf{Test description} & Actor (Sinh viên/Tutor) hủy buổi gặp đã lên lịch và nhập lý do hủy \\ \hline
\textbf{Related screens} & Màn hình hủy đăng ký buổi gặp mặt / hủy buổi gặp mặt \\ \hline
\textbf{Pre-conditions} & 1. Actor đã đăng nhập \newline
2. Buổi gặp chưa bắt đầu \newline
3. Buổi gặp đã được đăng ký, lên lịch \\ \hline
\textbf{Actions} & 1. Actor vào mục hủy đăng ký buổi gặp mặt / hủy buổi gặp mặt \newline
2. Hệ thống hiển thị danh sách buổi gặp đã đăng ký / đã lên lịch \newline
3. Actor chọn một buổi gặp cụ thể \newline
4. Hệ thống hiển thị form nhập lý do hủy \newline
5. Actor nhập lý do và bấm Xác nhận \newline
6. Hệ thống cập nhật slot rảnh của Tutor và trạng thái buổi gặp \newline
7. Hệ thống gửi thông báo xác nhận hủy: \newline
a. Nếu actor là Sinh viên → gửi thông báo cho Tutor \newline
b. Nếu actor là Tutor → gửi thông báo cho Sinh viên và Tutor \\ \hline
\textbf{Inputs} & Actor: Sinh viên/Tutor, Buổi gặp: 15/11/2025 14:00-15:00 , Lý do hủy: “Bận đột xuất” \\ \hline
\textbf{Expected Outputs} & 1. Hệ thống hiển thị “Hủy thành công” \newline
2. Cập nhật trạng thái slot: \newline
• Nếu actor là Sinh viên → mở slot Tutor, buổi gặp bị hủy cho sinh viên \newline
• Nếu actor là Tutor → buổi gặp bị hủy cho cả Tutor và Sinh viên \newline
3. Gửi thông báo hủy cho actor liên quan \\ \hline
\textbf{Testing environment} & Web – Windows 11 \\ \hline
\end{longtable}
\newpage
\subsubsection{Xử lý yêu cầu đặt lịch hẹn}

\begin{longtable}{|l|p{12cm}|}
\hline
\textbf{Test case} & UC-18-TC01 - Tutor đồng ý lịch hẹn\\ \hline
\textbf{Test description} & Tutor chấp nhận (đồng ý) một yêu cầu đặt lịch hẹn đang chờ xử lý từ sinh viên. \\ \hline
\textbf{Related screens} & - Giao diện danh sách yêu cầu lịch hẹn\newline 
- Giao diện xem chi tiết lịch hẹn\\ \hline
\textbf{Pre-conditions} & - Tutor đã đăng nhập vào hệ thống \newline
- Giao diện danh sách lịch hẹn đang hiển thị\newline 
- Tồn tại ít nhất một lịch hẹn với trạng thái đang chờ xử lí \\ \hline
\textbf{Actions} & - Từ giao diện danh sách yêu cầu lịch hẹn, tutor chọn một lịch hẹn để xem chi tiết \newline
- Hệ thống hiển thị thông tin chi tiết của lịch hẹn bao gồm họ tên sinh viên, nội dung, thời gian, và hình thức\newline
- Tutor nhấn nút "Phê duyệt"\newline 
- Hệ thống xử lí yêu cầu và hiển thị thông báo "Yêu cầu đã được phê duyệt"\newline
- Tutor chọn "Quay lại" hệ thống quay lại giao diện danh sách lịch hẹn\\ \hline
\textbf{Inputs} & - Danh sách yêu cầu lịch hẹn\\ \hline
\textbf{Expected Outputs} & Hệ thống xác nhận "Yêu cầu đã được phê duyệt" và gửi thông báo đến sinh viên đó\\ \hline
\textbf{Testing environment} & Giao diện web chạy trên trình duyệt, và vai trò người dùng là Tutor \\ \hline
\end{longtable}
\newpage
\begin{longtable}{|l|p{12cm}|}
\hline
\textbf{Test case} & UC-18-TC02 - Tutor từ chối lịch hẹn\\ \hline
\textbf{Test description} & Tutor từ chối một yêu cầu đặt lịch hẹn đang chờ xử lý từ sinh viên. \\ \hline
\textbf{Related screens} & - Giao diện danh sách yêu cầu lịch hẹn\newline 
- Giao diện xem chi tiết lịch hẹn\newline
- Pop-up "Lý do từ chối"\\ \hline
\textbf{Pre-conditions} & - Tutor đã đăng nhập vào hệ thống \newline
- Giao diện danh sách lịch hẹn đang hiển thị\newline 
- Tồn tại ít nhất một lịch hẹn với trạng thái đang chờ xử lí \\ \hline
\textbf{Actions} & - Từ giao diện danh sách yêu cầu lịch hẹn, tutor chọn một lịch hẹn để xem chi tiết \newline
- Hệ thống hiển thị thông tin chi tiết của lịch hẹn bao gồm họ tên sinh viên, nội dung, thời gian, và hình thức\newline
- Tutor nhấn nút "Từ chối"\newline 
- Hệ thống hiển thị Pop-up "Lý do từ chối", yêu cầu nhập lý do.\newline
- Tutor nhập lý do vào ô text và chọn "Xác nhận từ chối"\newline
- Hệ thống xử lý yêu cầu và thông báo "Đã từ chối lịch hẹn" \newline
- Tutor chọn "Quay lại" hệ thống quay lại giao diện danh sách lịch hẹn\\ \hline
\textbf{Inputs} & - Danh sách yêu cầu lịch hẹn\\ \hline
\textbf{Expected Outputs} & Hệ thống xác nhận "Đã từ chối lịch hẹn" và gửi thông báo đến sinh viên đó\\ \hline
\textbf{Testing environment} & Giao diện web chạy trên trình duyệt, và vai trò người dùng là Tutor \\ \hline
\end{longtable}