Trong quá trình xây dựng bài tập lớn, nhóm có sử dụng các công cụ AI tạo sinh 
(Generative AI) nhằm hỗ trợ phân tích yêu cầu, phát hiện lỗi trong thiết kế và đề xuất 
giải pháp kỹ thuật.

Đầu tiên, trong giai đoạn phân tích yêu cầu, AI được sử dụng để đối chiếu lại các yêu cầu 
chức năng và phi chức năng mà nhóm đã xác định. Công cụ AI hỗ trợ chỉ ra các điểm 
bất hợp lý hoặc thiếu sót trong quá trình diễn giải yêu cầu so với mô tả 
đề bài. Nhờ đó, nhóm có thể điều chỉnh lại use-case, luồng hoạt động và các mối quan 
hệ giữa các tác nhân, giúp mô hình yêu cầu trở nên chặt chẽ và chính xác hơn.

Tiếp theo, trong quá trình thiết kế kiến trúc, AI được dùng để gợi ý cách tổ chức mô hình 
MVC theo tầng (layered architecture), làm rõ vai trò của từng lớp Controller, Service, 
Repository và Entity. Công cụ AI hỗ trợ đánh giá mức độ hợp lý của Class Diagram, 
gợi ý tách lớp, chuẩn hóa tên phương thức và điều chỉnh quan hệ giữa các lớp sao cho 
phù hợp với mô hình hướng đối tượng. Những gợi ý này giúp nhóm phát hiện sớm các 
mâu thuẫn thiết kế và chỉnh sửa sơ đồ UML kịp thời trước khi bước vào giai đoạn hiện 
thực hoá.

Trong giai đoạn triển khai mã nguồn, AI được sử dụng ở mức độ tham khảo, chủ yếu để 
xem ví dụ về cấu trúc REST Controller trong Spring Boot, cách viết tầng Service theo 
chuẩn DI (Dependency Injection), hoặc mẫu component và hook trong React. Những đoạn 
mã mẫu do AI cung cấp chỉ mang tính định hướng, nhóm hoàn toàn tự triển khai lại 
theo phong cách lập trình và cấu trúc dự án của riêng mình. Đồng thời, AI còn giúp nhóm 
phát hiện một số lỗi tiềm ẩn trong logic nghiệp vụ, ví dụ các trường hợp chưa xử lý 
exception, chưa kiểm tra trạng thái dữ liệu hoặc thiếu validate ở một số API.