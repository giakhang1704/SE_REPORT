\subsubsection{Đăng ký Tutor}

\begin{longtable}{|p{3.5cm}|p{9cm}|}

\hline
\textbf{Use Case Name} & Đăng ký Tutor \\ \hline
\endfirsthead

\hline
\textbf{Use Case Name} & Đăng ký Tutor (tiếp) \\ \hline
\endhead

\textbf{Actor(s)} & Sinh viên \\ \hline

\textbf{Description} &
Cho phép sinh viên đăng ký tham gia chương trình Tutor/Mentor và chọn hoặc được gợi ý Tutor phù hợp. \\ \hline

\textbf{Trigger} &
Sinh viên truy cập hệ thống và chọn chức năng “Đăng ký Tutor”. \\ \hline

\textbf{Pre–Condition(s)} &
- Sinh viên đăng nhập thành công. \newline
- Chương trình Tutor đang mở đăng ký. \\ \hline

\textbf{Post–Condition(s)} &
- Sinh viên được ghi nhận là đã đăng ký chương trình Tutor. \newline
- Hệ thống ghi nhận nhu cầu/ưu tiên, tạo hồ sơ ghép cặp Tutor–Sinh viên. \newline
- Thông báo gửi cho sinh viên, tutor và điều phối viên. \newline
- Cập nhật báo cáo và ghi log hệ thống. \\ \hline

\textbf{Normal Flow} &
1. Sinh viên vào chức năng Đăng ký Tutor. \newline
2. Hệ thống hiển thị giao diện cho phép sinh viên chọn lĩnh vực/môn học, thời gian rãnh cần hỗ trợ. \newline
3. Sinh viên điền thông tin đầy đủ và bấm tiếp theo. \newline
4. Hệ thống hiển thị danh sách tutor phù hợp. \newline
5. Sinh viên chọn Tutor (hoặc chọn chế độ hệ thống tự xếp Tutor). \newline
6. Sinh viên bấm xác nhận đăng ký. \newline
7. Hệ thống lưu dữ liệu đăng ký, gắn cặp Tutor–Sinh viên. \newline
8. Hệ thống gửi thông báo cho Tutor và sinh viên, đồng thời cập nhật trạng thái cho điều phối viên. \newline
9. Sinh viên không tiếp tục đăng ký, bấm kết thúc. \newline
10. Hệ thống thông báo “Chúc mừng bạn hoàn tất đăng ký”. Kết thúc usecase. \\ \hline

\textbf{Alternative Flow} &
3a. Sinh viên không điền đầy đủ thông tin nhưng bấm tiếp theo. \newline
3a1. Hệ thống hiển thị thông báo lỗi yêu cầu điền đầy đủ. \newline
3a2. Quay lại bước 2. \newline\newline
4a. Hệ thống thông báo không có tutor phù hợp. \newline
4a1. Hệ thống gửi thông báo đến điều phối viên và hiển thị “Chúng tôi sẽ nhắc bạn khi có kết quả hỗ trợ sắp xếp được tutor phù hợp trong vòng 1 tuần làm việc.”. \newline
4a2. Hệ thống đưa lựa chọn “Đăng ký môn khác”. \newline
4a3. Nếu sinh viên đồng ý quay lại bước 2. \newline
4a4. Nếu sinh viên không đồng ý, hệ thống quay lại màn hình chính. Kết thúc usecase. \newline\newline
5a. Sinh viên tự chọn tutor thủ công. \newline
5a1. Sinh viên bấm chọn tutor yêu thích từ danh sách. \newline
5a2. Quay về bước 6. \newline\newline
5b. Sinh viên để hệ thống tự động chọn Tutor. \newline
5b1. Hệ thống gán ngẫu nhiên sinh viên với một tutor phù hợp. \newline
5b2. Quay về bước 6. \newline\newline
6a. Sinh viên không xác nhận đăng ký, chọn kết thúc. \newline
6a1. Hệ thống quay về màn hình chính. \newline
6a2. Kết thúc usecase. \newline\newline
6b. Sinh viên không xác nhận đăng ký, chọn sửa đổi. \newline
6b1. Quay về bước 2. \newline\newline
9a. Sinh viên tiếp tục đăng ký. \newline
9a1. Quay về bước 2. \\ \hline

\textbf{Exception Flow} &
- Lỗi mạng trong quá trình đăng ký → Hệ thống hiển thị thông báo lỗi mạng, yêu cầu sinh viên thực hiện lại sau. \newline
- Đợt đăng ký Tutor đã kết thúc → Hệ thống hiển thị thông báo “Đợt đăng ký Tutor đã kết thúc” → Kết thúc Use Case. \newline
- Lỗi lưu dữ liệu đăng ký → Hệ thống hiển thị thông báo “Không lưu được dữ liệu, vui lòng thử lại” → Quay về bước 6 Normal Flow. \\ \hline

\end{longtable}