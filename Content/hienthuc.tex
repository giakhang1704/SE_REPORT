Trong giai đoạn hiện thực hoá hệ thống, nhóm lựa chọn kết hợp hai công nghệ chính: 
\textbf{Java Spring Boot} cho phía backend và \textbf{ReactJS} cho phía frontend. 
Kiến trúc tổng thể của hệ thống được triển khai theo mô hình MVC kết hợp với 
layered architecture, giúp tách biệt giao diện, lớp điều khiển, nghiệp vụ và truy cập dữ liệu. 
Việc sử dụng các công nghệ hiện đại này giúp hệ thống phát triển theo hướng mở rộng, dễ bảo trì 
và phù hợp với yêu cầu xây dựng RESTful API của dự án.

\subsubsection{Spring Boot (Backend)}

Spring Boot được nhóm lựa chọn vì khả năng hỗ trợ tạo ứng dụng web nhanh chóng, 
cấu hình tối giản và tích hợp tốt với các thư viện trong hệ sinh thái Spring. 

\textbf{Ưu điểm:}
\begin{itemize}
    \item Hỗ trợ mạnh mẽ cho mô hình phát triển kiến trúc theo tầng: controller--service--repository.
    \item Tự động cấu hình, giảm đáng kể thời gian setup môi trường.
    \item Tích hợp Spring Security và khả năng kết nối REST API tốt.
    \item Dễ mở rộng và dễ kiểm thử nhờ cấu trúc module rõ ràng.
\end{itemize}

\textbf{Nhược điểm:}
\begin{itemize}
    \item Chi phí khởi động ban đầu cao hơn so với một số framework nhẹ (như NodeJS).
    \item Khó khăn cho người mới bắt đầu.
    \item Khi dự án lớn dần, cấu trúc Spring có thể phức tạp nếu không quản lý tốt.
\end{itemize}

\subsubsection{ReactJS (Frontend)}

ReactJS được nhóm sử dụng để xây dựng giao diện người dùng. 
Việc chia nhỏ giao diện thành các component giúp hệ thống dễ phát triển, 
tái sử dụng mã nguồn và đồng thời tối ưu trải nghiệm người dùng. 
Kết hợp React với REST API từ Spring Boot cho phép hệ thống hoạt động linh hoạt 
và tách biệt frontend--backend rõ ràng.

\textbf{Ưu điểm:}
\begin{itemize}
    \item Mô hình component giúp tái sử dụng và bảo trì giao diện dễ dàng.
    \item Nhiều thư viện hỗ trợ (Axios, React Router, v.v.)
\end{itemize}

\textbf{Nhược điểm:}
\begin{itemize}
    \item Khá phụ thuộc vào các thư viện bên ngoài; dễ gây lỗi giao diện nếu thiếu thống nhất trong việc thiết kế từng thành phần.
    \item Khó khăn cho người mới bắt đầu.
\end{itemize}