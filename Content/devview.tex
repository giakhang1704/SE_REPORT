\begin{figure}[H]
    \centering
    \includegraphics[width=1\linewidth]{Image/Developmentview.png}
    \caption{Component Diagram cho hệ thống}
    \label{fig:placeholder}
\end{figure}


\subsubsection{Subsystem 1: Authentication}
\textbf{Mục tiêu.}
Subsystem \emph{Authentication} chịu trách nhiệm xác thực người dùng và
kiểm soát truy cập dựa trên vai trò (role). Mọi yêu cầu cần đăng nhập,
kiểm tra token hoặc kiểm tra quyền đều đi qua subsystem này.


\begin{itemize}
    \item \textbf{Các component nội bộ:}
    \begin{itemize}
        \item \texttt{SSOIntegration}: kết nối tới hệ thống HCMUT\_SSO và
              HCMUT\_DATACORE để xác thực và đồng bộ thông tin người dùng.
        \item \texttt{TokenManager}: tạo và kiểm tra tính hợp lệ của token đăng nhập.
        \item \texttt{RoleManager}: quản lý role/permission và cung cấp chức năng
              kiểm tra quyền cho các action nghiệp vụ.
        \item \texttt{SessionManager}: quản lý phiên làm việc (session) của người dùng,
              ví dụ gia hạn hoặc huỷ session khi logout.
    \end{itemize}
\end{itemize}


%-------------------------------------------------------
\subsubsection{Subsystem 2: StudentManager}

\textbf{Mục tiêu.}
Subsystem \emph{StudentManager} quản lý các chức năng phía sinh viên như:
đăng ký tutor, tra cứu tutor
phù hợp, xem thông tin cá nhân,...
\begin{itemize}
\item\textbf{Các component nội bộ:}
\begin{itemize}
  \item \texttt{StudentService}: xử lý các thao tác liên quan đến hồ sơ sinh viên
        và giao diện sinh viên (xem thông tin bản thân, trạng thái tham gia).
  \item \texttt{TutorRegistration}: tiếp nhận và xử lý yêu cầu đăng ký tutor
        từ sinh viên.
  \item \texttt{MatchingEngine}: thực hiện ghép cặp sinh viên--tutor, đề xuất Tutor
\end{itemize}
\end{itemize}

%-------------------------------------------------------
\subsubsection{Subsystem 3: TutorManager}

\textbf{Mục tiêu.}
Subsystem \emph{TutorManager} quản lý thông tin tutor, lịch rảnh và tiến độ
học tập của sinh viên,... do tutor phụ trách.
\begin{itemize}
\item\textbf{Các component nội bộ:}
\begin{itemize}
  \item \texttt{ScheduleManager}: quản lý lịch rảnh và lịch làm việc của tutor.
  \item \texttt{TutorService}: xử lý các thao tác cập nhật thông tin, lịch rảnh của tutor.
  \item \texttt{ProgressTracker}: lưu trữ và truy vấn tiến độ học tập của sinh viên qua các buổi tư vấn.
\end{itemize}
\end{itemize}


%-------------------------------------------------------
\subsubsection{Subsystem 4: Appointment}

\textbf{Mục tiêu.}
Subsystem \emph{Appointment} quản lý toàn bộ vòng đời của cuộc hẹn
và buổi tư vấn (scheduling \& consultation), đồng thời cung cấp dịch vụ
gửi thông báo chung cho toàn hệ thống.
\begin{itemize}
\item\textbf{Các component nội bộ:}
\begin{itemize}
  \item \texttt{AppointmentService}: xử lý đặt lịch, xem lịch, huỷ lịch hẹn giữa sinh viên và tutor.
  \item \texttt{ConsultationService}: quản lý các buổi tư vấn theo nhóm hoặc lớp,
        cho phép tạo, đăng ký và huỷ tham gia.
  \item \texttt{NotificationService}: gửi email hoặc thông báo trong hệ thống
        tới sinh viên, tutor và coordinator.
\end{itemize}
\end{itemize}

\subsubsection{Subsystem 5: Document}

\textbf{Mục tiêu.}
Subsystem \emph{Document} quản lý tài liệu học tập được sử dụng trong
quá trình tư vấn, bao gồm lưu trữ nội bộ và tích hợp với hệ thống thư viện.
\begin{itemize}
\item\textbf{Các component nội bộ:}
\begin{itemize}
  \item \texttt{DocumentService}: cung cấp CRUD tài liệu trong hệ thống nội bộ.
  \item \texttt{LibraryIntegration}: đóng vai trò adapter kết nối với external
        system HCMUT\_LIBRARY.
  \item \texttt{FileUploadManager}: xử lý và kiểm tra file upload
        (định dạng, kích thước, \ldots).
\end{itemize}
\end{itemize}



%-------------------------------------------------------
\subsubsection{Subsystem 6: FeedbackManager}

\textbf{Mục tiêu.}
Subsystem \emph{FeedbackManager} thu thập feedback từ sinh viên,
quản lý dữ liệu tiến độ phục vụ đánh giá và tính toán các chỉ số
chất lượng hoạt động tutor.
\begin{itemize}
\item\textbf{Các component nội bộ.}
\begin{itemize}
  \item \texttt{FeedbackService}: tiếp nhận, lưu trữ và cung cấp dữ liệu feedback.
  \item \texttt{ProgressRecordService}: ghi nhận và truy xuất lịch sử tiến độ
        (view progress history) phục vụ đánh giá.
  \item \texttt{EvaluationManager}: kết hợp feedback và tiến độ để tính
        điểm chất lượng và các thống kê đánh giá.
\end{itemize}
\end{itemize}

%-------------------------------------------------------
\subsubsection{Subsystem 7: Coordinator}

\textbf{Mục tiêu.}
Subsystem \emph{Coordinator} chịu trách nhiệm điều phối \emph{khung chương trình chung}
(common program framework) cho hoạt động tutor và tổng hợp, gửi các báo cáo
thống kê . 
\begin{itemize}
\item\textbf{Các component nội bộ:}
 \begin{itemize}
  \item \texttt{CoordinatorService}: cung cấp API chính dành cho vai trò coordinator.
  \item \texttt{CurriculumManager}: quản lý và cập nhật khung chương trình chung,
        bao gồm cấu trúc buổi tư vấn, các chủ đề và quy tắc áp dụng.
  \item \texttt{ReportGenerator}: thu thập số liệu từ các subsystem khác
        và tạo báo cáo tổng hợp.
\end{itemize}
\end{itemize}
