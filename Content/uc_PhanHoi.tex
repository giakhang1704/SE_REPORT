\subsubsection{Phản hồi chất lượng}
\begin{longtable}{|p{3.5cm}|p{9cm}|}
\hline
\textbf{Use Case Name} & Phản hồi chất lượng \\ \hline
\textbf{Actor(s)} & Sinh viên (Student) \\ \hline
\textbf{Description} & Sinh viên muốn gửi phản hồi về chất lượng buổi học hoặc chất lượng hỗ trợ của Tutor. \\ \hline
\textbf{Trigger} & Sau khi kết thúc buổi học, sinh viên chọn mục “Phản hồi chất lượng”. \\ \hline
\textbf{Pre–Condition(s)} &
Sinh viên đã đăng nhập vào hệ thống. \newline
Sinh viên đã tham gia ít nhất một buổi học với Tutor. \\ \hline
\textbf{Post–Condition(s)} &
Phản hồi được lưu trữ thành công trong hệ thống. \newline
Hệ thống gửi thông báo đến Tutor và ghi nhận phản hồi cho báo cáo tổng hợp. \\ \hline
\textbf{Normal Flow} &
1. Sinh viên truy cập mục “Phản hồi chất lượng”. \newline
2. Hệ thống hiển thị danh sách các buổi học đã tham gia. \newline
3. Sinh viên chọn buổi học cần phản hồi. \newline
4. Hệ thống hiển thị form phản hồi (mức độ hài lòng, nội dung chi tiết). \newline
5. Sinh viên nhập nội dung phản hồi và chọn mức độ đánh giá. \newline
6. Sinh viên nhấn “Gửi”. \newline
7. Hệ thống lưu phản hồi vào cơ sở dữ liệu. \newline
8. Hệ thống gửi thông báo xác nhận cho sinh viên. \\ \hline
\textbf{Alternative Flow} &
6a. Sinh viên không nhập nội dung phản hồi. \newline
6a1. Hệ thống hiển thị thông báo: “Vui lòng nhập nội dung phản hồi trước khi gửi”. \newline
6a2. Giao diện quay lại bước 4. \newline\newline
6b. Sinh viên chọn “Quay lại” thay vì gửi phản hồi. \newline
6b1. Hệ thống hủy thao tác và không lưu thông tin. \newline
6b2. Use case kết thúc sớm. \\ \hline
\textbf{Exception Flow} &
7a. Sau khi sinh viên nhấn “Gửi” (bước 6), hệ thống cố gắng lưu dữ liệu nhưng gặp lỗi cơ sở dữ liệu (DB error). \newline
7a1. Hệ thống hiển thị thông báo: “Gửi phản hồi thất bại, vui lòng thử lại sau hoặc liên hệ bộ phận kỹ thuật”. \newline
7a2. Hệ thống ghi log lỗi và giữ trạng thái chưa gửi phản hồi. \newline
7a3. Use case kết thúc không thành công. \\ \hline
\end{longtable}